\documentclass[12pt]{article}
%\usepackage[14pt]{extsizes}
%\usepackage[a5paper, textwidth=12cm, textheight=17cm, headheight=14pt]{geometry}
\usepackage[a4paper, left=2cm, right=2cm, top=2cm, bottom=2cm]{geometry}

% Ссылки.
\usepackage[unicode=true]{hyperref}

% Улучшенные сноски.
\usepackage{footmisc}

% Продвинутые формулы.
\usepackage{amsmath}

% Продвинутые математические символы.
\usepackage{amssymb}

% Кастомизируемые теоремы.
\usepackage{amsthm}
%\usepackage{thmtools}

% Русский язык.
\usepackage{cmap}
\usepackage[T2A]{fontenc}
\usepackage[utf8]{inputenc}
\usepackage[russian]{babel}

% Кастомизируемые хэдеры и футеры.
%\usepackage{fancyhdr}

% Табуляция перед первым параграфом.
%\usepackage{indentfirst}

% Нижнее подчёркивание с переносами.
\usepackage[normalem]{ulem}

% Графики gnuplot.
\usepackage[shell, subfolder, cleanup]{gnuplottex}

% Работа с плавающими объектами.
\usepackage[section]{placeins}

% Обтекаемые изображения
\usepackage{wrapfig}

% Ячейки на несколько строк.
\usepackage{multirow}
\usepackage{makecell}

% Таблица с регулируемой шириной столбцов и работающими сносками.
\usepackage{tabularx}

% Вращение.
\usepackage{rotating}
\usepackage{pdflscape}

% Элементы на следующей странице.
\usepackage{afterpage}

% Задачи.
\usepackage[lastexercise]{exercise}

% Сброс счётчика.
\usepackage{chngcntr}

% Графика TikZ
\usepackage{tikz}
%\usepackage{tikz-qtree}
%\usetikzlibrary{calc}
\usetikzlibrary{trees,calc,arrows.meta,positioning,decorations.pathreplacing,bending,matrix}
%\usetikzlibrary{trees}

% Таблицы Юнга.
\usepackage{ytableau}

% Enum и item на одной строке.
\usepackage[inline]{enumitem}
\setlist[1]{itemsep=0pt} % Величина разрыва между \item


    %%%%%%%%%

    % КОМАНДЫ

    %%%%%%%%%


% Математические символы и прочие дефайны.
%% Математические символы и прочие дефайны.

\def\defarr{\overset{\triangle}{\Longleftrightarrow}} % <<По определению>>
\def\defeq{\overset{\triangle}{=}}                    % <<По определению равно>>
\def\symdiff{\,\triangle\,}                           % <<Симметрическая разность>>
\def\connected{\leftrightsquigarrow}                  % Связность в графах.

% Математическое операторы.
\DeclareMathOperator{\diam}{\textnormal{diam}}
\DeclareMathOperator{\rad}{\textnormal{rad}}
\DeclareMathOperator*{\argmin}{\arg\min}
\DeclareMathOperator*{\argmax}{\arg\max}
\DeclareMathOperator*{\dom}{\textnormal{dom}}
\DeclareMathOperator*{\range}{\textnormal{range}}


% Настройки пакета с упражнениями.
%% Настройки пакета с упражнениями.

% Заголовок упражнения.
\renewcommand{\ExerciseHeader}
{
    \leftline
    {
        \textbf
        {
            %\large
            Задача
            \ExerciseHeaderNB \ExerciseHeaderDifficulty
            \textit{\ExerciseHeaderTitle{}}
            \ExerciseHeaderOrigin{}
            \vspace{0.25\baselineskip}
            %\smallskip{}
            %\textit{\ExerciseName{}}
        }
    }
}

% Заголовок ответа.
\renewcommand{\AnswerHeader}
{
    %\medskip
    \leftline
    {
        \textbf
        {
            Решение задачи \ExerciseHeaderNB{}
        }
        \vspace{0.25\baselineskip}
    }
}

% Пропуск после упражнения и ответа.
\setlength{\ExerciseSkipBefore}{1.0\baselineskip}
\setlength{\ExerciseSkipAfter}{1.0\baselineskip}
\setlength{\AnswerSkipBefore}{0.0\baselineskip}
\setlength{\AnswerSkipAfter}{1.0\baselineskip}

% Счётчик для упражнений внутри секции.
\newcounter{SecExercise}
%\counterwithin{SecExercise}{section}
%\counterwithin*{SecExercise}{subsection}

% Счётчик для бонусных задач.
\newcounter{BonusExercise}


% Картинки.
% Счётчики таблиц и фигур.
\counterwithin{table}{section}
\counterwithin{figure}{section}



% Специальные обозначения бкув.
%% Специальные обозначения букв.

\def\N{\mathbb{N}}
\def\Z{\mathbb{Z}}
\def\Q{\mathbb{Q}}
\def\R{\mathbb{R}}
\def\f{\mathcal{F}}
\def\l{\mathcal{L}}
\def\t{\mathcal{T}}
\def\I{\mathbb{I}}


% Разделы без номеров.
% Глава без номера.
\newcommand{\silentchapter}[1]{
    \chapter*{#1}
    \markboth{\MakeUppercase{#1}}{#1}
    \addcontentsline{toc}{chapter}{#1}
}

% Секция без номера.
\newcommand{\silentsection}[1]{
    \section*{#1}
    \markboth{\MakeUppercase{#1}}{#1}
    \addcontentsline{toc}{section}{#1}
}

% Подсекция без номера.
\newcommand{\silentsubsection}[1]{
    \subsection*{#1}
    \markboth{\MakeUppercase{#1}}{#1}
    \addcontentsline{toc}{subsection}{#1}
}



% Настройки пакета asmthm.
%% Настройки пакета asmthm.

% Счётчик теорем и прочего.
\newcounter{TheoremCounter}
\counterwithin{TheoremCounter}{section}
%\counterwithin*{TheoremCounter}{subsection}

% Счётчики таблиц и фигур.
\counterwithin{table}{section}
\counterwithin{figure}{section}


% Теоремы, определения, замечания и так далее.
\newtheorem{theorem}[TheoremCounter]{Теорема}
\newtheorem{lemma}[TheoremCounter]{Лемма}
\newtheorem{corollary}[TheoremCounter]{Следствие}
\newtheorem{definition}[TheoremCounter]{Определение}
\newtheorem{remark}[TheoremCounter]{Замечание}
\newtheorem{statement}[TheoremCounter]{Утверждение}
\newtheorem{problem}[TheoremCounter]{Задача}
\newtheorem{example}[TheoremCounter]{Пример}



% Правила вывода.
\newcommand{\typerule}[2]{%
    \begin{tabular}{c}
    $#1$ \\
    \hline
    %\midrule
    $#2$
    \end{tabular}%
}


% Inline item.
\makeatletter
\newcommand{\inlineitem}[1][]
{%
    \ifnum\enit@type=\tw@
        {\descriptionlabel{#1}}
        \hspace{\labelsep}%
    \else
        \ifnum\enit@type=\z@
        \refstepcounter{\@listctr}\fi
        \quad\@itemlabel\hspace{\labelsep}%
\fi}
\makeatother


% Выделение в определении
%\newcommand{\defemph}[1]{\textbf{\textit{#1}}}
\DeclareTextFontCommand{\defemph}{\bfseries\em}

% Нумерация русскими буквами.
\renewcommand{\alph}[1]{\asbuk{#1}}

% Пространство между плавающими объектами.
\setlength{\floatsep}{20pt}




\title{Алгебра логики, комбинаторика и теория графов: семинары}
\author{Бутаков~И.\,Д.}
\date{2022}

%\pagenumbering{gobble}

\begin{document}

\numberwithin{equation}{section}

\maketitle

\tableofcontents

\silentsection{Предисловие} \label{sec:intro}

Перед вами сборник всех семинаров по АЛКТГ за авторством Бутакова\,И.\,Д.
Автор выражает благодарность Маланчук Софии Владимировне за помощь в составлении материалов.

\silentsubsection{Используемые обозначения}

\begin{center}
    \begin{tabularx}{\textwidth}{cl}
        $ \defarr $  & <<\ldots по определению тогда и только тогда, когда \ldots>> \\
        $ \defeq $   & <<\ldots по определению равно \ldots>> \\
        $ \forall $  & <<Для любого \ldots>>\footnote{\label{footnote:quantifiers}Более точное определение дано в разделе \ref{sec:sets}.} \\
        $ \exists $  & <<Существует \ldots>>\footref{footnote:quantifiers} \\%\footnote{См. \ref{footnote:quantifiers}} \\
    \end{tabularx}
\end{center}
          % Введение.
\section{Алгебра логики} \label{sec:boolean}

\defemph{Математика}~--- это общий язык для многих сфер деятельности человека, позволяющий формализовать объекты, которыми оперирует наш разум.
Однним из важнейших разделов математики считается \defemph{математическая логика},
так как именно она является фундаментом математических суждений, исследует природу математического доказательства в целом.
Поэтому не зря курс дискретного анализа начинается именно с \defemph{алгебры логики}~---
объекта математической логики, хоть и сравнительно простого, но важного и полезного.



\subsection{Булева алгебра}
\label{subsec:boolean:algebra}


Алгебра логики изучает логические операции над высказываниями.
При этом в простейшем случае считается, что высказывания могут быть только истинными (обозначается~<<$ 1 $>>) или ложными (обозначается~<<$ 0 $>>);
%такие высказывания являются \defemph{булевозначными}.

\begin{definition}
    Переменные, принимающие значения только $ 0 $ или $ 1 $, называются \defemph{булевыми переменными}.
    Аналогично, функции от булевых переменных, принимающие только значения $ 0 $ или $ 1 $~--- \defemph{булевы функции}, или \defemph{логические операции}, \defemph{связки}.
    \defemph{Высказываниям} ставится в соответствие либо булева переменная с фиксированным значением, либо значение булевой функции на фиксированных аргументах.
\end{definition}

Примеры логических операций:
<<НЕ>> (обозначается~<<$ \neg $>>),
<<И>> (обозначается~<<$ \wedge $>>),
<<ИЛИ>> (обозначается <<$ \vee $>>),
исключающее <<ИЛИ>> (обозначается <<$ \oplus $>>),
импликация (обозначается <<$ \rightarrow $>>),
эквивалентность (обозначается <<$ \leftrightarrow $>>).

Заметим, что как и любые другие функции с конечной областью определения, булевы функции можно задать таблицей, просто перечислив их значения на всех возможных значениях аргументов.
Данные таблицы называются \defemph{таблицами истинности}.

\begin{table}[ht!]
    \center
    \begin{tabular}{|c|c|c|c|c|c|c|c|}
         \hline
         $ A $  &  $ B $  &  $ \neg A $  &  $ A \wedge B $  &  $ A \vee B $  &  $ A \oplus B $  &  $ A \rightarrow B $  &  $ A \leftrightarrow B $ \\
         \hline
         \hline
         $ 0 $  &  $ 0 $  &  $ 1 $       &  $ 0 $           &  $ 0 $         &  $ 0 $           &  $ 1 $                &  $ 1 $                   \\
         $ 0 $  &  $ 1 $  &  $ 1 $       &  $ 0 $           &  $ 1 $         &  $ 1 $           &  $ 1 $                &  $ 0 $                   \\
         $ 1 $  &  $ 0 $  &  $ 0 $       &  $ 0 $           &  $ 1 $         &  $ 1 $           &  $ 0 $                &  $ 0 $                   \\
         $ 1 $  &  $ 1 $  &  $ 0 $       &  $ 1 $           &  $ 1 $         &  $ 0 $           &  $ 1 $                &  $ 1 $                   \\
         \hline
    \end{tabular}
    \caption{задание логических связок таблицами истинности}
    \label{tab:boolean:truth_tables}
\end{table}

%\begin{Exercise}[counter=SecExercise, title=(КЗ №1)]
%    \noindent
%    Для какого слова \textbf{ложно} высказывание <<Первая буква слова гласная $ \rightarrow $ (Вторая буква слова гласная $ \vee $ Последняя буква слова гласная)>>?
%    \begin{enumerate}[label=\arabic*)]
%        \item жара;
%        \inlineitem орда;
%        \inlineitem огород;
%        \inlineitem парад;
%    \end{enumerate}
%\end{Exercise}
%
%\begin{Answer}
%    \noindent
%    Подставляя слова в элементарные высказывания, получаем
%    \begin{enumerate}
%        \item $ 0 \rightarrow (1 \vee 1) = 1 $;
%        \item $ 1 \rightarrow (0 \vee 1) = 1 $;
%        \item $ 1 \rightarrow (0 \vee 0) = 0 $;
%        \item $ 0 \rightarrow (1 \vee 0) = 1 $;
%    \end{enumerate}
%    Можно было сразу отсеять первый и четвёртый варинат по ложной посылке, а ответ найти по ложному следствию.
%\end{Answer}

%\shipoutAnswer

Запись таблицы истинности можно сократить, если изначально условиться, в каком порядке перечисляются возможные значения аргументов, и оставить только столбец значений функции;
этот столбец тогда будет являться \defemph{булевым вектором} (\defemph{вектором значений}), задающим функцию.
Стандартный порядок перечисления значений аргументов таков, чтобы они образовывали двоичную запись номера строки (см. таблицу \ref{tab:boolean:truth_tables}).

\begin{example}
    Операция <<НЕ>> задаётся булевым вектором $ 10 $, а, например, импликация~--- $ 1101 $.
\end{example}

Булевы функции можно также задавать при помощи формул, <<собирая>> из других связок.
Вообще говоря, формула является деревом, задающим порядок применения составляющих формулу связок к аргументам и к значениям других связок.
Однако такой взгляд на вещи полностью эквивалентен стандартному написанию математических формул, если задать приоритет операций, или просто использовать скобки.
Приоритет изученных связок указан в таблице \ref{tab:boolean:full_info}, пример формулы и соответствующего ей дерева~--- на рис. \ref{fig:boolean:tree_example}.

\begin{table}[ht!]
    \center
    \begin{tabular}{|c|c|c|c|c|c|}
        \hline
        Связка               & Приоритет & \makecell{Краткое \\ название}     & Название        & Смысл                       \\
        \hline
        \hline
        $ \neg $             & $ 1 $     & <<НЕ>>                             & отрицание       & <<не $ A $>>                \\
        $ \wedge $           & $ 2 $     & <<И>>                              & конъюнкция      & <<$ A $ и $ B $>>           \\
        $ \vee $             & $ 3 $     & <<ИЛИ>>                            & дизъюнкиця      & <<$ A $ или $ B $>>         \\
        $ \oplus $           & $ 3 $     & \makecell{<<ИСКЛИЛИ>>, \\ <<XOR>>} & исключающее или & <<либо $ A $, либо $ B $>>  \\
        $ \rightarrow $      & $ 4 $     & --                                 & импликация      & <<если $ A $, то $ B $>>    \\
        $ \leftrightarrow $  & $ 5 $     & --                                 & эквивалентность & <<$ A $ равносильно $ B $>> \\
        \hline
    \end{tabular}
    \caption{информация о логических связках}
    \label{tab:boolean:full_info}
\end{table}


\begin{figure}[ht!]
    \center
    \begin{tikzpicture}[level distance=1.5cm,
        level 1/.style={sibling distance=4cm},
        level 2/.style={sibling distance=2cm},
        level 4/.style={sibling distance=1cm}
        %edge from parent/.style={draw, edge from parent path={(\tikzparentnode) -- (\tikzchildnode)}}
    ]
    \tikzstyle{every node}=[circle,draw]

    \node (Root) {$ \leftrightarrow $}
        child {
            node {$ \rightarrow $}
            child {
                node {$ \vee $}
                child {
                    node {$ \neg $}
                    child { node[red] { $x_1 $} }
                }
                child {
                    node {$ \wedge $ }
                    child { node[red] { $ x_2 $ } }
                    child { node[red] { $ x_3 $ } }
                }
            }
            child { node[red] {$ x_4 $} }
        }
        child {
            node {$ \oplus $}
            child { node[red] {$ x_5 $} }
            child { node[red] {$ x_6 $} }
        };

    \end{tikzpicture}
    \caption{дерево формулы $ \neg x_1 \vee x_2 \wedge x_3 \rightarrow x_4 \leftrightarrow x_5 \oplus x_6 $
    (или, что то же самое, $ (((\neg x_1) \vee (x_2 \wedge x_3)) \rightarrow x_4) \leftrightarrow (x_5 \oplus x_6) $)}
    \label{fig:boolean:tree_example}
\end{figure}

\FloatBarrier



\subsection{Логические законы}
\label{subsec:boolean:laws}



\defemph{Логический закон} (\defemph{тождество})~--- равенство двух булевых функций, заданных разными формулами.
Равенство булевых функций определяется так же, как и равенство любых других функций:
требуется равенство областей определений функций и равенство значений функций в любой точке области определения.
Равенство формул и области определения, очевидно, влечет равенство функций; обратное не верно.
Принято также считать, что если аргументы в формуле пронумерованы, то число аргументов равно максимальному индексу (<<нет пропусков>>).

\begin{example}
    \label{example:boolean:formulas}
    \begin{itemize}
        \item[] % КОСТЫЛЬ! :(
        \item С точки зрения правил школьной математики, функции $ \sqrt[3]{x} $ и $ (x)^{1/3} $ не равны.
        \item Формула, задающая функцию трёх аргументов: $ x_1 \wedge x_3 $.
        \item Две формулы, задающие одну и ту же функцию: $ x_1 \rightarrow x_2 $ и $ \neg x_1 \vee x_2 $.
        \item Равенство областей определения \textbf{существенно}: пусть
            \[
                f(A,B,C) = A \vee B, \quad g(A,B) = A \vee B
            \]
            Данные две функции заданы одной и той же формулой, но $ f \neq g $.
    \end{itemize}
\end{example}

В случае булевых функций определение равенства можно записать иначе, если воспользоваться понятием \defemph{тавтологии}.

\begin{definition}
    Функция $ g $ называется \defemph{тавтологией} $ \defarr $ она тождественна равна единице на всей области определения.
\end{definition}

\begin{remark}
    Для булевых функций $ f $ и $ g $ их равенство эквивалентно равенству областей определений и тавтологичности $ f \leftrightarrow g $.
\end{remark}


Пользуясь определением связок из таблицы \ref{tab:boolean:truth_tables}, можно доказать множество логических законов. %(тождеств).
Самые тривиальные законы~--- \textit{коммутативность} и \textit{ассоциативность} операций $ \wedge $, $ \vee $ и $ \oplus $:
\[
    x_1 \wedge x_2 = x_2 \wedge x_1, \qquad
    x_1 \vee x_2   = x_2 \vee x_1, \qquad
    x_1 \oplus x_2 = x_2 \oplus x_1
\]
\[
    x_1 \wedge (x_2 \wedge x_3) = (x_1 \wedge x_2) \wedge x_3, \qquad
    x_1 \vee (x_2 \vee x_3)     = (x_1 \vee x_2) \vee x_3, \qquad
    x_1 \oplus (x_2 \oplus x_3) = (x_1 \oplus x_2) \oplus x_3
\]
Также имеет смысл выписать некоторые \textit{дистрибутивные} законы:
\[
    x_1 \wedge (x_2 \vee x_3) = (x_1 \wedge x_2) \vee (x_1 \wedge x_3)
\]
\[
    x_1 \vee (x_2 \wedge x_3) = (x_1 \vee x_2) \wedge (x_1 \vee x_3)
\]
Другие полезные законы:
\begin{enumerate}[label=\arabic*)]
    \item $ x_1 \vee (x_1 \wedge x_2) = x_1; \quad x_1 \wedge (x_1 \vee x_2) = x_1 $ \hspace*{\fill} \textit{(законы поглощения)}
    \item $ \neg (x_1 \wedge x_2) = \neg x_1 \vee \neg x_2; \quad \neg (x_1 \vee x_2) = \neg x_1 \wedge \neg x_2 $ \hspace*{\fill} \textit{(законы де Моргана)}
    \item $ x_1 \rightarrow x_2 = \neg x_2 \rightarrow \neg x_1 $ \hspace*{\fill} \textit{(закон контрапозиции)}
    %\item $ (x_1 \rightarrow x_2) \vee x_3 = x_1 \rightarrow (x_2 \vee x_3) $ \hspace*{\fill} \textit{(внос в следствие)}
    \item $ x_1 \leftrightarrow x_2 = \neg (x_1 \oplus x_2) = (x_1 \rightarrow x_2) \wedge (x_2 \rightarrow x_1) $
    \item $ x_1 \to x_2 = \neg x_1 \vee x_2 $ \label{item:implication_disjunction}
\end{enumerate}
Остальные законы и свойства можно найти в конспекте лекций Александра Александровича.
Отдельно отметим, что из пункта \ref{item:implication_disjunction} следует,
что импликация дистрибутивна относительно конъюнкции:
\[
    x_1 \rightarrow (x_2 \wedge x_3) = \neg x_1 \vee (x_2 \wedge x_3) = (\neg x_1 \vee x_2) \wedge (\neg x_1 \vee x_3) = (x_1 \rightarrow x_2) \wedge (x_1 \rightarrow x_3)
\]

\begin{Exercise}[counter=SecExercise, label={ex:boolean:chain}]
    \noindent
    Докажите равенство функций
    \[
        (x_1 \rightarrow x_2) \wedge (x_2 \rightarrow x_3) \wedge \ldots \wedge (x_{k-1} \rightarrow x_k) \wedge (x_k \rightarrow x_1) =
        \left( \bigwedge_{i=1}^{k-1} (x_i \rightarrow x_{i+1}) \right) \wedge (x_k \rightarrow x_1)
    \]
    и
    %\[
    %    (x_1 \leftrightarrow x_2) \wedge (x_2 \leftrightarrow x_3) \wedge \ldots \wedge (x_{k-1} \leftrightarrow x_k) \wedge (x_k \leftrightarrow x_1) =
    %    \left( \bigwedge_{i=1}^{k-1} (x_i \leftrightarrow x_{i+1}) \right) \wedge (x_k \leftrightarrow x_1)
    %\]
    \[
        x_1 \leftrightarrow x_2 \leftrightarrow x_3 \leftrightarrow \ldots \leftrightarrow x_k
    \]
\end{Exercise}

\begin{Answer}
    \noindent
    % Несложно заметить, что если $ \forall i, j \; x_i = x_j $, то оба высказывания истинны.
    % Если же $ \exists i,j: \; x_i \neq x_j $, то второе высказывание ложно.
    % Но что насчёт первого?
    % Понятно, что раз $ \exists i,j: \; x_i \neq x_j $, то и $ \exists m: \; x_m = 1 \neq 0 = x_{m+1} $
    % (считаем, что $ x_{k+1} = x_1 $, то есть нумерация зациклена).
    % Но тогда и первое высказывание ложно.
    %% ВЕРСИЯ БЕЗ КВАНТОРОВ %%
    Несложно заметить, что если при любых $ i, j $ выполнено $ x_i = x_j $, то оба высказывания истинны.
    Если же существуют $ i,j $ такие, что $ x_i \neq x_j $, то второе высказывание ложно.
    Но что насчёт первого?
    Понятно, что раз существуют $ i,j $ такие, что $ x_i \neq x_j $, то и существует $ m $ такое, что $ x_m = 1 \neq 0 = x_{m+1} $
    (считаем, что $ x_{k+1} = x_1 $, то есть нумерация зациклена).
    Но тогда и первое высказывание ложно.
\end{Answer}



\subsection{Преобразование и упрощение формул}
\label{subsec:boolean:transformation}



Преобразование формул при помощи одних тождеств позволяют находить другие.
Так при помощи привёдённых в предыдущем разделе законов можно доказать следующее утверждение:
\begin{statement}[Закон вноса в посылку и следствие]
    \label{statement:boolean:disjunction_implication}
    Верны тождества
    \[
        (x_1 \rightarrow x_2) \vee x_3 = x_1 \rightarrow (x_2 \vee x_3) = (x_1 \vee x_3) \rightarrow (x_2 \vee x_3)
    \]
\end{statement}

\begin{proof}
    Первое равенство доказывается тривиально:
    \[
        (x_1 \rightarrow x_2) \vee x_3 = \neg x_1 \vee x_2 \vee x_3 = \neg x_1 \vee (x_2 \vee x_3) = x_1 \rightarrow (x_2 \vee x_3)
    \]
    Перейдём ко второму.
    Для этого распишем импликацию через дизъюнкцию и воспользуемся правилом де Моргана:
    \[
        (x_1 \vee x_3) \rightarrow (x_2 \vee x_3) =
        \neg (x_1 \vee x_3) \vee x_2 \vee x_3 =
        (\neg x_1 \wedge \neg x_3) \vee x_2 \vee x_3
    \]
    Воспользовавшись дистрибутивностью дизъюнкции относительно конъюнкции, получаем
    \begin{multline*}
        (\neg x_1 \wedge \neg x_3) \vee x_2 \vee x_3 =
        \left[ (\neg x_1 \vee x_3) \wedge (\neg x_3 \vee x_3) \right] \vee x_2 = \\ =
        \left[ (\neg x_1 \vee x_3) \wedge 1 \right] \vee x_2 =
        \neg x_1 \vee x_3 \vee x_2 =
        x_1 \rightarrow (x_2 \vee x_3)
    \end{multline*}
\end{proof}

%При помощи доказанного факта решим следующую задачу:
\begin{Exercise}[counter=SecExercise, label={ex:boolean:disjunction_distributivity_over_equivalence}]
    \noindent
    Докажите дистрибутивность дизъюнкции относительно эквивалентности:
    \[
        x \vee (y \leftrightarrow z) = (x \vee y) \leftrightarrow (x \vee z)
    \]
\end{Exercise}

\begin{Answer}
    \noindent
    Пойдём от конца, начиная цепочку эквивалентных преобразований от правой формулы.
    Перепишем эквивалентность через две импликации:
    \[
        (x \vee y) \leftrightarrow (x \vee z) = \left[ (x \vee y) \rightarrow (x \vee z) \right] \wedge \left[ (x \vee z) \rightarrow (x \vee y) \right]
    \]
    Вынесем дизъюнкцию из посылки и следствия (см. утверждение \ref{statement:boolean:disjunction_implication}):
    \[
        \left[ x \vee \left( y \rightarrow z \right) \right] \wedge \left[ x \vee \left( z \rightarrow y \right) \right]
    \]
    Воспользуемся дистрибутивностью дизъюнкции относительно конъюнкции и <<вынесем $ x $ за скобки>>:
    \[
        x \vee \left( \left[ y \rightarrow z \right] \wedge \left[ z \rightarrow y \right] \right)
    \]
    Наконец, собирая обратно операцию эквивалентности, получаем искомую формулу:
    \[
        x \vee (y \leftrightarrow z)
    \]
    Заметим, что из связи дизъюнкции и импликации (см. тождество \ref{item:implication_disjunction}) также имеем
    \[
        x \rightarrow (y \leftrightarrow z) = (x \rightarrow y) \leftrightarrow (x \rightarrow z)
    \]
\end{Answer}


В последнем пункте примера \ref{example:boolean:formulas} видно, что значение $ f $ не зависит от значения $ C $.
В этом случае говорится, что $ C $~--- \defemph{несущественная} или \defemph{фиктивная переменная}.

\begin{definition}
    $ x_i $ ~--- \defemph{фиктивная переменная} функции $ f(x_1, \ldots, x_k) $ $ \defarr $ $ f \big|_{x_i = 0} $ и $ f \big|_{x_i = 1} $ равны как функции,
    то есть
    \[
        \forall x_1, \ldots, x_{i-1}, x_{i+1}, \ldots x_k \quad f(x_1, \ldots, x_{i-1}, 0, x_{i+1}, \ldots, x_k) = f(x_1, \ldots, x_{i-1}, 1, x_{i+1}, \ldots, x_k)
    \]
    Все нефиктивные переменные называются \defemph{существенными}.
\end{definition}


\begin{Exercise}[counter=SecExercise, label={ex:boolean:insignificant_variables}]
    \noindent
    Найдите все фиктивные переменные функции $ f $, заданной формулой
    \[
        f(x_1, \ldots, x_6) = (0 \wedge x_1) \vee \neg (1 \vee x_2) \vee \neg (0 \rightarrow x_3) \vee (x_4 \wedge x_5) \vee (x_4 \wedge (\neg x_5 \rightarrow x_6) \wedge \neg x_6)
    \]
\end{Exercise}

\begin{Answer}
    \noindent
    Первые три скобки, очевидно, равны нулю, поэтому $ x_1 $, $ x_2 $ и $ x_3 $~--- несущественные переменные, и формула упрощается:
    \[
        f(x_1, \ldots, x_6) = (x_4 \wedge x_5) \vee (x_4 \wedge (\neg x_5 \rightarrow x_6) \wedge \neg x_6)
    \]
    Заметим, что вторая скобка в новой формуле принимает истинное значение только тогда, когда $ x_4 = 1 $ и $ x_5 = 1 $,
    то есть, только тогда, когда первая скобка принимает истинное значение.
    Но тогда формула упрощается еще сильнее:
    \[
        f(x_1, \ldots, x_6) = x_4 \wedge x_5
    \]
    Отсюда $ x_6 $~--- еще однa несущественная переменная;
    все остальные переменные существенны.
\end{Answer}

%\shipoutAnswer



Из решения задачи выше можно извлечь и обобщить одну важную идею, позволяющую упрощать логические формулы
\begin{lemma}[Обобщённые правила поглощения]
    Пусть $ f $ и $ g $~--- булевы функции, определённые на общем наборе аргументов.
    Тогда
    \begin{enumerate}[label=\arabic*)]
        \item Если $ g = 1 $ только на тех значениях аргументов, на которых $ f = 1 $, то функции $ f \vee g $ и $ f $ равны.
        \item Если $ g = 0 $ только на тех значениях аргументов, на которых $ f = 0 $, то функции $ f \wedge g $ и $ f $ равны.
        \item Если $ g = 0 $ только на тех значениях аргументов, на которых $ f = 1 $, то функции $ f \rightarrow g $ и $ g $ равны.
    \end{enumerate}
\end{lemma}

\begin{proof}
    Докажем только первый пункт, так как остальные делаются аналогично.
    Требуется доказать тавтологичность высказывания
    %\[
    %    ((g \rightarrow f) \wedge (f \vee g)) \leftrightarrow f
    %\]
    %Эквивалентная формула:
    %\[
    %    ((\neg g \vee f) \wedge (g \vee f)) \leftrightarrow f
    %\]
    %Воспользовавшись дистрибутивностью, имеем
    %\[
    %    ((g \wedge \neg g) \vee f) \leftrightarrow f
    %\]
    %\[
    %    (0 \vee f) \leftrightarrow f
    %\]
    %\[
    %    f \leftrightarrow f
    %\]
    \[
        (g \rightarrow f) \rightarrow ((g \vee f) \leftrightarrow f)
    \]
    Перепишем формулу, используя только $ \neg $, $ \wedge $ и $ \vee $:
    \[
        \neg (\neg g \vee f) \vee ((\neg (g \vee f) \vee f) \wedge ((g \vee f) \vee \neg f))
    \]
    Применим правила де Моргана:
    \[
        (g \wedge \neg f) \vee (((\neg g \wedge \neg f) \vee f) \wedge ((g \vee f) \vee \neg f))
    \]
    Используя тавтологичность $ f \vee \neg f $, упрощаем формулу:
    \[
        (g \wedge \neg f) \vee (\neg g \wedge \neg f) \vee f
    \]
    Воспользовавшись дистрибутивностью, получаем
    \[
        (\neg f \wedge (g \vee \neg g)) \vee f
    \]
    \[
        (\neg f \wedge 1) \vee f
    \]
    \[
        \neg f \vee f
    \]
    Действительно, имеем тавтологию.
\end{proof}

Упрощение формул~--- важная задача, так как она облегчает проверку некоторых свойств задаваемой формулой функции, например,
тавтологичность или \defemph{выполнимость} (истинность хотя бы на каком-то наборе значений аргументов).
<<Трюки>> по упрощению формул активно используются в SAT-солверах~--- программах, проверяющих указанные выше свойства.

Здесь открывается еще одна важная сторона логики: возможность записать какие-то утверждения формально влечет возможность их алгоритмической проверки.
Например, можно написать программу, проверяющую верность теоремы, или по набору ограничений находящую верные утверждения.
%Также алгебра логики позволяет проще решать некоторые головоломки.

\begin{Exercise}[counter=SecExercise, label={ex:boolean:investigation}]
    \noindent
    Во время полуночного бала в поместье де Моргана произошло убийство.
    Убийца не оставил после себя никаких существенных улик, однако точно известно, что в момент преступления он был с жертвой наедине.
    В поместье есть две комнаты, доступные гостям: зал и библиотека.
    Гостями в ту ночь были Алиса, Боря, Вася, Гоша и Дима.

    Ниже приведены утверждения, в которых полиция уверена точно:
    \begin{enumerate}
        \item Жертвой стал Дима.
        \item Тело нашли в библиотеке.
        \item В зале в полночь точно была Алиса или Боря или Гоша.
        %\item Боря или Вася был в полночь в зале.
        \item Неверно, что либо Гоша, либо Боря были в библиотеке.
        \item Если Алиса была в зале, то и Боря тоже.
        \item Боря был в зале, или Вася был в библиотеке.
        \item Неверно, что Вася был в библиотеке, а Гоша был в зале.
        \item Алиса и Вася были в разных комнатах.
    \end{enumerate}
\end{Exercise}

\begin{Answer}
    \noindent
    Пусть $ x_i $~--- утверждение, что $ i $-ый гость был в библиотеке в момент убийства.
    Требуется найти аргументы, при которых истинно выражение
    \[
        x_5 \wedge (\neg x_1 \vee \neg x_2 \vee \neg x_4 ) \wedge \neg (x_2 \oplus x_4) \wedge
        (\neg x_1 \rightarrow \neg x_2) \wedge (\neg x_2 \vee x_3) \wedge \neg (x_3 \wedge \neg x_4) \wedge (x_1 \oplus x_3)
    \]
    Пользуясь тем, что $ \neg (a \oplus b) = a \leftrightarrow b = (a \rightarrow b) \wedge (b \rightarrow a) $, а также законами де Моргана и контрапозиции,
    получаем эквивалентную формулу:
    \[
        x_5 \wedge (\neg x_1 \vee \neg x_2 \vee \neg x_4 ) \wedge (x_2 \leftrightarrow x_4) \wedge (x_2 \rightarrow x_1) \wedge (x_2 \rightarrow x_3) \wedge
        (x_3 \rightarrow x_4) \wedge (x_1 \oplus x_3)
    \]
    Из задачи \ref{ex:boolean:chain} получаем следующую эквивалентную формулу:
    \[
        x_5 \wedge (\neg x_1 \vee \neg x_2 \vee \neg x_4 ) \wedge (x_2 \leftrightarrow x_3 \leftrightarrow x_4) \wedge (x_2 \rightarrow x_1) \wedge (x_1 \oplus x_3)
    \]
    Ей, в свою очередь, эквивалентна формула
    \[
        x_5 \wedge (\neg x_1 \vee \neg x_2 ) \wedge (x_2 \leftrightarrow x_3 \leftrightarrow x_4) \wedge (x_2 \rightarrow x_1) \wedge (x_1 \oplus x_2)
    \]
    Две последние скобки истинны только при $ x_2 = 0 $, $ x_1 = 1 $.
    Отсюда следующее упрощение:
    \[
        x_1 \wedge \neg x_2 \wedge x_5 (\neg x_1 \vee \neg x_2 ) \wedge (x_2 \leftrightarrow x_3 \leftrightarrow x_4)
    \]
    \[
        x_1 \wedge \neg x_2 \wedge \neg x_3 \wedge \neg x_4 \wedge x_5
    \]
    То есть, убийцей оказалась Алиса.
\end{Answer}


%\begin{Exercise}[counter=SecExercise]
%    \noindent
%    В одном из своих приключений расхитительница гробниц Лара наткнулась на дверь, ведущую к сокровищам.
%    Однако она оказалась запертой, да к тому же под охраной двух магических стражей.
%    Древняя надпись гласила, что ровно один страж сумеет открыть дверь, если его попросить; другой же на просьбу ответит оружием.
%    Позволялось задать ровно один вопрос любому из стражей, однако лишь один страж ответит на него верно; другой обязательно соврёт.
%    Какой вопрос надо задать, чтобы по ответу однозначно понять, кто из стражей открывает дверь?
%\end{Exercise}
%
%\begin{Answer}
%    \noindent
%\end{Answer}



\subsection{Булевы схемы}
\label{subsec:boolean:schemes}

Данная тема посвящена более глубокому изучению булевых функций и способов их задания.
Здесь мы формально рассмотрим булевы формулы, введём понятие булевой схемы.
Для всего этого потребуется материал следующих глав (множества, функции, ориентированные графы),
поэтому к данному разделу предлагается вернуться после изучения упомянутых тем.

Ранее уже говорилась, что булеву формулу можно интерпретировать как дерево,
задающее порядок применения логических связок.
Однако не было явно сказано, какие именно математические средства используются для <<задания>> и <<хранения>> этого порядка.
Решим эту проблему, дав формальное определение более общей структуры~--- \defemph{булевой схемы}.

\begin{definition}
    \label{definition:boolean:boolean_scheme}
    Зафиксируем некоторое множество булевых функций $ B $, которое далее будем называть \defemph{базисом}.
    \\[0.25\baselineskip]
    \defemph{Булевой схемой} в базисе $ B $ называется последовательность присвоений,
    которая начинается со всех переменных $ x_1, x_2, \ldots, x_n $, используемых схемой,
    и продолжается присваиваниями $ s_{n+1}, s_{n+2}, \ldots, s_m $,
    где каждое присваивание имеет вид $ s_i = g(s_{j_1} , s_{j_2}, \ldots, s_{j_k} ) $,
    где $ j_1, \ldots, j_k < i $, $ g \in B $;
    под $ s_1, \ldots, s_n $ понимаются переменные $ x_1, \ldots, x_n $.
    Булева схема задаёт булеву функцию $ f(x_1, \ldots, x_n) $,
    значение которой совпадает со значением последнего присваивания $ s_m $ (после проделанных вычислений).
\end{definition}

\begin{definition}
    \label{definition:boolean:standard_basis}
    Базис $ B = \{ \vee, \wedge, \neg \} $ называется \defemph{стандартным}.
\end{definition}

Говоря неформально, главное отличие схем от формул~--- в схемах возможно <<переиспользование>> промежуточных результатов.
Формализуем это.

\begin{definition}
    \label{definition:boolean:boolean_formula}
    \defemph{Булевой формулой} называется булева схема,
    в кторой ни одно присваивание, кроме переменных, не используется дважды в правой части присваиваний.
\end{definition}

\begin{statement}
    \label{statement:boolean:scheme_to_formula}
    Любую булеву схему можно преобразовать к формуле,
    разделив дважды и более используемые присваивания на несколько отдельных.
\end{statement}

Вернёмся к вопросу о связи формул и схем с их графическим представлением.
Держа в уме определения теории графов, можно сформулировать и доказать следующее утверждение:

\begin{statement}
    \label{statement:boolean:schemes_to_graphs}
    Если все функции базиса сохраняют своё значение при любой перестановке любого вектора аргументов
    (для бинарных операций это означает их коммутативность),
    то булеву схему можно представить в виде ориентированного ациклического графа.
    Вершинам соответствуют присваивания, рёбрам~--- подстановки в качестве аргументов;
    при этом топологическая сортировка порождает допустимый порядок присвоений.
    \\[0.25\baselineskip]
    Если булева схема является формулой,
    то соответствующий ей граф при удалении вершин-переменных и ориентации рёбер будет деревом.
\end{statement}

Упомянутое в утверждении условие на функции базиса важно,
так как входящие в вершину рёбра неразличимы между собой.

\begin{Exercise}[counter=SecExercise, label={exercise:boolean:graph_to_function}]
    \noindent
    Найдите функцию, которую вычисляет схема в стандартном базисе, представленная графически на рис. \ref{fig:boolean:scheme_example}.
\end{Exercise}

\begin{figure}[ht!]
    \center
    \begin{tikzpicture}[scale=1, transform shape]
        \tikzstyle{every node}=[circle,draw, minimum size=0.5cm]
        \node (s1)  at (0,0) {$ x_1 $};
        \node (s2)  at (0,2) {$ x_2 $};
        \node (s3)  at (0,4) {$ x_3 $};

        \node (s4)  at (2,0) {$ \vee $};
        \node (s5)  at (2,4) {$ \neg $};

        \node (s6)  at (4,0) {$ \wedge $};
        \node (s7)  at (4,2) {$ \wedge $};

        \node (s8)  at (6,4) {$ \wedge $};

        \node (s9)  at (8,2) {$ \vee $};

        \path [->] (s1) edge  (s4);
        \path [->] (s2) edge  (s4);
        \path [->] (s3) edge  (s5);

        \path [->] (s4) edge  (s6);
        \path [->] (s3) edge  (s6);
        \path [->] (s1) edge  (s7);
        \path [->] (s2) edge  (s7);

        \path [->] (s5) edge  (s8);
        \path [->] (s7) edge  (s8);

        \path [->] (s6) edge  (s9);
        \path [->] (s8) edge  (s9);
    \end{tikzpicture}
    \caption{Пример булевой схемы в стандартном базисе.}
    \label{fig:boolean:scheme_example}
\end{figure}

\begin{Answer}
    \noindent
    Идя от последнего присваивания вниз к переменным, выписываем формулу.
    Если какая-то вершина уже встречалось ранее в правой части присваивания,
    <<копируем>> соответствующую часть формулы.
    \[
        f(x_1, x_2, x_3) = \left[ (x_1 \vee x_2) \wedge x_3 \right] \vee \left[ (x_1 \wedge x_2) \wedge \neg x_3 \right]
    \]
    Можно заметить, что переиспользований не было;
    данная булева изначально ялвяется формулой.
    В качестве упражнения роверьте справедливость утверждения \ref{statement:boolean:schemes_to_graphs}.
\end{Answer}


\begin{Exercise}[counter=SecExercise, label={exercise:boolean:f_existance}]
    \noindent
    Существует ли такая булева функция $ f $ от двух переменных,
    что схема в базисе $ \{\wedge, f \} $
    \[
        x_1, \; x_2, \; s_3 = f(x_1, x_2), \; s_4 = f(x_2, x_1), \; s_5 = s_3 \wedge s_4
    \]
    вычисляет \textbf{а)} функцию $ x_1 $? \textbf{б)} функцию $ x_1 \oplus x_2 $?
\end{Exercise}

\begin{Answer}
    \noindent
    \begin{enumerate}[label=\textbf{\alph*)}]
        \item
            Заметим, что схема симметрична относительно перестановки $ x_1 $ и $ x_2 $.
            Из этого следует, что вычисляемая функция также должна быть симметрична относительно перестановки аргументов местами.
            Но для функции $ x_1 $ это неверно.
            Значит, ответ~--- нет.
        \item
            Да, $ f(x_1, x_2) = x_1 \oplus x_2 $.
            Тогда схема эквивалентна формуле
            \[
                f(x_1, x_2) \wedge f(x_2, x_1) = (x_1 \oplus x_2) \wedge (x_2 \oplus x_1) = x_1 \oplus x_2
            \]
    \end{enumerate}
\end{Answer}


\begin{Exercise}[counter=SecExercise, label={exercise:boolean:move_negation_to_arguments}]
    \noindent
    Докажите, что всякую булеву схему в стандартном базисе размера $ m $ с $ n $ переменными можно переделать в булеву схему размера не более $ 2 \cdot m $,
    вычисляющую ту же формулу, и в которой все отрицания применяются только к переменными.
\end{Exercise}

\begin{Answer}
    \noindent
    Приведём явный алгоритм данного преобразования.
    Для каждой схемы введём $ q $~--- суммарное число связок $ \vee $ и $ \wedge $,
    $ p $~--- число связок $ \neg $,
    а также $ k $~--- такое число,
    что все присваивания начиная с $ (k + 1) $-го гарантированно не являются отрицаниями.
    Изначально $ k $ положим равным $ m $, так как последнее присваивание может быть отрицанием.
    Также введём число $ d $, равное числу связок $ \vee $ и $ \wedge $, имеющих номер, не больший $ k $.

    Алгоритм построен так, что $ d $ и $ k $ не увеличиваются, $ q $ не меняется.
    Непосредственно описание алгоритма приведено ниже:
    \begin{enumerate}
        \item $ k \leftarrow m $, $ d \leftarrow q = m - n - p $.
        \item
            Если $ d = 0 $, то нет ни одного отрицания, применяемого к конъюнкции или дизъюнкции.
            Тогда алгоритм завершается.

            В противном случае рассмотрим $ k $-ое присвоение.
            Имеем следующие варианты, чем оно является:
            \begin{enumerate}
                \item[$ x_i $]
                    Реализация данного случая возможна,
                    только если в схеме не осталось отрицаний.
                    Таким образом, алгоритм завершается.
                \item[$ \vee $, $ \wedge $]
                    Данное присвоение не является результатом отрицания.
                    Можем уменьшить $ k $ и $ d $ на единицу.
                    Схему оставляем неизменной.
                \item[$ \neg $]
                    Рассмотрим, к чему применяется отрицание.
                    Имеем следующие варианты:
                    \begin{enumerate}
                        \item[$ \neg $]
                            Пользуясь тождеством $ \neg \neg x $, убираем два отрицания.
                            Числа $ p $ и $ k $ уменьшаются на два,
                            число остальных связок не меняется;
                            $ d $ также неизменно.
                        \item[$ \vee $, $ \wedge $]
                            Пользуясь законами де Моргана,
                            вносим отрицание в аргументы связки;
                            тем самым $ p $ увеличивается на единицу, $ k $ не меняется, $ d $ уменьшается на единицу.
                            Поскольку обязательно $ d \geqslant 0 $,
                            данный пункт реализуется не более $ q = const $ раз.
                        \item[$ x_i $]
                            Рассматриваемое отрицание уже применяется к переменной.
                            Изменим нумерацию в схеме так, чтобы выбранное отрицание стояло до любой из связок $ \vee $ или $ \wedge $
                            (если это уже так, ничего делать не надо).
                            Данный пункт реализуется не более $ m $ раз,
                            поскольку число таких отрицаний не превышает $ k \leqslant m $,
                            и каждое из них рассматривается единожды.
                    \end{enumerate}
            \end{enumerate}
        \item
            Перейти к предыдущему пункту.
    \end{enumerate}

    Алгоритм завершит свою работу, так как $ k $ не может быть отрицательным,
    а на каждом шаге оно не увеличивается;
    при этом число шагов, на которых $ k $ оставалось неизменным, ограничено.

    В ходе работы алгоритма $ q $ оставалось неизменным,
    а $ p $ либо уменьшалось на $ 2 $ за какие-то шаги,
    либо увеличивалось на $ 1 $ за шаги, когда $ k $-ое присвоение являлось отрицанием конъюнкции или дизъюнкции;
    второй вариант реализовался не более $ m $ раз.
    Таким образом, размер схемы увеличится не более, чем в два раза.
\end{Answer}



\subsection{Классы булевых функций}
\label{subsec:boolean:boolean_functions_classes}

В прошлом разделе мы ввели понятие булевой схемы, заданной некоторым базисом.
Связь описательной способности булевых схем с базисом очевидна:
если базис слишком <<беден>> (например, состоит из одного только отрицания),
далеко не все булевы функции представимы в виде схемы в выбранном базисе.
В данном разделе эта связь будет исследована полее подробно и формально.
Начнём с определений.


\begin{definition}
    \label{definition:boolean:all_boolean_functions}
    Определим\footnote{Это исключительно авторское обозначение. Не рекомендуется использовать без определения.}
    \defemph{множество всех булевых функций} от конечного числа переменных:
    $ \boolfun \defeq \bigcup_{k=0}^\infty \{0, 1\}^k \times \{0, 1\} $.
\end{definition}


\begin{definition}
    \label{definition:boolean:closure}
    \defemph{Замыканием} $ \closure (B) $ некоторого базиса $ B \subseteq \boolfun $ будем называть множество булевых функций,
    представимых в виде схемы в данном базисе.
    \\[0.25\baselineskip]
    Множество $ B $ булевых функций называется \defemph{замкнутым} в случае $ B = \closure (B) $.
\end{definition}


\begin{statement}
    \label{statement:boolean:closure_is_closed}
    Замыкание любого множества функций замкнуто.
    Иначе говоря, $ \forall B \subseteq \boolfun \;\, \closure(B) = \closure(\closure(B)) $.
\end{statement}


\begin{definition}
    \label{definition:boolean:complete_class}
    Множество $ B $ булевых функций называется \defemph{полным} в случае $ \closure (B) = \boolfun $.
\end{definition}


Оказывается, один полный базис нам уже известен.
Это стандартный базис.
Для того, чтобы доказать его полноту, введём несколько впомогательных определений.

\begin{definition}
    \label{definition:boolean:literal}
    \defemph{Литералом} называют переменную или отрицание переменной.
    Введём следующие обозначения для литералов: $ x^0 = \neg x $, $ x^1 = x $.
\end{definition}

\begin{definition}
    \label{definition:boolean:clause}
    \defemph{Конъюнктом} и \defemph{дизъюнктом} называют конъюнкцию и дизъюнкцию литералов соответственно:
    \[
        C = l_1 \wedge l_2 \wedge \ldots \wedge l_n, \qquad D = l_1 \vee l_2 \vee \ldots \vee l_m
    \]
\end{definition}


\begin{remark}
    \label{remark:boolean:clause_true_false}
    Конъюнкт $ x_1^{\alpha_1} \wedge x_2^{\alpha_2} \wedge \ldots \wedge x_n^{\alpha_n} $ принимает истинное значение только на наборе $ x_i = \alpha_i $.
    Дизъюнкт $ x_1^{\alpha_1} \vee x_2^{\alpha_2} \vee \ldots \vee x_m^{\alpha_m} $ принимает ложное значение только на наборе $ x_i = \neg \alpha_i $.
\end{remark}


\begin{definition}
    \label{definition:DNF_CNF}
    Формулы вида
    \[
        \bigvee_{(\alpha_1, \ldots, \alpha_n) \in \mathcal{A}} x_1^{\alpha_1} \wedge x_2^{\alpha_2} \wedge \ldots \wedge x_n^{\alpha_n}
        \qquad
        \bigwedge_{(\alpha_1, \ldots, \alpha_n) \in \mathcal{A}} x_1^{\alpha_1} \vee x_2^{\alpha_2} \vee \ldots \vee x_n^{\alpha_n}
    \]
    называются \defemph{дизъюнктивной} и \defemph{конъюнктивной нормальными формами} соответственно.
    Здесь $ \mathcal{A} \subseteq \{0, 1\}^n $.
\end{definition}


\begin{theorem}
    \label{theorem:boolean:standard_basis_complete}
    Стандартный базис является полным, то есть $ \closure(\{\vee, \wedge, \neg\}) = \boolfun $.
\end{theorem}

\begin{proof}
    Пусть имеется произвольная булева функция $ f: \{0, 1\}^n \rightarrow \{0, 1\} $.
    Заметим, что
    \[
        f(x_1, \ldots, x_n) = \bigvee_{(\alpha_1, \ldots, \alpha_n) \in f^{-1}(1)} x_1^{\alpha_1} \wedge x_2^{\alpha_2} \wedge \ldots \wedge x_n^{\alpha_n}
    \]
    Действительно, согласно замечанию \ref{remark:boolean:clause_true_false},
    каждый конъюнкт принимает истинное значение только на одном наборе значений переменных.
    По построению этот набор является выполняющим набором для функции $ f $.
    Также из определения полного прообраза $ f^{-1}(1) $ следует, что все подобные наборы были рассмотрены.
\end{proof}

Для доказательства было использовано разложение в ДНФ.
Точно также можно было бы представить функцию и в виде КНФ.

\begin{Exercise}[counter=SecExercise, label={exercise:boolean:DNF_example}]
    \noindent
    Разложите в ДНФ и КНФ булеву функцию, заданную вектором
    значений: $ f(x_1, x_2, x_3) = 10100101 $.
\end{Exercise}

\begin{Answer}
    \noindent
    Облегчим себе жизнь, заметив, что $ x_2 $~--- фиктивная переменная.
    Пользуясь алгоритмом из теоремы \ref{theorem:boolean:standard_basis_complete},
    \[
        f(x_1, x_2, x_3) = (\neg x_1 \wedge \neg x_3) \vee (x_1 \wedge x_3) = (\neg x_1 \vee x_2) \wedge (x_1 \vee \neg x_2)
    \]
\end{Answer}


\begin{corollary}
    \label{corollary:boolean:substandard_basis_complete}
    Базисы $ \{ \vee, \neg \} $ и $ \{ \wedge, \neg \} $ полны.
\end{corollary}

\begin{proof}
    Следует из законов де Моргана и полноты стандартного базиса
\end{proof}


\begin{corollary}
    \label{corollary:boolean:Zhegalkin_basis_complete}
    Базис $ \{ 1, \oplus, \wedge \} $ полон.
\end{corollary}

\begin{proof}
    Аналогично, используя $ \neg x = 1 \oplus x $.
\end{proof}


\begin{definition}
    \label{definition:boolean:Zhegalkin_polynomials}
    Формулы, заданные в базисе $ \{1, \oplus, \wedge \} $, называются \defemph{полиномами Жегалкина}.
\end{definition}

Следствия выше демонстрируют один из способов доказательства полноты базиса, а именно сведение к стандартному базису.
Однако открытым остаётся вопрос, как доказывать неполноту.
Оказывется, для этого есть специальная теорема.
Прежде чем к ней перейти, определим так называемые \defemph{классы Поста}:

\begin{definition}
    \label{definition:boolean:Post_classes}
    \defemph{Классами Поста} называются следующие пять множеств булевых функций:
    \begin{enumerate}
        \item[($ M $)] \defemph{Монотонные функции}:
            \[
                M = \{ f \in \boolfun \mid \forall x_1, x_2 \in \{0, 1\}^n \;\, (x_1 \leqslant x_2) \rightarrow (f(x_1) \leqslant f(x_2)) \},
            \]
            где порядок на $ \{0, 1\}^n $ берётся покоординатный.
        \item[($ L $)] \defemph{Линейные функции}: $ L = \closure(\{1, \oplus\}) $.
        \item[($ T_0 $)] \defemph{Функции, сохраняющие ноль}: $ T_0 = \{ f \in \boolfun \mid f(0, \ldots, 0) = 0 \} $.
        \item[($ T_1 $)] \defemph{Функции, сохраняющие единицу}: $ T_1 = \{ f \in \boolfun \mid f(1, \ldots, 1) = 1 \} $.
        \item[($ S $)] \defemph{Самодвойственные функции}: $ M = \left\{ f \in \boolfun \mid \neg f(\neg x_1, \ldots, \neg x_n) = f(x_1, \ldots, x_n) \right\} $
    \end{enumerate}
\end{definition}

\begin{theorem}[Поста]
    \label{theorem:boolean:Post}
    Замкнутое множество булевых функций является неполным тогда и только тогда, когда содержится в одном из классов Поста.
\end{theorem}


\begin{Exercise}[counter=SecExercise, label={exercise:boolean:preserves_0_1}]
    \noindent
    Сколько имеется булевых функций от $ n $ переменных,
    сохраняющих одновременно <<$ 0 $>> и <<$ 1 $>>?
\end{Exercise}

\begin{Answer}
    \noindent
    Условие сохранения <<$ 0 $>> и <<$ 1 $>> <<блокирует>> первую и последнюю из $ 2^n $ строк в таблице истинности;
    на остальных же наборах переменных данные функции могут принимать произвольные значения.
    Тогда ответ~--- $ 2^{2^n - 2} $.
\end{Answer}


\begin{Exercise}[counter=SecExercise, label={exercise:boolean:Post_example}]
    \noindent
    Являются ли полными следующие базисы?
    При отрицательном ответе укажите, в каких из классов $ T_0 $, $ T_1 $, $ M $, $ L $, $ S $ лежит замыкание базиса.
    \begin{enumerate}[label=\textbf{\alph*)}]
        \item $ \{ \neg, \rightarrow \} $, где $ \rightarrow $~--- импликация;
        \item $ \{ \downarrow \} $, где $ x \downarrow y $ равна $ \neg x \wedge \neg y $ (стрелка Пирса);
        \item $ \{ \wedge, \vee, \setminus \} $, где $ x \setminus y $ равна $ x \wedge \neg y $;
        \item $ \{ 1, \oplus \} $;
        \item $ \{ \neg, \leftrightarrow \} $, где $ x \leftrightarrow y $ равна $ (x \rightarrow y) \wedge (y \rightarrow x) $.
    \end{enumerate}
\end{Exercise}

\begin{Answer}
    \noindent
    \begin{enumerate}[label=\textbf{\alph*)}]
        \item
            Базис полный, так как $ \neg x \rightarrow y = x \vee y $,
            а потому $ \closure(\{\neg, \rightarrow\}) = \closure(\{\neg, \vee\}) = \boolfun $.
        \item
            Базис полный, так как $ x \downarrow x = \neg x $,
            $ \neg x \downarrow \neg y = x \wedge y $,
            а потому $ \closure(\{ \downarrow \}) = \closure(\{\neg, \downarrow\}) = \closure(\{\neg, \wedge\}) = \boolfun $.
        \item
            Нет, данный базис неполный: все функции сохраняют ноль.
            При этом $ \wedge \notin L, S $; $ \setminus \notin T_1, M $.
            Таким образом, базис целиком лежит только в одном классе: $ \{\wedge, \vee, \setminus \} \subseteq T_0 $.
        \item
            Нет, данный базис неполный: все функции линейны.
            При этом $ 1 \notin T_0, S $; $ \oplus \notin M, T_1 $.
        \item
            Нет, данный базис неполный: все функции линейны: $ \neg x = 1 \oplus x $, $ x \leftrightarrow y = 1 \oplus x \oplus y $.
            При этом $ \neg \notin T_0, T_1 $; $ \leftrightarrow \notin S, M $.
    \end{enumerate}
\end{Answer}


\begin{Exercise}[counter=SecExercise, label={exercise:boolean:linear_closure}]
    \noindent
    Постройте замыкание базиса $ \{\neg, \oplus\} $.
\end{Exercise}

\begin{Answer}
    \noindent
    Заметим, что $ 1 = \neg (x \oplus x) $ и $ \neg x = 1 \oplus x $.
    Отсюда следует, что $ \closure(\{\neg, \oplus\}) \subseteq \closure(\{1, \oplus\}) $ и $ \closure(\{1, \oplus\}) \subseteq \closure(\{\neg, \oplus\}) $.
    Значит, $ \closure(\{\neg, \oplus\}) = \closure(\{1, \oplus\}) = L $.
\end{Answer}

\begin{Exercise}[counter=SecExercise, label={exercise:boolean:monotonous_and_MAJ}]
    \noindent
    Назовём функцией большинства $ \text{MAJ}(x_1, x_2,\ldots , x_n) $ булеву функцию,
    значение которой совпадает с тем значением,
    которое принимает большинство переменных (если мнения разделились поровну, $ \text{MAJ} = 0 $).
    Схемы в базисе $ \{\vee, \wedge, 1, 0 \} $ называются монотонными.
    Вычисляется ли $ \text{MAJ} $ монотонной схемой?
\end{Exercise}

\begin{Answer}
    \noindent
    Да, вычисляется.
    Заметим, что
    \[
        \text{MAJ}(x_1, \ldots, x_n) = \bigvee_{I \in \binom{\{1, \ldots, m\}}{\lfloor n/2 \rfloor + 1}} \bigwedge_{i \in I} x_i
    \]
    Действительно, любой набор значений $ x_1, \ldots, x_n $, в котором единиц большинство, выполняет хотя бы один конъюнкт справа.
    Также любой набор значений входных переменных, выполняющий правую формулу, содержит более $ n/2 $ единиц,
    из чего следует выполнение на нём же и $ \text{MAJ} $.
    Таким образом, получена ДНФ без отрицаний, что и требовалось по условию.
\end{Answer}


\begin{Exercise}[counter=SecExercise, label={exercise:boolean:self_dual}]
    \noindent
    \begin{enumerate}[label=\textbf{\alph*)}]
        \item
            Являются ли самодвойственными функции $ x_1 \vee x_2 $, $ x_1 \wedge x_2 $?
        \item
            Докажите, что схема в базисе, состоящем из самодвойственных функций,
            вычисляет самодвойственную функцию.
    \end{enumerate}
\end{Exercise}

\begin{Answer}
    \noindent
    \begin{enumerate}[label=\textbf{\alph*)}]
        \item
            Нет: $ \neg(\neg x_1 \vee \neg x_2) = x_1 \wedge x_2 \neq x_1 \vee x_2 $.
        \item
            Эквивалентное определение самодвойственности:
            \[
                \neg f(x_1, \ldots, x_n) = f(\neg x_1, \ldots, \neg x_n)
            \]
            Для начала перейдём от схемы к формуле.
            Далее рассмотрим отрицание формулы и согласно эквивалентному определению самодвойственности будем <<спускать>> отрицания до тех пор,
            пока не дойдём до аргументов.
            Получим, что отрицание формулы равно формуле от отрицаний аргументов.
            Это эквивалентно самодвойственности.
    \end{enumerate}
\end{Answer}


\begin{Exercise}[counter=SecExercise, label={exercise:boolean:not_self_dual}]
    \noindent
    Пусть $ f(x_1, \ldots, x_n) $~--- несамодвойственная функция.
    Докажите, что константы $ 0 $, $ 1 $ вычисляются в базисе $ \{\neg, f \} $.
\end{Exercise}

\begin{Answer}
    \noindent
    Раз $ f $ не самодвойственная,
    $ \exists x_1, \ldots x_n: \neg f(\neg x_1, \ldots, \neg x_n) \neq f(x_1, \ldots, x_n) $.
    Обозначим любой их таких наборов значений как $ \{ \alpha_i \} $.
    Тогда заметим, что $ f(x^{\alpha_1}, \ldots, x^{\alpha_n}) = const $.
    Действительно, если это не так, то либо $ f(x^{\alpha_1}, \ldots, x^{\alpha_n}) = x $,
    либо $ f(x^{\alpha_1}, \ldots, x^{\alpha_n}) = \neg x $
    (других булевых функций одной переменной нет).
    Но $ f_1(x) = x $ и $ f_2(x) = \neg x $ самодвойственны.
    Тогда $ f(1^{\alpha_1}, \ldots, 1^{\alpha_n}) = \neg f(0^{\alpha_1}, \ldots, 0^{\alpha_n}) $
    что противоречит выбору $ \alpha_i $.
    Значит $ f(x^{\alpha_1}, \ldots, x^{\alpha_n}) = const $,
    а другую константу можно получить, взяв отрицание.
\end{Answer}
        % Булева алгебра.
\section{Множества и логика}
\label{sec:sets}

\defemph{Множество}~--- ещё один низкоуровневый объект математики, который трудно определить строго.
Формально, в наше время множеством называют объект, удовлетворяющий определённой системе аксиом\footnote{Аксиоматическая теория множеств, система аксиом ZF.}.
Однако в ходе этого курса так глубоко копать не придётся: нам будет достаточно лишь сравнительно неформального определения множества.

\subsection{Теория множеств}
\label{subsec:sets:theory}

\begin{definition}
    \defemph{Множеством} называют совокупность неповторяющихся объектов без указания отношения между ними.
\end{definition}

Для краткости утверждение <<элемент $ x $ принадлежит множеству $ A $>> обозначают как $ x \in A $.
Аксиоматически полагается существование \defemph{пустого множества} (то есть, множества, которому не принадлежит ни один элемент), которое обозначают $ \varnothing $.

Множества можно задавать различными способами:
\begin{enumerate}[label=\arabic*)]
    \item
        Перечислением всех элементов: $ A = \{1, 2, 4\} $.
    \item
        Перечисление посредством правила:
        \[
            \N_0 = \{ 0, 1, 2, \ldots \}, \qquad \Z = \{\ldots, -2, -1, 0, 1, 2, \ldots \} = \{0, 1, -1, 2, -2, \ldots \}
        \]
        \textit{Понятно, что данный способ неприменим, если из перечисления читателю угадать правило невозможно.}
    \item
        \label{itm:sets:condition}
        Задание условием:
        \[
            \{ x \mid f(x) \} = \text{<<множество всех $ x $, для которых верно высказывание $ f(x) $>>}
        \]
        Пример:
        \begin{equation}
            \label{eq:sets:example_formula}
            [0; 1) = \{ x \mid (x \geqslant 0) \wedge (x < 1) \}
        \end{equation}
        \[
            \Q = \left\{ \frac{m}{n} \, \middle | \, m \in Z, \; n \in \N \right\}
        \]
\end{enumerate}

Последний способ вызвает некоторые вопросы; в частности, какие $ x $ <<пойдут на вход>> условию $ f(x) $.
Верно ли, например, что $ \sqrt{2} \in \{ x \mid \text{<<$ x $ не делится на $ 2 $>>} \} $?
А то, что $ \text{``стул''} \in \{ x \mid \text{<<$ x $ не делится на $ 2 $>>} \} $?
Для того, чтобы обойти эти трудности, вводится понятие множества всех элементов, или полного множества, или \defemph{юнивёрсума}, которое обычно обозночают $ U $.

\begin{definition}
    \defemph{Юнивёрсум}~--- множество всех элементов, на которых происходит проверка условия при соответствующем задании некоторого множества.
\end{definition}

Очевидно, что изначально нет никаких ограничений в выборе юнивёрсума; этот выбор обусловлен лишь постановкой задачи.
Поэтому требуется оговаривать заранее, что такое множество $ U $, или хотя бы упоминать его при каждом задании множества условием:
\[
    \{ x \in U \mid f(x) \} \qquad (\text{например,} \; [0; 1) = \{ x \in \R \mid (x \geqslant 0) \wedge (x < 1) \} )
\]

Множество~--- настолько фундаментальный объект, что с его помощью даже пытались описать всю математику.
\begin{example}
    С помощью множеств можно определить, что такое упорядоченная пара:
    \[
        (x, y) = \left\{ x, \{y\} \right\}
    \]
    Или что такое натуральные числа:
    \[
        \varnothing, \; \{ \varnothing \}, \; \left\{ \varnothing, \{ \varnothing \} \right\}, \; \left\{ \varnothing, \left\{ \varnothing, \{ \varnothing \} \right\} \right\}, \ldots
    \]
\end{example}

Однако камнем, на который нашла коса теории множеств, оказался, по сути, всё тот же юнивёрсум.

\begin{example}[Парадокс Рассела]
    Назовём некоторое множество <<обычным>>, если оно не содержит себя в качестве элемента.
    Пусть $ U_0 $~--- множество всех <<обычных>> множеств.
    Можно проверить, что не выполнено как $ U_0 \in U_0 $, так и $ U_0 \notin U_0 $, что является парадоксом.

    Заметим, что запрет на существование <<необычных>> множеств сильно обедняет теорию, ведь тогда не будет сущестовать юнивёрсум всех множеств.
\end{example}

Данный парадокс можно решить, лишь отказавшись от теории множеств в пользу более фундаментальной теории~--- теории типов, но это уже совсем другая история.



\subsection{Множества и логика}
\label{subsec:sets:logic}

Исследуя базовые понятия теории множеств, мы совсем не рассматривали операции, которые можно над множествами совершать.
В этом разделе мы убьём сразу двух зайцев: исследуем связь алгебры логики и теории множеств, а также, наконец, определим упомянутые операции.

Вернёмся к определению множества посредством задания условия.
Понятно, что если юнивёрсум~--- $ \{ 0, 1 \} $, то $ f(x) $ в пункте \ref{itm:sets:condition}~--- булева функция.
В более общем случае говорят, что $ f(x) $~--- \defemph{унарный предикат}.

\begin{definition}
    \defemph{Предикатом} арности $ n $ называют булевозначную функцию от $ n $ аргументов из $ U $. %из $ U^n $ в $ \{ 0, 1 \} $, где
    %\[
    %    U^n = \underbrace{U \times U \times \ldots \times U}_{n \; \text{раз}} = \text{<<множество всех кортежей длины $ n $ из элементов $ U $>>}
    %\]
\end{definition}

Аналогично булевым функциям, сложные предикаты можно конструировать из простых при помощи формул и логических связок (см. \eqref{eq:sets:example_formula}).
Однако наличие юнивёрсума позволяет использовать при построении предикатов ещё более мощный инструмент~--- \defemph{кванторы}.
Ограничимся лишь неформальным их определением:
\begin{align*}
    \forall \; (\text{всеобщности}): &\quad \forall x \, P(x) = \text{<<для любого $ x \in U $ истеннен предикат $ P(x) $>>} \\
    \exists \; (\text{существования}): &\quad \exists x \, P(x) = \text{<<существует $ x \in U $ такой, что истеннен предикат $ P(x) $>>}
\end{align*}

\begin{remark}
    В случае, когда юнивёрсум конечный, любую формулу с кванторами можно заменить эквивалентной формулой без них, используя связки $ \wedge $, $ \vee $ и $ \neg $.
    В общем случае это не так.
\end{remark}

%Без доказательства также приведём следующее замечание:
\begin{remark}[Обобщённый закон де Моргана]
    Справедливо равенство
    \[
        \forall x \, P(x) = \neg \left( \exists x \, \neg P(x) \right)
    \]
\end{remark}

Наконец, можно установить соответствие между любым множеством и некоторым предикатом:
\begin{equation}
    \label{eq:sets:indicator}
    A = \{ x \mid f(x) \} \qquad \Longleftrightarrow \qquad f(x) = \text{<<$ x \in A $>>}
\end{equation}

Предикат, соответствующий согласно равенству \eqref{eq:sets:indicator} некоторому множеству $ A $, будем называть \defemph{индикаторной функцийей} $ A $ и обозначать $ \I_A(x) $.
Используя индикаторные функции и алгебру логики,
можно определить все привычные и непривычные вам теоретико-множественные операции (таблица \ref{tab:sets:operations}) и отношения (таблица \ref{tab:sets:relations}).
Заметим, что пустое множество является подмножеством любого множества!

\begin{table}[ht!]
    \center
    \begin{tabular}{|c|c|c|c|}
        \hline
        Название & Обозначение & Описание & Формула \\
        \hline
        \hline
        Пересечение & $ A \cap B $      & \makecell{Элементы, входящие \\ как в $ A $, так и в $ B $}      & $ \I_{A \cap B}(x) = \I_A(x) \wedge \I_B(x) $ \\
        Объединение & $ A \cup B $      & \makecell{Элементы, входящие \\ в $ A $ или $ B $}               & $ \I_{A \cup B}(x) = \I_A(x) \vee \I_B(x) $ \\
        Разность    & $ A \setminus B $ & \makecell{Элементы, входящие \\ в $ A $, но не в $ B $}          & $ \I_{A \setminus B}(x) = \I_A(x) \wedge \neg \I_B(x) $ \\
        \makecell{Симметрическая \\ разность} & $ A \symdiff B $ & \makecell{Элементы, входящие \\ либо в $ A $, либо в $ B $} & $ \I_{A \symdiff B}(x) = \I_A(x) \oplus \I_B(x) $ \\
        Дополнение  & $ A^c, \; U \setminus A $ & \makecell{Элементы юнивёрсума, \\ не входящие в $ A $}    & $ \I_{A^c}(x) = \neg \I_A(x) $ \\
        \hline
    \end{tabular}
    \caption{теоретико-множественные операции}
    \label{tab:sets:operations}
\end{table}

\FloatBarrier

\begin{table}[ht!]
    \center
    \begin{tabular}{|c|c|c|c|}
        \hline
        Название & Обозначение & Описание & Формула; $ \forall x \ldots $ \\
        \hline
        \hline
        Равенство                              & $ A = B $                              & \makecell{Все элементы $ A $ являются \\ элементами $ B $, и наоборот} & $ \I_A(x) \leftrightarrow \I_B(x) $ \\
        Подмножество                           & $ A \subseteq B $                      & \makecell{Все элементы $ A $ являются \\ элементами $ B $}             & $ \I_A(x) \rightarrow \I_B(x) $ \\
        \makecell{Строгое \\ подмножество}     & $ A \subset B $, $ A \varsubsetneq B $ & \makecell{Все элементы $ A $ являются \\ элементами $ B $, но $ A \neq B $} & \makecell{$ (\I_{A}(x) \rightarrow \I_B(x)) \wedge $ \\ $ \wedge \neg (\I_B(x) \rightarrow \I_A(x)) $} \\
        \hline
    \end{tabular}
    \caption{теоретико-множественные отношения}
    \label{tab:sets:relations}
\end{table}

\FloatBarrier

Множества, отношения и операции с ними часто бывает удобно схематично изображать в виде диаграмм Эйлера-Венна.
В общем случае это набор геометрических фигур, пересечения, вложения и прочие отношения между которыми обозначают те же отношения между соответствующими исходными множествами.

\begin{figure}[ht!]
    \center
    \def\firstcircle{(0,0) circle (30pt)}
    \def\secondcircle{(0:40pt) circle (30pt)}

    \colorlet{circle edge}{blue!50}
    \colorlet{circle area}{blue!20}

    \tikzset{filled/.style={fill=circle area, draw=circle edge, thick},
        outline/.style={draw=circle edge, thick}}

    \setlength{\parskip}{5mm}
    % Set A and B
    \begin{tikzpicture}
        \begin{scope}
            \clip \firstcircle;
            \fill[filled] \secondcircle;
        \end{scope}
        \draw[outline] \firstcircle node {$A$};
        \draw[outline] \secondcircle node {$B$};
        \node[anchor=south] at (current bounding box.north) {$A \cap B$};
    \end{tikzpicture}%
    %
    \hspace{2\baselineskip}%
    %
    % Set A or B
    \begin{tikzpicture}
        \draw[filled] \firstcircle node {$A$}
                      \secondcircle node {$B$};
        \node[anchor=south] at (current bounding box.north) {$A \cup B$};
    \end{tikzpicture}

    %\hspace{2\baselineskip}%
    %
    % Set A but not B
    \begin{tikzpicture}
        \begin{scope}
            \clip \firstcircle;
            \draw[filled, even odd rule] \firstcircle node {$A$}
                                         \secondcircle;
        \end{scope}
        \draw[outline] \firstcircle
                       \secondcircle node {$B$};
        \node[anchor=south] at (current bounding box.north) {$A \setminus B$};
    \end{tikzpicture}%
    %
    \hspace{2\baselineskip}%
    %
    %Set A or B but not (A and B) also known a A xor B
    \begin{tikzpicture}
        \draw[filled, even odd rule] \firstcircle node {$A$}
                                     \secondcircle node{$B$};
        \node[anchor=south] at (current bounding box.north) {$A \symdiff B$};
    \end{tikzpicture}

    \caption{диаграммы Эйлера-Венна для двух пересекающихся множеств и результатов базовых операций с ними}
\end{figure}

\begin{figure}[ht!]
    \center
    \begin{tikzpicture}
        \tikzset{venn circle/.style={draw,circle,minimum width=120pt,fill=#1,opacity=0.4,line width=1pt}}

        \node[venn circle = red] (A) at (0,0) {};% {$A$};
        \node[venn circle = blue] (B) at (60:60pt) {};% {$B$};
        \node[venn circle = green] (C) at (0:60pt) {};% {$C$};

        \node[text opacity=0.4] at (barycentric cs:A=7,B=-1,C=-1 ) {$A$};
        \node[text opacity=0.4] at (barycentric cs:A=-1,B=7,C=-1 ) {$B$};
        \node[text opacity=0.4] at (barycentric cs:A=-1,B=-1,C=7 ) {$C$};

        \node at (barycentric cs:A=1,B=1,C=-2/3 ) {$A \cap B$};
        \node at (barycentric cs:A=1,B=-2/3,C=1 ) {$A \cap C$};
        \node at (barycentric cs:A=-2/3,B=1,C=1 ) {$B \cap C$};
        \node at (barycentric cs:A=2,B=1,C=2 ){$A \cap B \cap C$};
    \end{tikzpicture}
    \caption{диаграмма Эйлера-Венна для трёх множеств}
\end{figure}

\FloatBarrier

\begin{remark}
    Интересно, что невозможно нарисовать диаграмму Эйлера-Венна, состоящую из окружностей (пусть и произвольного размера), для всех вариантов отношений между четырьмя и более множествами.
\end{remark}

Наконец, введём последнее обозначение, необходимое в этом разделе.
\begin{definition}
    \defemph{Множеством всех подмножеств} (или \defemph{булеаном}) некоторого множества $ A $ называется множество $ \mathcal{P}(A) = 2^A = \{ x \mid x \subseteq A \} $.
\end{definition}


\begin{Exercise}[counter=SecExercise]
    \noindent
    Задайте формально следующие множества:
    \begin{enumerate}[label=\arabic*)]
        \item Множество простых чисел.
        \item Множество всех отрезков на числовой прямой.
        \item Множество всех действительных корней всевозможных нетривиальных квадратных многочленов с целыми коэффициентами.
              Что изменится, если убрать условие на коэффициенты?
    \end{enumerate}
\end{Exercise}

\begin{Answer}
    \noindent
    \begin{enumerate}[label=\arabic*)]
    \item $ \mathbb{P} = \left \{ x \in \N \mid \neg \left( \exists a \exists b \, (x = a \cdot b) \wedge (a \neq 1) \wedge (a \neq x) \right) \wedge (x \neq 1) \right \} $.\\
        По умолчанию считаем, что $ a $ и $ b $ принадлежат тому же юнивёрсуму, что и $ x $.
    \item $ S = \left\{ x \in 2^{\R} \mid \exists a \exists b \, (a \in \R) \wedge (b \in \R) \wedge \left( \forall c \, (a \leqslant c \leqslant b) \leftrightarrow (c \in x) \right) \right\} $
    \item $ R = \left\{ x \in \R \mid \exists a \exists b \exists c \, (a \in \Z) \wedge (b \in \Z) \wedge (c \in \Z) \wedge (a^2 + b^2 + c^2 \neq 0) \wedge (a x^2 + b x + c = 0) \right\} $.\\
          Если коэффициенты сделать произвольными, то, очевидно, получится $ \R $, так как любое число $ a $ является корнем $ x - a = 0 $.
    \end{enumerate}
\end{Answer}


\begin{Exercise}[counter=SecExercise]
    \noindent
    Нарисуйте диаграмму Эйлера-Венна для трёх множеств из предыдущей задачи.
\end{Exercise}

\begin{Answer}
    \noindent
    Очевидно, $ S $ никак не пересекается с $ \mathbb{P} $ и $ R $ в силу разной природы объектов.
    Также понятно, что для любого простого числа $ p $ можно построить многочлен $ x - p $ с целыми коэффициентами, корнем которого $ p $ и будет являться;
    также, например, $ \sqrt{2} \in R $.
    Итого, $ \mathbb{P} \varsubsetneq R $.

    \begin{tikzpicture}
        \tikzset{venn circle/.style={draw,circle,minimum width=80pt,fill=#1,opacity=0.4,line width=1pt}}

        \node[venn circle = red, minimum width=40pt] (A) at (60:15pt) {};% {$A$};
        \node[venn circle = blue] (B) at (0:100pt) {};% {$B$};
        \node[venn circle = green] (C) at (0,0) {};% {$C$};

        \node at (barycentric cs:A=1,B=0,C=0 ) {$\mathbb{P}$};
        \node at (barycentric cs:A=0,B=1,C=0 ) {$S$};
        \node at (barycentric cs:A=-3,B=0,C=5 ) {$R$};
    \end{tikzpicture}
\end{Answer}


\begin{Exercise}[counter=SecExercise]
    \noindent
    Верно ли, что если $ B \subseteq A $, то $ A \setminus B = A \symdiff B $?
\end{Exercise}

\begin{Answer}
    \noindent
    Если $ B \subseteq A $, то
    \[
        \I_{A \symdiff B}(x) = \underbrace{(\I_A(x) \vee \I_B(x))}_{\I_A(x), \; \text{т.к.} \; \forall x \, \I_B(x) \rightarrow \I_A(x)} \wedge
        \underbrace{\neg (\I_A(x) \wedge \I_B(x))}_{\neg \I_B(x), \; \text{т.к.} \; \forall x \, \I_B(x) \rightarrow \I_A(x)} =
        \I_A(x) \wedge \neg \I_B(x) = \I_{A \setminus B}(x)
    \]
    Аналогичное доказательство в теоретико-множественных обозначениях:
    \[
        A \symdiff B = \underbrace{(A \cup B)}_{A} \setminus \underbrace{(A \cap B)}_{B} = A \setminus B
    \]
    Наконец, проведём также лобовое доказательство через проверку тавтологичности следующей формулы:
    \[
        (b \rightarrow a) \rightarrow \left( (a \wedge \neg b) \leftrightarrow (a \oplus b) \right)
    \]
    Здесь введены обозначения $ a = \I_A(x) $, $ b = \I_B(x) $.
    Используя закон де Моргана, а также тождества $ x \to y = \neg x \vee y $ и $ x \oplus y = \neg (x \leftrightarrow y) $, перепишем формулу в виде
    \[
        (b \rightarrow a) \rightarrow \left( \neg (a \rightarrow b) \leftrightarrow \neg (a \leftrightarrow b) \right)
    \]
    Операция <<$ \leftrightarrow $>> позволяет убрать отрицания с обоих аргументов:
    \[
        (b \rightarrow a) \rightarrow \left( (a \rightarrow b) \leftrightarrow (a \leftrightarrow b) \right)
    \]
    Воспользуенмся дистрибутивностью импликации относительно эквивалентности (см. конец решения задачи \ref{ex:boolean:disjunction_distributivity_over_equivalence}):
    \[
        \left[ (b \rightarrow a) \rightarrow (a \rightarrow b) \right] \leftrightarrow \left[ (b \rightarrow a) \rightarrow (a \leftrightarrow b) \right]
    \]
    Перепишем эквивалентность в последней скобке через конъюнкцию импликаций
    и воспользуемся дистрибутивностью импликации относительно конъюнкции:
    \[
        \left[ (b \rightarrow a) \rightarrow (a \rightarrow b) \right] \leftrightarrow \left[ (b \rightarrow a) \rightarrow \left( (a \rightarrow b) \wedge (b \rightarrow a) \right) \right]
    \]
    \[
        \left[ (b \rightarrow a) \rightarrow (a \rightarrow b) \right] \leftrightarrow
        \left[ \left( (b \rightarrow a) \rightarrow (a \rightarrow b) \right) \wedge \left( (b \rightarrow a) \rightarrow (b \rightarrow a) \right) \right]
    \]
    Второй операнд конъюнкции тавтологичен, и формула, наконец, упрощается:
    \[
        \left[ (b \rightarrow a) \rightarrow (a \rightarrow b) \right] \leftrightarrow
        \left[ (b \rightarrow a) \rightarrow (a \rightarrow b) \right]
    \]
    Здесь тавтологичность уже очевидна в силу равенства формул для левой и правой частей.
\end{Answer}

Третий способ доказательства в предыдущей задаче получился самым громоздким, но самым формальным,
а потому подвластным компьютеру (см. автоматические доказательства теорем).
Более того, в нём еще раз была продемонстрирована красивое свойство логики:
возможность формально описывать саму себя на своём языке.

\begin{Exercise}[counter=SecExercise]
    \noindent
    Используя только операции $ \symdiff $ и $ \cap $, выразите $ A \cup B \cup C $
\end{Exercise}

\begin{Answer}
    \noindent
    Докажем, что
    \[
        A \cup B \cup C = U \symdiff A \symdiff B \symdiff C \symdiff A \cap B \symdiff A \cap C \symdiff B \cap C \symdiff A \cap B \cap C
    \]
    Действительно, если это переписать в терминах индикаторных функций, то получим
    \[
        %1 \oplus \I_A(x) \oplus \I_B(x) \oplus \I_C(x) \oplus \I_A(x) \wedge \I_B(x) \oplus \I_A(x) \wedge \I_C(x) \oplus \I_B(x) \wedge \I_C(x) \oplus \I_A(x) \wedge \I_B(x) \wedge \I_C(x)
        1 \oplus a \oplus b \oplus c \oplus a \wedge b \oplus a \wedge c \oplus b \wedge c \oplus a \wedge b \wedge c
    \]
    где $ a = \I_A(x) $, $ b = \I_B(x) $, $ c = \I(x) $.

    Видно, что если $ a = b = c = 0 $, то выражение ложно.
    Также заметим, что если $ a = 1 $, а $ b = c = 0 $, то выражение истинно.
    Далее можно заметить, что до тех пор, пока есть хотя бы одна истинная переменная, формула не меняет своего значения при изменении любой другой переменной:
    свои значения всегда будет менять чётное число слагаемых.
    Но тогда из всего вышесказанного следует, что формула задаёт функцию $ a \vee b \vee c $.
    Возвращаясь к множествам, получаем, что исходное тождество доказано.

    Кратко упомянем и иное доказательство: достаточно проверить равенство только в случаях $ (a,b,c) = (0,0,0) $, $ (1,0,0) $, $ (1,1,0) $ и $ (1,1,1) $, а потом воспользоваться симметричностью формулы.
\end{Answer}


\iffalse
\subsection{Мощность множества}
\label{subsec:sets:cardinality}

Помимо состава множеств и взаимоотношений между ними нас часто будет интересовать то, насколько некоторое множество <<велико>>.
Легко определить <<размер>> множества в случае, когда оно конечно: это просто число элементов.
Но что делать, если множество содержит бесконечно много элементов?
Хочется сказать, что если два множества бесконечны, то они <<равновелики>>.
Однако это противоречит интуитивным представлениям о том, что, например, $ 2^A $ содержит элементов больше, чем $ A $.

Оказывается, эти интуитивные представления можно формализовать, если по-другому взглянуть на размер конечных множеств.
Если множества $ A $ и $ B $ конечны, то можно сказать, что они равновелики, если в них одинаковое число элементов.
По сути, это эквивалентно тому, что можно задать взаимнооднозначное правило соответствия между каждым элементом $ A $ и $ B $.
%достаточно пронумеровать элементы любым способом, и в соответствие друг другу ставить элементы с одинаковым номером;
%обратное следствие тоже очевидно: если мы пронумеруем элементы одного множества, то, благодаря правилу, будут пронумерованы элементы и другого множества,
%причём не будет ни повторений, ни пропущенных элементов.

\begin{statement}
    Если $ A $ и $ B $~--- конечные множества, то они содержат одинаковое число элементов тогда и только тогда,
    когда существует взаимнооднозначное соответствие (\defemph{биекция}) между элементами множеств.
\end{statement}

Формальное доказательство утверждения становится очевидным, если любым способом пронумеровать элементы множеств.

Данное утверждение позволяет по-иному формально определить размер, или \defemph{мощность} множества, и обобщить это определение на все множества вообще.
\begin{definition}
    Множества $ A $ и $ B $ называются \defemph{равномощными} в том и только том случае,
    если существует взаимнооднозначное соответствие между элементами множеств.
\end{definition}
Стоит обратить внимание, что формальное определение взаимнооднозначного соответствия будет дано гораздо позже.
Также понятно, что это соответствие не обязано быть единственным.

\begin{definition}
    Множество $ A $ называется \defemph{счётным} $ \defarr $ $ A $ равномощно $ \N_0 $.
\end{definition}

\begin{example}
    \begin{enumerate}
        \item[]
        \item
            Множества $ \{1, 2\} $ и $ \{a, x\} $ равномощны, причём можно построить два взаимнооднознычных соответствия между ними:
            \[
                1 \sim a, \; 2 \sim x \qquad \text{или} \qquad 1 \sim x, \; 2 \sim a
            \]
        \item
            Множества $ \N_0 $ и $ E = \{ x \in \N_0 \mid \exists k \, (x = 2k) \} $ равномощны, соответствие задаётся, например, правилом $ E \ni x = 2 \cdot k $, где $ k $~--- любой элемент $ \N_0 $.
        \item
            Множества $ \Q $ и $ \N_0 $ равномощны.
            Идея доказательства: $ \Q $ можно задать бесконечной таблицей, номер строки и столбца в которой~--- числитель и знаменатель.
            А все ячейки таблицы можно пронумеровать, идя <<змейкой>> (при этом сократимые дроби не нумеруются).
    \end{enumerate}
\end{example}

Может создасться впечатление, что все бесконечные множества счётны.
Однако это неверно.

\begin{theorem}[Кантора]
    Для любого $ A $ множества $ A $ и $ 2^A $ неравномощны.
\end{theorem}

\begin{statement}
    Множество $ \R $ несчётно.
\end{statement}

Доказательство данного утверждения обычно приводят в курсе математического анализа.

\begin{Exercise}[counter=SecExercise]
    \noindent
    Счётно ли множество всех корректных программ, написанных на языке C++?
\end{Exercise}

\begin{Answer}
    \noindent
    Да, оно счётно.
    Для доказательства этого заметим, что можно построить следующую таблицу:
    номер строки равен длине программы в символах, а номер столбца~--- лексикографическому порядковому номеру программы среди всех программ заданной длины.
    Обходя таблицу <<змейкой>>, получаем взаимнооднозначную нумерацию всех программ.
\end{Answer}
\fi
           % Теория множеств.
\section{Математические определения, утверждения и доказательства}
\label{sec:formal_systems}

В предыдущих разделах мы ввели внушительный математический аппарат, более-менее строго описывающий базовые математические объекты и связи между ними.
Настало время применить этот аппарат и для формального описания более сложных объектов.

\begin{definition}
    \label{definition:formal_systems:definition}
    \defemph{Определением} называется некоторый унарный предикат;
    он является индикаторной функцией множества объектов, удовлетворяющих определению.
\end{definition}

\begin{definition}
    \defemph{Математическим утверждением} называется формула, не имеющая свободных переменных (параметров), а потому либо истинная, либо ложная.
    Иногда утвеждением также называется предикат арности, большей нуля; в таком случае подразумевается, что перед ним стоят кванторы всеобщности по всем свободным переменным.
\end{definition}

\begin{definition}
    \defemph{Теоремой}, \defemph{леммой}, \defemph{предложением} или \defemph{утверждением} называется истинное математическое утверждение.
    Выбор конкретного названия обусловлен лишь ролью утверждения в математическом тексте.
\end{definition}

\begin{definition}
    \defemph{Критерием} называется истинное математическое утверждение вида
    \[
        \forall x \, \left( A(x) \leftrightarrow B(x) \right)
    \]
\end{definition}

Определим также несколько вспомогательных терминов, возникающих при рассмотрении математических утверждений определённого вида.
\begin{definition}
    Пусть имеется математическое утверждение вида
    \[
        \forall x \, \left( A(x) \rightarrow B(x) \right)
    \]
    Тогда говорится, что условие $ B(x) $ \defemph{необходимо} для выполнения $ A(x) $, а условие $ A(x) $ \defemph{достаточно} для выполнения $ B(x) $.
    Или, по-другому, условие $ A(x) $ более \defemph{сильное}, чем $ B(x) $, а $ B(x) $~--- более \defemph{слабое}, чем $ A(x) $.

    Если также имеются математические утверждения вида
    \[
        \forall x \, \left( A(x) \rightarrow C(x) \right), \qquad
        \forall x \, \left( B(x) \rightarrow C(x) \right),
    \]
    то второе из них считается более \defemph{сильным}, так как в нём $ C(x) $ следует из более слабого математического утверждения $ B(x) $.
\end{definition}



\subsection{Доказательства}
\label{subsec:formal_systems:proofs}

Наверняка вы заметили, как просто было доказать тавтологичность формул в алгебре логики: если совсем никак не получается это сделать преобразованием формул,
верный ответ всегда даст таблица истинности.
Проблемы начинаются при первой же попытке перейти к чему-то более сложному.
Реальные теоремы простым перебором всех аргументов предикатов не докажешь, ведь юнивёрсум может быть бесконечным.
Остаются только преобразования формул.

Во многом, формальная логика~--- это наука о переписывании формул.
В основу формальной системы ложатся некоторые аксиомы, тавтологичность которых постулируется, и некоторые правила вывода.
%позволяющие упрощать запись некоторого утверждения до тех пор, пока не будет показана его тавтологичность.
%Также наоборот, правила вывода позволяют получать из некоторого утверждения другие, истинные в предположении верности первоначального.
%Если в результате вывода получается противоречие, исходное утверждение ложно.
Основным правилом является \defemph{modus ponens}:
\[
    \typerule{A \rightarrow B, \quad A}{B}
\]
Данная запись, по существу, означает, что если истинно как утверждение $ A $, так и $ A \rightarrow B $, то истинно и $ B $.
На его основе можно сконструировать и более сложные правила, некоторые из которых даже имеют своё название.
\begin{example}
    \label{example:formal_systems:rules}
    \begin{enumerate}
        \item[]
        \item Несколько безымянных правил вывода:
           \[
                \typerule{A \vee B, \quad \neg A}{B} \qquad
                \typerule{A_1, \quad \ldots, \quad A_{n-1}, \quad \neg \left( \bigwedge_{k=1}^n A_k  \right)}{\neg A_n} \qquad
                \typerule{A \wedge B}{A} \qquad
                \typerule{B}{A \rightarrow B}
            \]
        \item Правило контрапозиции:
            \[
                \typerule{A \rightarrow B}{\neg B \rightarrow \neg A}
            \]
            (то есть, $ (A \rightarrow B) \leftrightarrow (\neg B \rightarrow \neg A) $~--- тавтология, как уже было показано в разделе про алгебру логики).
        \item <<От противного>>:
            \[
                \typerule{A \rightarrow B, \quad \neg B}{\neg A} \qquad
            \]
    \end{enumerate}
\end{example}

\begin{remark}
    В общем случае запись $ \; \displaystyle \typerule{A_1, \quad \ldots, \quad A_n}{B} \; $
    эквивалентна тавтологичности формулы $ \left( \bigwedge_{k=1}^n A \right) \rightarrow B $.
\end{remark}

Доказательство теоремы сводится либо к упрощению при помощи правил вывода входной формулы до уровня, когда тавтологичность проверяется легко,
либо к выводу из исходного утверждения и аксиом некоторого противоречия, которое будет говорить о том, что исходное утверждение ложно.
В нашем курсе, конечно, не придётся опускаться до таких формальностей, чтобы даже самые простые утверждения доказывать исключительно итеративным явным применением правил вывода.

Отдельно отметим правило вывода, не сводящееся к modus ponens: \defemph{математическую индукцию}.
Формальная запись:
\[
    \typerule{A(0), \quad \forall n \, \left( A(n) \rightarrow A(n+1) \right)}{\forall n \, A(n)}
\]
Данное правило либо постулируется, либо выводится из некоторых других аксиом.

\begin{Exercise}[counter=SecExercise]
    \noindent
    Получите формально правило
    \[
        \typerule{A \oplus B, \quad B}{\neg A}
    \]
    из modus ponens, правил в примере \ref{example:formal_systems:rules}, правил преобразования к связкам $ \neg $ и $ \rightarrow $, а также тавтологичности $ A \rightarrow (B \rightarrow A) $.
\end{Exercise}

\begin{Answer}
    \noindent
    Перейдём к отрицаниям и импликациям:
    \begin{multline*}
        A \oplus B = \neg(A \leftrightarrow B) = \neg \left[ (A \rightarrow B) \wedge (B \rightarrow A) \right] = \\
        =\neg (A \rightarrow B) \vee \neg (B \rightarrow A) = (A \rightarrow B) \rightarrow \neg (B \rightarrow A)
    \end{multline*}
    Из правила в примере имеем
    \[
        \typerule{B}{A \rightarrow B}
    \]
    Применяя modus ponens к исходному утверждению и полученной формуле, получаем
    \[
        \typerule{A \rightarrow B, \quad (A \rightarrow B) \rightarrow \neg (B \rightarrow A)}{\neg (B \rightarrow A)}
    \]
    % Применим правило контрапозиции:
    % \[
    %     \typerule{A \rightarrow (B \rightarrow A)}{\neg (B \rightarrow A) \rightarrow \neg A}
    % \]
    % Наконец,
    % \[
    %     \typerule{\neg (B \rightarrow A) \rightarrow \neg A, \quad \neg (B \rightarrow A)}{\neg A}
    % \]
    Правило <<от противного>>:
    \[
        \typerule{A \rightarrow (B \rightarrow A), \quad \neg (B \rightarrow A)}{\neg A}
    \]
\end{Answer}

Еще раз акцентируем внимание на том, что \textbf{не требуется излишне формализовывать процесс доказательства, если в этом нет необходимости!}
Изложенная выше теория должна помочь вам понять общую структуру процесса доказательства, увидеть некоторую его модульность,
переиспользование каких-то распространённых схем вывода как самостоятельных правил.

%Осталось упомянуть ещё один важный приём~--- \defemph{доказательстве от противного}.
%Он основывается на том факте, что если некоторое математическое утверждение $ A $ истинно,
%%(пускай мы это даже не знаем заранее, и тем более пускай мы не знаем доказательства истинности $ A $),
%то при предположении истинности $ \neg A $ истинно также $ A \wedge \neg A $.
%Это уже является противоречием, но при реальном доказательстве, конечно, именно это противоречие можно получить,
%только если доказать $ A $, что, казалось бы, делает этот способ бессмысленным.

%Однако на помощь приходит тот факт, что если доказуемо одно противоречие, то доказуемо всё, что угодно, в том числе и какое-то другое противоречие.
%Значит, если в процессе доказательства будет найдено любое другое противоречие, исходное предположение об истинности $ \neg A $ ложно.



\subsection{Примеры}
\label{subsec:formal_systems:examples}

Рассмотрим несколько примеров использования изученных в предыдущем разделе методов доказательства.

\begin{Exercise}[counter=SecExercise]
    \noindent
    Пусть $ A $, $ B $, $ C $~--- множества.
    Верно ли, что если $ A \cap B $ не является подмножеством $ C $, то или $ A \not\subseteq C $, или $ B \not\subseteq C $?
\end{Exercise}

\begin{Answer}
    \noindent
    Требуется проверить тавтологичность
    \[
        \neg (A \cap B \subseteq C) \rightarrow \left[ \neg (A \subseteq C) \wedge \neg (B \subseteq C) \right]
    \]
    Сделаем это, применив правило контрапозиции.
    \[
        \left[ (A \subseteq C) \vee (B \subseteq C) \right] \rightarrow (A \cap B \subseteq C)
    \]
    Можно <<добить>> это выражение формальными преобразованиями, но, в принципе, уже понятно, что если хотя бы одно множество полностью лежит в $ C $, то и пересечение тоже.

    Для тренировки упростим формулу до конца, введя стандартные обозначения для индикаторных функций и перейдя к алгебре логики:
    \[
        \left[ (a \rightarrow c) \vee (b \rightarrow c) \right] \rightarrow (a \wedge b \rightarrow c)
    \]
    Раскроем все импликации:
    \[
        \neg \left[ (\neg a \vee c) \vee (\neg b \vee c) \right] \vee (\neg(a \wedge b) \vee c)
    \]
    \[
        \neg [\neg a \vee \neg b \vee c] \vee (\neg a \vee \neg b \vee c)
    \]
    Тавтологичность доказана.
\end{Answer}


\begin{Exercise}[counter=SecExercise]
    \noindent
    Докажите, что если у числовой последовательности есть предел, то он единственнен.
\end{Exercise}

\begin{Answer}
    \noindent
    Характерный пример доказательство от противного.
    Предположим, что $ a \neq b $~--- два предела $ \{ x_n \}_{n=1}^{\infty} $.
    Определение предела:
    \[
        \forall \varepsilon > 0 \;\; \exists N: \; \forall n > N \;\; |x_n - a\,(b)| < \varepsilon
    \]
    Возьмём $ \varepsilon_0 = |b - a| / 3 > 0 $.
    Тогда
    \[
        \exists N_a: \; \forall n > N_a \;\; |x_n - a| < \varepsilon_0 \qquad
        \exists N_b: \; \forall n > N_b \;\; |x_n - b| < \varepsilon_0
    \]
    Обозначим $ x' = x_{\max \{N_a, N_b\} + 1} $.
    Тогда $ 3\varepsilon_0 = |a - b| = |a - x' + x' - b| \leqslant |a - x'| + |x' - b| < \varepsilon_0 + \varepsilon_0 <  3 \varepsilon_0 $.
    Противоречие
    Значит, либо $ a = b $, либо $ \{x_n\}_{n=1}^{\infty} $ не имеет предела.
\end{Answer}


\begin{Exercise}[counter=SecExercise]
    \noindent
    Пусть последовательность $ \{ x_n \}_{n=0}^{\infty} $ задана рекуррентным соотношением $ x_{n+1} = a x_n + b $, $ a \neq 1 $.
    Докажите, что
    \begin{equation}
        \label{eq:formal_systems:recursive}
        x_n = x_0 \cdot a^n + b \cdot \frac{a^n - 1}{a - 1}
    \end{equation}
\end{Exercise}

\begin{Answer}
    \noindent
    Докажем методом математической индукции.\\
    \textbf{База:} $ x_0 = x_0 \cdot 1 + b \cdot 0 = x_0 $.\\
    \textbf{Шаг:} пусть для некоторого $ n $ верно \eqref{eq:formal_systems:recursive}.
    Тогда
    \[
        x_{n+1} = a x_n + b = x_0 \cdot a \cdot a^n + b \cdot \left[ 1 + a \cdot \frac{a^n - 1}{a - 1} \right] = x_0 \cdot a^{n+1} + b \cdot \frac{a^{n+1} - 1}{a - 1}
    \]
    Итого, по индукции доказано.
\end{Answer}


Заметим, что с математическими утверждениями следует работать аккуратно;
часто встречаются ошибки, связанные с опусканием или неправильной перестановкой кванторов.

\begin{Exercise}[counter=SecExercise]
    \noindent
    Убедитесь в истинности утверждения (при произвольных $ A $ и $ B $):
    \begin{equation}
        \label{eq:formal_systems:tautological_example}
        (A \rightarrow B) \vee (B \rightarrow A)
    \end{equation}
    Зафиксируем произвольную параболу.
    Пусть A~--- утверждение <<ветви параболы направлены вверх>>,
    а $ B $~--- <<парабола пересекает $ 0 $ (прямую $ y = 0 $)>>.
    Проследите за следующими рассуждениями.
    Утверждение <<если ветви параболы направлены вверх, то парабола пересекает 0>>, очевидно, ложно;
    тогда истинным должно быть утверждение <<если парабола пересекает $ 0 $, то ветви параболы направлены вверх>>, но оно также ложно.
    То есть оба утверждения в дизъюнкции \eqref{eq:formal_systems:tautological_example} ложны
    (при некотором выборе утверждений A и B), но сама дизъюнкция истинна!
    Найдите ошибку в рассуждениях.
\end{Exercise}

\begin{Answer}
    \noindent
    Действительно, \eqref{eq:formal_systems:tautological_example} тавтологична независимо от выбора $ A $ и $ B $:
    \[
        (A \rightarrow B) \vee (B \rightarrow A) = \neg A \vee B \vee \neg B \vee A = \neg A \vee A \vee \neg B \vee B = 1 \vee 1 = 1
    \]
    Ошибка в рассуждениях заключается в неправильной работе с кванторами.
    Пусть $ P(x) $~--- <<ветви параболы $ x $ направлены вверх>>,
    а $ Q(x) $~--- <<парабола $ x $ пересекает прямую $ y = 0 $>>.
    Тогда $ A = \forall x \, P(x) $, $ B = \forall x \, Q(x) $.
    Рассуждения всего-навсего доказывают следующее неравенство:
    \[
        \left[ \forall x \, (P(x) \rightarrow Q(x)) \right] \vee \left[ \forall x \, (Q(x) \rightarrow P(x)) \right] \neq
        (\underbrace{\forall x \, P(x)}_{A} \rightarrow \underbrace{\forall x \, Q(x)}_{B}) \vee (\forall x \, Q(x) \rightarrow \forall x \, P(x))
    \]
    Действительно, под формулу \eqref{eq:formal_systems:tautological_example} подходит только правая часть неравенства,
    так как в левой части в импликациях стоят не утверждения (предикаты арности ноль), а унарные предикаты.

    Таким образом, рассуждения в условии лишь иллюстрируют невозможность (в общем случае) <<вноса>> квантора всеобщности внутрь импликации.
\end{Answer}

 % Формальные системы.
\section{Неориентированные графы}
\label{sec:graphs}

\defemph{Графы}~--- это математические объекты, довольно часто встречающиеся в реальных задачах (логистика, интернет, социальные связи),
но при этом достаточно простые, чтобы без труда формально определить их при помощи изученного нами математического аппарата.
Неформально говоря, граф~--- это абстракция, применимая к множеству любой природы, в случае, когда интересны только парные связи между его элементами.

Граф часто представляется в виде изображения следующего формата: точки или кружки (элементы множества, \defemph{вершины}) соединены линиями или стрелками (\defemph{рёбрами}),
изображающими связи между элементами.
Вершины и рёбра могут иметь некоторые \defemph{атрибуты} (числа, строки, любые другие объекты).

\begin{figure}[ht!]
    \center
    \begin{tikzpicture}
        \node[shape=circle,draw=black] (NK)  at (1,0) {НК};
        \node[shape=circle,draw=black] (KPM) at (-1,-1) {КПМ};
        \node[shape=circle,draw=black] (GK)  at (3,2) {ГК};
        \node[shape=circle,draw=black] (DIG) at (2,4) {Цифра};
        \node[shape=circle,draw=black] (ARK) at (0,6) {Арктика};
        \node[shape=circle,draw=black] (LK)  at (5,2) {ЛК};
        \node[shape=circle,draw=black] (AC)  at (7,0) {АК};
        \node[shape=circle,draw=black] (KSP) at (6,-2) {КСП};
        \node[shape=rectangle,draw=black] (ARMY) at (11,0) {Военкомат};

        \path [-] (NK) edge node[sloped, anchor=center, above] {$500$}  (KPM);
        \path [-] (NK) edge node[sloped, anchor=center, above] {$1500$} (GK);
        \path [-] (GK) edge node[sloped, anchor=center, above] {$700$}  (DIG);
        \path [-] (GK) edge node[sloped, anchor=center, above] {$400$}  (LK);
        \path [-] (DIG)edge node[sloped, anchor=center, above] {$300$}  (ARK);
        \path [-, dashed](LK) edge node[sloped, anchor=center, above] {$500$} (AC);
        \path [<->, dashed](AC) edge node[sloped, anchor=center, above] {$450$} (KSP);
        \path [<->, dashed](NK) edge [bend left=-30] node[sloped, anchor=center, above] {$600$} (KSP);
        \path [->, dashed](AC) edge node[sloped, anchor=center, above] {$0.1$} (ARMY);
    \end{tikzpicture}

    \caption{граф ежедневного перемещения людей между некоторыми связанными с Физтехом зданиями.
    Рёбра обозначают тип и направление перехода, а также ежедневный поток студентов.}
    \label{fig:graphs:people_phystech}
\end{figure}


Изучение графов мы начнём с самых простых их разновидностей, постепенно усложняя конструкцию.
Но перед этим потребуется ввести пару вспомогательных определений.

\begin{definition}
    \defemph{Множеством всех подмножеств мощности $ k $} некоторого множества $ A $ будем называть множество
    \[
        \binom{A}{k}
        =
        \{ B \mid (B \subseteq A) \wedge (|B| = k) \}
    \]
\end{definition}

\begin{definition}
    \defemph{Множеством всех неупорядоченных пар} некоторого множества $ A $ будем называть множество $ \displaystyle \begin{pmatrix} A \\ 2 \end{pmatrix} $.
\end{definition}

\begin{remark}
    \label{remark:graphs:subsets_cardinality}
    Если $ |A| = n < +\infty $, и $ 0 \leqslant k \leqslant n $, то
    \[
        \left|
        \binom{A}{k}
        \right|
        =
        \frac{n!}{k! (n-k)!}
        \defeq
        \binom{n}{k}
        \defeq
        C_n^k
    \]
\end{remark}



\subsection{Простые неориентированные графы}
\label{subsec:graphs:simple_graphs}

%Начнём с \defemph{простых неориентированных графов}, то есть, с графов, вершины и рёбра которых не имеют никаких атрибутов (в том числе, направления).
Начнём с самой простой конструкции графа, для построения которой достаточно понятия неупорядоченной пары.
\begin{definition}
    \defemph{(Простой неориентированный) граф}~--- это упорядоченная пара $ (V, E) $ множества \defemph{вершин} $ V $
    и \defemph{рёбер} $ E \subseteq \binom{V}{2} $.
    Введём также следующие обозначения, если $ V $ и $ E $ фиксированы:
    \[
        G = (V, E)\text{~--- граф} \qquad \Longrightarrow \qquad
        G(V, E) \defeq (V, E), \; V(G) \defeq V, \; E(G) \defeq E
    \]
\end{definition}

\defemph{<<Простой>>} означает, что в графе нет \defemph{петель} (рёбер вида $ \{v, \; v\} = \{ v \} $) и \defemph{кратных рёбер} (каждое ребро входит в $ E $ единожды).
\defemph{<<Неориентированный>>} означает, что ребро является неупорядоченной парой.


Для следующих определений и утверждений зафиксируем граф $ G = G(V, E) $.
\begin{definition}
    Вершины $ v $ и $ v' $ называются \defemph{смежными} или \defemph{соседями}, если они образуют ребро: $ \{v, v'\} \in E $.
    Рёбра $ e $ и $ f $ называются \defemph{смежными}, если они имеют общую вершину: $ e \cap f \neq \varnothing $.
    Вершина $ v $ \defemph{инцидентна} ребру $ e $ (или, что то же самое, ребро $ e $ \defemph{инцидентно} вершине $ v $), если $ v \in e $.
    Вершины $ v $ и $ v' $, инцидентные ребру $ e $, называются его \defemph{концами};
    говорят, что $ e $ \defemph{соединяет} $ v $ и $ v' $.
    Рёбра часто записывают сокращённо: $ uv $ вместо $ \{u, v\} $.
\end{definition}

\begin{definition}
    \defemph{Степенью} вершины $ v $ называется число $ d(v) $ инцидентных $ v $ рёбер.
\end{definition}

\begin{Exercise}[counter=SecExercise, label={exercise:graphs:inconsistent_degrees}]
    \noindent
    Докажите, что не существует графа с пятью вершинами,
    степени которых равны $ 4 $, $ 4 $, $ 4 $, $ 4 $, $ 2 $.
\end{Exercise}

\begin{Answer}
    \noindent
    Если $ 4 $ из $ 5 $ вершин имеют степень $ 4 $, то они соединены со всеми вершинами, кроме себя, каждая.
    Но пятая вершина по условию соединена только с двумя другими вершинами.
    Получаем, что графа с указанными свойствами существовать не может.
\end{Answer}

\begin{theorem}[о рукопожатиях]
    \label{theorem:graphs:sum_of_degs}
    $ \displaystyle \sum_{v \in V} d(V) = 2 |E| $
\end{theorem}

\begin{proof}
    $ \displaystyle \sum_{v \in V} d(V) = \sum_{v \in V} \sum_{\substack{e \in E, \\ v \in e}} 1 = \sum_{e \in E} \sum_{\substack{v \in V, \\ v \in e}} 1 = \sum_{e \in E} 2 = 2 |E| $
\end{proof}

\begin{Exercise}[counter=SecExercise, label={exercise:graphs:sum_of_degs_less}]
    \noindent
    В графе $ 100 $ вершин и $ 800 $ рёбер.
    \begin{enumerate}[label=\textbf{\alph*)}]
        \item
            докажите, что в этом графе есть хотя бы одна вершина степени не меньше $ 16 $.
        \item
            может ли так случиться, что все вершины этого графа имеют степень $ 16 $?
    \end{enumerate}
\end{Exercise}

\begin{Answer}
    \noindent
    \begin{enumerate}[label=\textbf{\alph*)}]
        \item
            Предположим, что такой вершины нет;
            тогда $ \forall v \in V \;\, d(v) < 16 $.
            Применяя теорему о рукопожатиях, имеем
            \[
                1600 = 2 |E| = \sum_{v \in V} d(v) < 16 \cdot |V| = 1600
            \]
            Противоречие.
        \item
            Да.
            Мысленно расположим все $ 100 $ вершин по кругу и соединим каждую вершину с $ 8 $-ю ближайшими по часовой стрелке и с $ 8 $-ю ближайшими против.
            Получим, что каждая вершина имеет степень $ 16 $, причём $ \displaystyle |E| = \frac{1}{2} \sum_{v \in V} d(v) = 8 |V| = 800 $, как и требуется.
    \end{enumerate}
\end{Answer}

Определим отдельно несколько частных случаев простого неориентированного графа.
Примеры всех указанных ниже случаев, кроме пустого графа, можно также найти на рис. \ref{fig:graphs:basic_graphs}.
\begin{definition}
    \label{def:graphs:named_graphs}
    \begin{enumerate}[label=\arabic*)]
        \item[]
        \item
            \defemph{Граф-путь} $ P_n $, $ n \geqslant 0 $~--- это граф вида
            \[
                V(P_n) = \{ v_0, \ldots, v_n \}, \qquad
                E(P_n) = \left\{ \{v_0, v_1\}, \{v_1, v_2\}, \ldots, \{v_{n-1}, v_n\} \right\}
            \]
            Вершины $ v_1 $ и $ v_n $ называются \defemph{концами пути}, а $ n = |E| $~--- \defemph{длиной}.
            \textit{Ещё раз акцентируем внимание на том, что $ n \geqslant 0 $, а вершины нумеруются с нуля.}
        \item
            \defemph{Граф-цикл} $ C_n $, $ n \geqslant 3 $~--- это граф вида
            \[
                V(G) = \{ v_1, \ldots, v_n \}, \qquad
                E(G) = \left\{ \{v_1, v_2\}, \{v_2, v_3\}, \ldots, \{v_{n-1}, v_n\}, \{v_n, v_1\} \right\}
            \]
            \textit{Ещё раз акцентируем внимание на том, что $ n \geqslant 3 $.}
        \item
            \defemph{Полный граф} (или \defemph{граф-клика}) $ K_n(V, E) $, $ n \geqslant 1 $~--- это граф, заданный равенством
            \[
                K_n(V, E) = \left( V, \binom{V}{2} \right), \; n = |V| \qquad \left(\text{ то есть } E = \binom{V}{2} \right)
            \]
        \item
            \defemph{Граф-звезда} $ S_n $, $ n \geqslant 0 $~--- это граф вида
            \[
                V(S_n) = \{ v_0, v_1, \ldots, v_n \} \qquad
                E(S_n) = \left\{ \{v_0, v_1\}, \{v_0, v_2\}, \ldots, \{v_0, v_n\} \right\}
            \]
            \textit{Ещё раз акцентируем внимание на том, что $ n \geqslant 0 $, а вершины нумеруются с нуля.}
        \item
            \defemph{Пустой граф}~--- граф, у которого $ V = E = \varnothing $.
            Его принято обозначать тоже $ \varnothing $, хотя, формально, это $ (\varnothing, \varnothing) \neq \varnothing $.
    \end{enumerate}
\end{definition}

\begin{figure}[ht!]
    \center
    \raisebox{-0.5\height}{  % Центрирование по вертикали.
    \begin{tikzpicture}
        \node[circle,fill,inner sep=1.5pt,label=below:$v_0$] (v0)  at (0,0) {};
        \node[circle,fill,inner sep=1.5pt,label=below:$v_1$] (v1)  at (1,0) {};
        \node[circle,fill,inner sep=1.5pt,label=below:$v_2$] (v2)  at (2,0) {};
        \node[circle,fill,inner sep=1.5pt,label=below:$v_3$] (v3)  at (3,0) {};
        \node[circle,fill,inner sep=1.5pt,label=below:$v_4$] (v4)  at (4,0) {};
        \node (P4) at (2,-1) {$ P_4 $};

        \path [-] (v0) edge node {}  (v1);
        \path [-] (v1) edge node {}  (v2);
        \path [-] (v2) edge node {}  (v3);
        \path [-] (v3) edge node {}  (v4);
    \end{tikzpicture}
    }%
    %
    \hspace{2\baselineskip}%
    %
    \raisebox{-0.5\height}{
    \begin{tikzpicture}
        \node[circle,fill,inner sep=1.5pt,label=below:$v_1$] (v1)  at (-54:2) {};
        \node[circle,fill,inner sep=1.5pt,label=right:$v_2$] (v2)  at (18:2)  {};
        \node[circle,fill,inner sep=1.5pt,label=above:$v_3$] (v3)  at (90:2)  {};
        \node[circle,fill,inner sep=1.5pt,label=left:$v_4$]  (v4)  at (162:2) {};
        \node[circle,fill,inner sep=1.5pt,label=below:$v_5$] (v5)  at (234:2) {};
        \node (C5) at (0,-2.5) {$ C_5 $};

        \path [-] (v1) edge node {}  (v2);
        \path [-] (v2) edge node {}  (v3);
        \path [-] (v3) edge node {}  (v4);
        \path [-] (v4) edge node {}  (v5);
        \path [-] (v5) edge node {}  (v1);
    \end{tikzpicture}
    }%

    \vspace{\baselineskip}

    \raisebox{-0.5\height}{
    \begin{tikzpicture}
        \node[circle,fill,inner sep=1.5pt,label=below:$v_1$] (v1)  at (-54:2) {};
        \node[circle,fill,inner sep=1.5pt,label=right:$v_2$] (v2)  at (18:2)  {};
        \node[circle,fill,inner sep=1.5pt,label=above:$v_3$] (v3)  at (90:2)  {};
        \node[circle,fill,inner sep=1.5pt,label=left:$v_4$]  (v4)  at (162:2) {};
        \node[circle,fill,inner sep=1.5pt,label=below:$v_5$] (v5)  at (234:2) {};
        \node (K5) at (0,-2.5) {$ K_5 $};

        \path [-] (v1) edge node {}  (v2);
        \path [-] (v1) edge node {}  (v3);
        \path [-] (v1) edge node {}  (v4);
        \path [-] (v1) edge node {}  (v5);
        \path [-] (v2) edge node {}  (v3);
        \path [-] (v2) edge node {}  (v4);
        \path [-] (v2) edge node {}  (v5);
        \path [-] (v3) edge node {}  (v4);
        \path [-] (v3) edge node {}  (v5);
        \path [-] (v4) edge node {}  (v5);
    \end{tikzpicture}
    }%
    %
    \hspace{2\baselineskip}%
    %
    \raisebox{-0.5\height}{
    \begin{tikzpicture}
        \node[circle,fill,inner sep=1.5pt,label={[label distance = 5]below:$v_0$}] (v0)  at (0,0) {};
        \node[circle,fill,inner sep=1.5pt,label=below:$v_1$] (v1)  at (-54:2) {};
        \node[circle,fill,inner sep=1.5pt,label=right:$v_2$] (v2)  at (18:2)  {};
        \node[circle,fill,inner sep=1.5pt,label=above:$v_3$] (v3)  at (90:2)  {};
        \node[circle,fill,inner sep=1.5pt,label=left:$v_4$]  (v4)  at (162:2) {};
        \node[circle,fill,inner sep=1.5pt,label=below:$v_5$] (v5)  at (234:2) {};
        \node (S5) at (0,-2.5) {$ S_5 $};

        \path [-] (v0) edge node {}  (v1);
        \path [-] (v0) edge node {}  (v2);
        \path [-] (v0) edge node {}  (v3);
        \path [-] (v0) edge node {}  (v4);
        \path [-] (v0) edge node {}  (v5);
    \end{tikzpicture}
    }%

    \caption{базовые графы}
    \label{fig:graphs:basic_graphs}
\end{figure}

%\FloatBarrier


\subsection{Теоретико-множественные операции с графами}
\label{subsec:graphs:graph_operations}

Известные нам теоретико-множественные операции можно обобщить на графы.
\begin{definition}
    \label{definition:graphs:set_theory_operations}
    Пусть $ G(V, E) $ и $ H(W, I) $~--- графы.
    Тогда \defemph{объединение}, \defemph{персечение} $ G $ и $ H $, а также \defemph{дополнение} $ G $ определяются как, соответственно,
    \[
        G \cup H = \left( V \cup W, E \cup I \right), \quad
        G \cap H = \left( V \cap W, E \cap I \right), \quad
        G^c = \left( V, \binom{V}{2} \setminus E \right)
    \]
    Множество $ \binom{V}{2} \setminus E $ называется множеством \defemph{нерёбер} графа $ G(V, E) $.
\end{definition}

\begin{remark}
    \label{remark:graphs:complement}
    $ G \cup G^c = K_{|V(G)|} $.
\end{remark}

\begin{Exercise}[counter=SecExercise]
    \noindent
    Существует ли такой граф-цикл, дополнение которого тоже является графом-циклом?
\end{Exercise}

\begin{Answer}
    \noindent
    Так как при дополнении число вершин не меняется, если такой граф и есть, то, в силу \ref{remark:graphs:subsets_cardinality} и \ref{remark:graphs:complement}, выполено равенство
    \[
        |V(C_n)| = n = \frac{n(n-1)}{2} - n = |V(K_n)| - |V(C_n)|, \qquad n \geqslant 1
    \]
    Отсюда $ n = 5 $.
    Тогда легко привести единственный пример такого графа:
    \begin{center}
    \raisebox{-0.5\height}{
    \begin{tikzpicture}
        \node[circle,fill,inner sep=1.5pt,label=below:$v_1$] (v1)  at (-54:2) {};
        \node[circle,fill,inner sep=1.5pt,label=right:$v_2$] (v2)  at (18:2)  {};
        \node[circle,fill,inner sep=1.5pt,label=above:$v_3$] (v3)  at (90:2)  {};
        \node[circle,fill,inner sep=1.5pt,label=left:$v_4$]  (v4)  at (162:2) {};
        \node[circle,fill,inner sep=1.5pt,label=below:$v_5$] (v5)  at (234:2) {};
        \node (C5) at (0,-2.5) {$ C_5 $};

        \path [-] (v1) edge node {}  (v2);
        \path [-] (v2) edge node {}  (v3);
        \path [-] (v3) edge node {}  (v4);
        \path [-] (v4) edge node {}  (v5);
        \path [-] (v5) edge node {}  (v1);
    \end{tikzpicture}
    }%
    %
    \hspace{2\baselineskip}
    %
    \raisebox{-0.5\height}{
    \begin{tikzpicture}
        \node[circle,fill,inner sep=1.5pt,label=below:$v_1$] (v1)  at (-54:2) {};
        \node[circle,fill,inner sep=1.5pt,label=right:$v_2$] (v2)  at (18:2)  {};
        \node[circle,fill,inner sep=1.5pt,label=above:$v_3$] (v3)  at (90:2)  {};
        \node[circle,fill,inner sep=1.5pt,label=left:$v_4$]  (v4)  at (162:2) {};
        \node[circle,fill,inner sep=1.5pt,label=below:$v_5$] (v5)  at (234:2) {};
        \node (C5c) at (0,-2.5) {$ C_5^c = C'_5 $};

        \path [-] (v1) edge node {}  (v3);
        \path [-] (v1) edge node {}  (v4);
        \path [-] (v2) edge node {}  (v4);
        \path [-] (v2) edge node {}  (v5);
        \path [-] (v3) edge node {}  (v5);
    \end{tikzpicture}
    }%

    \end{center}
\end{Answer}



\subsection{Подграфы}
\label{subsec:graphs:subgraphs}

Иногда бывает интересно исследовать какую-то часть графа как отдельный граф.
Это может понадобиться как при решении реальных задач, так и при доказательстве вспомогательных фактов.

\begin{definition}
    \label{definition:graphs:subgraph}
    Граф $ H(W, I) $ является \defemph{(рёберным) подграфом} графа $ G(V, E) $ $ \defarr $ $ W \subseteq V $ и $ I \subseteq E $.
    Это обозначается как $ H \subseteq G $.
    Случай, когда $ H \subseteq G $ и $ H \neq G $, обозначается как $ H \varsubsetneq G $;
    если при этом ещё и $ H \neq \varnothing $, то подграф называется \defemph{несобственным}.
\end{definition}

\begin{definition}
    Пусть $ G(V, E) $~--- граф, $ U \subseteq V $.
    Будем называть \defemph{индуцированным (множеством $ U $) подграфом} граф $ G[U] \defeq \left( U, \binom{U}{2} \cap E \right) $.
    %Индуцированный подграф обозначается $ G[U] $.
\end{definition}

Неформально, индуцированный множеством $ U $ подграф~--- это подграф на вершинах $ U $, в ктором провели все возможные рёбра, которые есть в исходном графе.

\begin{definition}
    Множество $ U \subseteq V $ называется \defemph{независимым} множеством вершин графа $ G(V, E) $ $ \defarr $ $ G[U] $ не содержит рёбер.
\end{definition}

\begin{statement}
    \label{statement:graphs:subgraph_set_theory_operations}
    Пусть $ G(V, E) $~--- граф, $ U_1 \subseteq V $, $ U_2 \subseteq V $.
    Тогда $ G[U_1 * U_2] = G[U_1] * G[U_2] $, где вместо <<$ * $>> можно взять любую определённую в \ref{definition:graphs:set_theory_operations} операцию.
    Также верно и $ G^c[U_1] = (G[U_1])^c $.
\end{statement}

\begin{definition}
    \defemph{Подграфом-путём}/\defemph{циклом}/\defemph{кликой}/\defemph{звездой} некоторого графа $ G $ называется подграф $ G $,
    являющийся путём/циклом/полным графом/звездой соответственно.
\end{definition}

Данное определение естественным образом обобщается и на любые другие именные частные случаи графов.


\begin{Exercise}[counter=SecExercise, label={exercise:graphs:clique_independent_set}]
    \noindent
    Докажите, что граф содержит клику на $ n $ вершинах тогда и только тогда,
    когда его дополнение содержит независимое множество на $ n $ вершинах.
\end{Exercise}

\begin{Answer}
    \noindent
    Докажем более сильное утверждение:
    для любого графа $ G(V, E) $ и любого подмножества $ U \subseteq V $ верно
    \[
        H' = G[U] = K_n \quad \Longleftrightarrow \quad H'' = G^c[U] = (U, \varnothing) = K_n^c,
    \]
    %где $ H' $ рассматривается как подграф $ G $, а $ H'' $~--- $ G^c $.
    Действительно, данное утверждение напрямую следует из определения клики:
    раз все вершины в клике соединены рёбрами, то после взятия дополнения в соответствующем подграфе рёбер не будет.
    В обратную сторону доказательство аналогичное: раз в $ (U, \varnothing) $ нет рёбер,
    то после взятия дополнения будут проведены рёбра между всеми парами вершин.
    Мы доказали частный случай утврждения \ref{statement:graphs:subgraph_set_theory_operations}.
\end{Answer}


\begin{Exercise}[counter=SecExercise, label={exercise:graphs:disjoined}]
    \noindent
    Про граф известно, что в нём $ 1000 $ вершин и $ 2022 $ ребра.
    Верно ли, что в таком графе может не оказаться ни одного пути длины $ 64 $?
\end{Exercise}

\begin{Answer}
    \noindent
    Верно, приведём пример.
    Будем строить граф $ G $ следующим образом:
    сначала добавим в граф $ 200 $ клик на пяти вершинах;
    получим $ 1000 $ вершин и $ 200 \cdot 5 \cdot 4 / 2 = 2000 $ рёбер.
    Далее выберем произвольно $ 44 $ клики, составим из них $ 22 $ пары
    и проивзольным образом соединим каждую полученную пару ребром.
    В итоге имеем $ 2022 $ ребра.

    Полученный граф разбивается на $ 200 - 22 = 178 $ непересекающихся подграфов,
    не соединённых рёбрами, в каждом из которых не более $ 10 $ вершин.
    Значит, в таком графе невозможно встретить подграф-путь,
    содержащий более $ 9 $ рёбер.
\end{Answer}


\begin{definition}
    Пусть предикат $ P(x) $ определён на множестве графов (он задаёт некоторое свойство/определение, см. \ref{definition:formal_systems:definition}).
    Обозначим $ P_G = \{ x \subseteq G \mid P(x) \} $ множество всех подграфов графа $ G $, удовлетворяющих свойству $ P(x) $.
    \newline
    Подграф $ H \in P_G $ является \defemph{максимальным} среди подграфов со свойством $ P $ $ \defarr $ $ \forall H' \in P_G \;\; (H \subseteq H') \rightarrow (H = H') $.
\end{definition}

\begin{Exercise}[counter=SecExercise, label=exercise:graphs:is_maximal]
    \noindent
    Верно ли, что определение максимального подграфа со свойством $ P $ эквивалентно следующему:
    $ H \in P_G $~--- максимальный подграф графа $ G $ со свойством $ P $ $ \defarr $ $ \forall H' \in P_G \;\; H' \subseteq H $?
\end{Exercise}

\begin{Answer}
    \noindent
    Нет, неверно.
    В частности, если рассмотреть $ P(x) = \text{<<$ x $~--- граф-путь>>} $, $ G = P_2 \sqcup P_1 $%
    \footnote{Операция $ \sqcup $ означает то же самое, что и $ \cup $, просто с такой записью уточняется, что объединяемые множества не пересекаются.},
    то максимальным подграфом-путём в нём будет $ P_2 $, но определению из условия задачи он удовлетворять не будет,
    так как $ P_1 \not\subseteq P_2 $.
\end{Answer}



\subsection{Связность}
\label{subsec:graphs:connectivity}

Граф, полученный в решении задачи \ref{exercise:graphs:disjoined}, состоит как бы из множества <<независимых>>, или \defemph{несвязных} подграфов.
В реальных задачах часто требуется обнаружить подобные случаи (например, чтобы определить недостижимые части страны по карте дорог, или отдельные социальные группы по графу связей).

\begin{definition}
    Вершина $ u $ в графе $ G $ является \defemph{достижимой} из вершины $ v $ $ \defarr $ существует подграф-путь графа $ G $, концами которого являются вершины $ u $ и $ v $.
    Это обозначается как $ u \leadsto v $.
\end{definition}

\begin{remark}
    \label{remark:graphs:connectivity_relation}
    В случае неориентированного графа верно $ (u \leadsto v) \leftrightarrow (v \leadsto u) $ (\defemph{сим\-ме\-трич\-ность}).
    Также $ u \leadsto u $ (\defemph{рефлексивность}) и $ \left[ (u \leadsto v) \wedge (v \leadsto w) \right] \rightarrow (u \leadsto w) $ (\defemph{транзитивность}).
\end{remark}

\begin{definition}
    \label{definition:graphs:connectivity_component}
    \defemph{Компонентой связности} графа $ G(V, E) $ будем называть подграф $ G $, индуцированный на некотором \underline{непустом} множестве $ U \subseteq V $,
    удовлетворяющем свойству $ \forall u, v \in U \; (u \leadsto v) $ и являющемся максимальным относительно него.
    Компонента связности, состоящая из одной вершины, называется \defemph{изолированной вершиной}.
    Граф, являющийся компонентой связности самого себя, называется \defemph{связным}.
\end{definition}


\begin{Exercise}[counter=SecExercise, label={exercise:graphs:connected_graphs_union}]
    \noindent
    Докажите, что если $ H_1 $ и $ H_2 $~--- связные подграфы графа $ G $,
    такие что $ H_1 \cap H_2 \neq \varnothing $, то подграф $ H_1 \cup H_2 $ связен.
\end{Exercise}

\begin{Answer}
    \noindent
    Так как $ H_1 \cap H_2 \neq \varnothing $, $ \exists v_0 \in H_1 \cap H_2 $.
    Рассмотрим произвольные вершины $ u, v \in V(H_1 \cup H_2) $.
    Имеем $ u \leadsto v_0 $, так как $ u, v_0 \in V(H_1) $ или $ u, v_0 \in V(H_2) $,
    а графы $ H_1 $ и $ H_2 $ связны.
    Аналогично $ v_0 \leadsto v $.
    Тогда, по транзитивности связности, $ u \leadsto v $.
\end{Answer}


\begin{Exercise}[counter=SecExercise, label={exercise:graphs:graph_or_complement_connected}]
    \noindent
    Докажите, что граф или его дополнение связны (возможно оба связны).
\end{Exercise}

\begin{Answer}
    \noindent
    Предположим, что $ G $ не является связным.
    То есть $ \exists u, u' \in V(G): u \not\leadsto u' $.
    Но тогда $ \forall v \in V(G) \; \left[ (\{u, v\} \notin E(G)) \vee (\{v, u'\} \notin E(G)) \right] $;
    действительно, если бы для некоторой $ v $ оба этих ребра присутствовали в графе, то $ u \leadsto v \leadsto u' $,
    что не так по предположению.
    Но тогда любая вершина графа $ G $ соединена неребром с $ u $ или $ u' $.
    При взятии дополнения нерёбра перейдут в рёбра;
    при этом $ \{u, u'\} \in E(G^c) $.
    Значит, $ G^c $ окажется связным: $ \forall v, v' \in V(G^c) \; \left[ (v \leadsto u \leadsto u' \leadsto v') \vee (v \leadsto u' \leadsto u \leadsto v') \right] $.
\end{Answer}


\begin{Exercise}[counter=SecExercise, label={exercise:graphs:BFS_sort}]
    \noindent
    Докажите, что вершины связного графа $ G $ можно упорядочить так,
    что для каждого $ i $, $ 1 \leqslant i \leqslant |V(G)| $,
    индуцированный подграф $ G[\{v_1, \ldots, v_i \}] $ будет связным.
\end{Exercise}

\begin{Answer}
    \noindent
    Решение опирается на широко известный алгоритм~--- \emph{поиск в ширину}.
    Он прост для понимания, изучите его сами.
    Для решения задачи достаточно запустить поиск в ширину из любой вершины
    и использовать в качестве нумерации порядок встречи новых вершин.
    Поиск в глубину также подходит.
\end{Answer}


\begin{remark}
    \label{remark:graphs:cc_partition}
    Позже будет доказано, что из замечания \ref{remark:graphs:connectivity_relation} следует, что любой граф разбивается на компоненты связности, причём единственным образом:
    \[
        G = H_1 \sqcup H_2 \sqcup \ldots \sqcup H_k
    \]
\end{remark}


\begin{Exercise}[counter=SecExercise, label={exercise:graphs:max_edges_in_disconnected_graph}]
    \noindent
    Какое максимальное число рёбер может быть в несвязном графе с $ n $ вершинами?
\end{Exercise}

\begin{Answer}
    \noindent
    Пусть граф $ G $ на $ n $ вершинах несвязный и имеет максимальное количество рёбер.
    Рассмотрим его разбиение на компоненты связности: $ G = H_1 \sqcup H_2 \sqcup \ldots \sqcup H_k $.
    Заметим, что каждая компонента связности должна быть кликой:
    действительно, в противном случае недостающие до клики рёбра можно было бы добавить в $ G $,
    не нарушая свойства несвязности.

    Заметим, что компонент связности всего две,
    так как в противном случае можно было бы соединить новым ребром какую-либо пару компонент связности,
    не делая граф связным.

    Наконец, заметим, что одна из двух компонент связности должна состоять ровно из одной вершины.
    Действительно, пусть это не так.
    Возьмём тогда меньшую по размеру компоненту связности, пусть в ней будет $ m \leqslant n/2 $ вершин.
    Возьмём произвольную вершину из выбранной компоненты связности и удалим всё рёбра,
    исходящие из данной вершины.
    Далее соединим выбранную вершину рёбрами со всеми вершинами противоположной компоненты связности.
    В результате таких действий число рёбер увеличится ($ \Delta |E| = (n - m) - (m - 1) = n - 2 m + 1 > n - 2 n / 2 = 0 $).

    Таким образом, максимальное число рёбер достигается в случае, когда $ G $ является кликой на $ (n - 1) $-ой вершине,
    к которой добавили одну изолированную вершину.
    Тогда максимальное число рёбер~--- $ (n - 1)(n - 2) / 2 $.
\end{Answer}


\begin{Exercise}[counter=SecExercise, label={exercise:graphs:union_of_cycles}]
    \noindent
    Каждая вершина графа $ G $ имеет степень 2.
    Докажите, что $ G = H_1 \sqcup H_2 \sqcup \ldots \sqcup H_k $, где $ H_i $~--- граф цикл.
\end{Exercise}

\begin{Answer}
    \noindent
    Рассмотрим следующий алгоритм:
    из графа выбирается некоторая вершина и одна соседняя с ней,
    и начинается обход до тех пор, пока не встретится ранее пройдённая вершина
    (это точно случится, так как вершин конечное число).
    Так как степень каждой вершины равна двум, обход однозначно определяется первыми двумя вершинами (старт и направление).
    В результате обхода будет получен подграф-цикл, являющийся компонентой связности в $ G $.
    Убирая полученную компоненту связности из графа и повторяя процедуру до тех пор, пока $ G $ не окажется пустым,
    получаем искомое разбиение.
\end{Answer}


\begin{Exercise}[counter=SecExercise]
    \noindent
    Пусть $ G $ и $ H $~--- простые графы, причём $ G \cap H = \varnothing $.
    Определим граф $ G \times H $ следующим образом:
    \[
        V(G \times H) = \left\{ \{u, v\} \mid u \in V(G), v \in V(H) \right\}
    \]
    \begin{multline*}
        %E(G \times H) = \Big\{ \big\{ \{u_1, v_1 \}, \{u_2, v_2 \} \big\} \, \Big| \, \left( \{u_1, u_2\} \in E(G) \right) \wedge \left( \{v_1, v_2\} \in E(H) \right) \Big\}
        E(G \times H) = \Big\{ \big\{ \{u_1, v \}, \{u_2, v \} \big\} \, \Big| \, \left( \{u_1, u_2\} \in E(G) \right) \wedge \left( v \in V(H) \right) \Big\} \cup \\
        \cup \Big\{ \big\{ \{u, v_1 \}, \{u, v_2 \} \big\} \, \Big| \, \left( \{v_1, v_2\} \in E(H) \right) \wedge \left( u \in V(G) \right) \Big\}
    \end{multline*}
    Пусть $ G $ имеет $ n $ компонент связности, а $ H $~--- $ m $.
    Сколько компонент связности имеет $ G \times H $?
    Как они устроены?
\end{Exercise}

\begin{Answer}
    \noindent
    Заметим, что $ \{ u_1, v_1 \} \leadsto \{ u_2, v_2 \} \; \Longleftrightarrow \; (u_1 \leadsto u_2) \wedge (v_1 \leadsto v_2) $.
    \newline
    Действительно,
    \begin{enumerate}
        \item[$ \Leftarrow $]
            Если $ u_1 \leadsto u_k $ и $ v_1 \leadsto v_l $, то существуют пути $ P_G $ и $ P_H $ из $ u_1 $ в $ u_k $ и из $ v_1 $ в $ v_l $ соответственно
            (нумерация введена уже для вершин путей).

            Заметим, что все рёбра вида $ \big \{ \{u_i, v_j\}, \{u_{i+1}, v_j\} \big \} $ и $ \big \{ \{u_i, v_j\}, \{u_i, v_{j+1}\} \big \} $,
            где $ \{ u_i, u_{i+1} \} \in E(P_G) $ и $ \{ v_j, v_{j+1} \} \in E(P_H) $,
            лежат в $ E(G \times H) $ по построению.

            Но тогда в $ G \times H $ есть путь вида
            \[
                \{ u_1, v_1 \} \to \{ u_2, v_1 \} \to \ldots \to \{ u_k, v_1 \} \to \{ u_k, v_2 \} \to \ldots \to \{ u_k, v_l \}
            \]
            То есть $ \{ u_1, v_1 \} \leadsto \{ u_k, v_l \} $.
        \item[$ \Rightarrow $]
            Если $ \{ u_1, v_1 \} \leadsto \{ u_k, v_l \} $, то существует путь $ P_{G \times H} $ из $  \{ u_1, v_1 \} $ в $ \{ u_k, v_l \} $.

            Из определения $ G \times H $ следует, можно перенумеровать $ u_i $ и $ u_j $ так, что путь $ P_{G \times H} $ имеет рёбра только вида
            $ \big \{ \{u_i, v_j\}, \{u_{i+1}, v_j\} \big \} $ или $ \big \{ \{u_i, v_j\}, \{u_i, v_{j+1}\} \big \} $.

            Но тогда в $ G $ и $ H $ есть пути $ u_1 \to u_2 \to \ldots \to u_k $ и $ v_1 \to v_2 \to \ldots \to v_l $ соответственно.
            То есть $ u_1 \leadsto u_k $ и $ v_1 \leadsto v_l $.
    \end{enumerate}
    Но тогда получаем, что все компоненты связности графа $ G \times H $ будут иметь вид $ G_i \times H_j $,
    где $ G_i $ и $ H_j $~--- компоненты связности графов $ G $ и $ H $ соответственно.
    Тогда ответ: $ n \cdot m $.
\end{Answer}



\subsection{Деревья}
\label{subsec:graphs:trees}

Часто бывает удобно исследовать, в некотором смысле, \defemph{минимальные} по числу рёбер связные графы.
Такие графы называются \defemph{деревьями} и обладают множеством полезных свойств.

\begin{definition}
    \defemph{Деревом} назовём \defemph{минимально связный граф}, то есть граф, теряющий свойство связности при удалении любого ребра.
\end{definition}

\begin{Exercise}[counter=SecExercise, label={exercise:graphs:star_tree}]
    \noindent
    Дерево имеет $ 2022 $ вершины.
    Верно ли, что в нём найдется путь длины $ 3 $?
\end{Exercise}

\begin{Answer}
    \noindent
    Нет, рассмотрим $ S_{2022} $.
    Это действительно дерево, так как удаление любого ребра ведёт к образованию новой компоненты связности.
    Но все пути в данном графе имею длину не больше двух.
\end{Answer}

Если подходить к деревьям именно со стороны минимальности, то кажется осмысленным сначала привести некоторую оценку,
насколько вообще можно сделать <<малым>> граф, сохраняя его связность.

\begin{theorem}
    Пусть $ \text{\#КС}(G) $ обозначает число компонент связности некоторого графа $ G(V, E) $.
    Тогда $ \forall G \;\, \text{\#КС}(G) \geqslant |V(G)| - |E(G)| $,
    причём случай равенства достижим.
\end{theorem}

\begin{proof}
    По индукции для фиксированного $ |V| $ по числу рёбер от $ 0 $ до $ |V| $.
\end{proof}

\begin{corollary}
    Если граф связный, то $ |E| \geqslant |V| - 1 $.
\end{corollary}

Из следствия очевидным образом можно получить эквивалентное определение дерева.
На самом деле этим множество эквивалентных определения дерева не ограничивается.

\begin{theorem}
    \label{theorem:graphs:trees_defs}
    Следующие свойства эквивалентны:
    \begin{enumerate}[label=(\arabic*)]
        \item
            \label{item:graphs:trees_defs:min_connected}
            Граф $ G(V,E) $ минимально связный.
        \item
            \label{item:graphs:trees_defs:e_v_minus_one}
            Граф $ G(V,E) $ связный и $ |E| = |V| - 1 $.
        \item
            \label{item:graphs:trees_defs:acyclic}
            Граф $ G(V,E) $ связный и \defemph{ациклический} (не имеет подграфов-циклов).
        \item
            \label{item:graphs:trees_defs:two_paths}
            В графе $ G(V,E) $ из любой вершины в любую другую есть путь, причём единственный.
    \end{enumerate}
\end{theorem}

В ходе доказательства теоремы выше обычно используется следующая (полезная и в отдельности) лемма:
\begin{lemma}
    \label{lemma:graphs:cycle_two_paths}
    Граф содержит цикл тогда и только тогда, когда между некоторыми вершинами графа есть два различных пути.
\end{lemma}

Также стоит отметить следующее утверждение:
\begin{statement}
    \label{statement:graphs:leaf_in_every_tree}
    В любом дереве более чем с одной вершиной есть хотя бы две вершины степени $ 1 $.
\end{statement}

\begin{proof}
    Следует из теоремы \ref{theorem:graphs:sum_of_degs} и пункта \ref{item:graphs:trees_defs:e_v_minus_one} теоремы \ref{theorem:graphs:trees_defs}.
\end{proof}

\begin{Exercise}[counter=SecExercise, label={exercise:graphs:small_tree_big_deg}]
    \noindent
    Существует ли дерево на $ 9 $ вершинах, в котором $ 2 $ вершины имеют степень $ 5 $?
\end{Exercise}

\begin{Answer}
    \noindent
    Нет. Предположим противное.
    Так как в дереве не может быть висячих вершин, степень остальных $ 7 $ вершин не меньше $ 1 $.
    По теореме о рукопожатиях
    \[
        |E| > \frac{7 \cdot 1 + 2 \cdot 5}{2} = 8.5
    \]
    Но тогда $ |E| \neq |V| - 1 = 8 $.
\end{Answer}

\begin{Exercise}[counter=SecExercise, label={exercise:graphs:more_then_half_leaves}]
    \noindent
    В дереве нет вершин степени $ 2 $.
    Докажите, что количество висячих вершин (т.е. вершин степени $ 1 $) больше половины общего количества вершин.
\end{Exercise}

\begin{Answer}
    \noindent
    Пусть общее число вершин~--- $ n $, число висячих вершин~--- $ m $.
    По условию задачи $ n - m $ вершин имеют степень не меньше трёх.
    По теореме о рукопожатиях и одному из определений дерева имеем
    \[
        n - 1 = |E| > \frac{m \cdot 1 + (n - m) \cdot 3}{2}
    \]
    То есть $ 2 n - 2 > 3 n - 2 m $,
    что эквивалентно $ 2 m > n + 2 $ или $ m > n/2 + 1 $.
\end{Answer}

\begin{definition}
    \label{definition:graphs:spanning_tree}
    Подгораф $ H \subseteq G $ называют \defemph{остовным деревом} графа $ G $ в случае,
    если $ H $~--- дерево, и $ V(H) = V(G) $.
\end{definition}

\begin{statement}
    \label{statement:graphs:connected_has_spanning_tree}
    Любой связный граф имеет остовное дерево.
\end{statement}

\begin{proof}
    Пусть $ G $~--- связный граф.
    Покажем путём последовательного удаления рёбер, что он имеет остовное дерево.

    Если $ G $ изначально является деревом,
    то он же является и своим остовным деревом.
    Поэтому рассмотрим случай, когда $ G $~--- не дерево.
    Тогда, согласно пункту \ref{item:graphs:trees_defs:two_paths} теоремы \ref{theorem:graphs:trees_defs},
    $ G $ имеет два разных подграфа-пути с общими концами.
    Можно рассмотреть любой из этих подграфов и удалить в нём любое ребро;
    граф останется связным благодаря второму подграфу-пути,
    связывающему концы того подграфа-пути, из которого удалили ребро.
    Продолжаем удалять рёбра до тех пор, пока не окажется, что в графе нет двух разных путей с общими концами;
    в результате получим дерево но всех вершинах графа $ G $, то есть остовное дерево.
\end{proof}

\begin{Exercise}[counter=SecExercise, label={exercise:graphs:can_delete_one_vertex}]
    \noindent
    Имеется связный граф.
    Докажите, что в нём можно выбрать одну из вершин так,
    что после её удаления вместе со всеми ведущими из неё рёбрами останется связный граф.
\end{Exercise}

\begin{Answer}
    \noindent
    Случай, когда в графе не более одной вершины, несколько неоднозначный с точки зрения определения связности,
    а потому рассматриваться не будет;
    считаем, что $ |V(G)| \geqslant 2 $.

    Так как граф связный, он имеет остовное дерево (см. утверждение \ref{statement:graphs:connected_has_spanning_tree}).
    Рассмотрим любую висячую вершину любого остовного дерева
    (такая вершина найдётся в любом дереве согласно утверждению \ref{statement:graphs:leaf_in_every_tree}).
    Заметим, что её и все исходящие из неё рёбра можно удалить из исходного графа,
    и он не потеряет связности.
    Действительно, после удаления висячей вершины вернем в остовное дерево все остальные рёбра из исходного графа,
    за исключением удалённых.
    Так как остовное дерево связно, а добавление ребра не нарушает этого свойства, утверждение доказано.
\end{Answer}



\subsection{Расстояние между вершинами. Диаметр графа}
\label{subsec:graphs:dist_and_diam}

Поскольку графы часто используются для моделирования транспортных сетей, имеет смысл ввести некоторое <<расстояние>> между двумя вершинами, а также характеристики, связанные с ним.

\begin{definition}
    Пусть $ G $~--- связный граф, $ u, v \in V(G) $.
    Тогда \defemph{расстоянием} между вершинами $ u $ и $ v $ называется длина кратчайшего пути между ними:
    \[
        \rho(u, v) = \min_{P_{u \leadsto v} \subseteq G} |E(P_{u \leadsto v})|,
    \]
    где $ P_{u \leadsto v} $~--- подграф-путь из $ u $ в $ v $.%
    \footnote{Это исключительно авторское обозначение. Не рекомендуется использовать без определения.}
\end{definition}

\begin{definition}
    \defemph{Диаметром} графа $ G $ называется наибольшее растояние между какими-то двумя его вершинами:
    \[
        \diam G = \max_{u, v \in V(G)} \rho(u, v)
    \]
\end{definition}

\begin{definition}
    \defemph{Центром} графа $ G $ называется вершина, наименее удалённая от всех остальных, то есть
    \[
        c(G) = \argmin_{u \in V(G)} \max_{v \in V(G)} \rho(u, v)
    \]
    \textit{На самом деле, таких вершин может быть несколько. Тогда, в зависимости от соглашения, под $ c(G) $ понимают либо их множество, либо любую из них.}
    Максимальное расстояние от центра графа до какой-либо вершины называется \defemph{радиусом} графа $ G $:
    \[
        \rad G = \max_{v \in V(G)} \rho\left( c(G), v \right) = \min_{u \in V(G)} \max_{v \in V(G)} \rho(u, v)
    \]
\end{definition}

\begin{Exercise}[counter=SecExercise, label={exercise:graphs:rad_and_diam}]
    \noindent
    \begin{enumerate}
        \item Докажите, что $ \rad(G) \leqslant \diam(G) \leqslant 2 \rad(G) $.
        \item Приведите пример графа $ G $, для которого $ \rad(G) = \diam(G) $.
    \end{enumerate}
\end{Exercise}

\begin{Answer}
    \noindent
    \begin{enumerate}
        \item
            Максимальное расстояние от центра до какой-либо другой вершины всегда не больше расстояния между любой другой парой вершин:
            \[
                \rad G = \max_{v \in V(G)} \rho\left( c(G), v \right) \leqslant \max_{u, v \in V(G)} \rho(u, v) = \diam(G)
            \]
            С другой стороны, расстояние в графе удовлетворяет неравенству треугольника,
            так как кратчайший путь не длиннее любого другого пути, в том числе и пути,
            проходящего через некоторую фиксированную вершину.
            Поэтому
            \[
                \forall u, v \in V(G) \quad \rho(u, v) \leqslant \rho(u, c(G)) + \rho(c(G), v)
            \]
            Отсюда
            \[
                \diam(G) = \max_{u, v \in V(G)} \rho(u, v) \leqslant \max_{u \in V(G)} \rho(u, c(G)) + \max_{v \in V(G)} \rho(c(G), v)
            \]
        \item Граф~--- $ P_1 $.
    \end{enumerate}
\end{Answer}



\subsection{Правильные раскраски}
\label{subsection:graphs:coloring}

При исследовании графов часто возникает задача разбиения вершин на некоторое количество групп (задача \defemph{раскраски}).
Иногда также получается, что какая-то задача из совершенно другой области математики может быть проинтерпретированна как задача раскраски некоторого графа.
Поэтому так важен вопрос построения \defemph{раскрасок}, обладающих определёнными свойствами. %, важен как с теоретической, так и с практической точек зрения.

\begin{definition}
    \defemph{Раскраской} (\defemph{$ k $-раскраской}) графа $ G $ назывется функция $ f $,
    принимающая в качестве аргумента $ v \in V(G) $, и выдающая число из $ \{ 1, \ldots, k \} $.
    То есть $ f: \; V(G) \to \{1, \ldots, k \} $.
\end{definition}

\begin{definition}
    Раскраска $ f $ графа $ G $ является \defemph{правильной} $ \defarr $ никакие две смежные вершины не окрашены в один цвет, то есть
    \[
        \forall u, v \in V(G) \;\, \left( \{u, v\} \in E(G) \right) \rightarrow \left( f(u) \neq f(v) \right)
    \]
\end{definition}

\begin{Exercise}[counter=SecExercise, label={exercise:graphs:clique_max_colors}]
    \noindent
    Докажите, что если $ G $ содержит клику размера $ n $,
    то его вершины нельзя раскрасить правильно в $ n - 1 $ цветов.
\end{Exercise}

\begin{Answer}
    \noindent
    В клике $ n $ вершин, а доступных цветов всего $ n - 1 $.
    Значит, согласно принципу Дирихле, хотя бы две вершины в клике будут раскрашены в один цвет.
    Поскольку эта пара соединена ребром, раскраска не может быть правильной.
\end{Answer}

\begin{definition}
    Граф $ G $ является \defemph{$ k $-раскрашиваемым} $ \defarr $ для $ G $ существует правильная раскраска из $ k $ цветов.
    \defemph{Хроматическим числом} графа $ G $ называется число $ \chi(G) $, равное минимальному $ k $ такому, что $ G $ $ k $-раскрашиваемый.
\end{definition}

Задача проверки $ 2 $-раскрашиваемости (двураскрашиваемости) графа является сравнительно <<лёгкой>>.
Полного перебора позволяет избежать следующий критерий:
\begin{theorem}
    \label{theorem:graphs:2_coloring_criterion}
    Граф $ G $ является двураскрашиваемым тогда и только тогда, когда в нём нет циклов нечётной длины.
\end{theorem}

\begin{corollary}
    \label{corollary:graphs:tree_2_coloring}
    Дерево двураскрашиваемо.
\end{corollary}

\begin{proof}
    Раз в дереве нет циклов, то нет и циклов нечётной длины.
\end{proof}

\begin{Exercise}[counter=SecExercise, label={exercise:graphs:tree_2_coloring}]
    \noindent
    Сколько есть правильных 2-раскрасок у дерева?
\end{Exercise}

\begin{Answer}
    \noindent
    Заметим, что цвет произвольной вершины дерева однозначно задаёт правильную 2-рас\-крас\-ку:
    цвет любой другой вершины можно взять соответственно чётности или нечётности расстояния от неё до зафиксированной ранее вершины.
    Такая раскраска является правильной, так как в дереве нет соединённых ребром вершин, равноудалённых от фиксированной
    (иначе в дереве есть как минимум два разных пути из одной вершины в другую).
    Так как цвет фиксированной вершины можно выбрать двумя способами,
    всего правильных 2-раскрасок любого дерева - две.
\end{Answer}

Общая задача~--- задача определения $ \chi(G) $ в случае, когда оно заведомо больше двух~--- является уже гораздо более <<сложной>>:
на такущий момент уровень развития науки человеческой цивилизации не позволяет придумать непереборный алгоритм.
Казалось бы, и что с того?
Задача, с виду, не очень практически важная.
На самом деле, важная, это иллюстрирует следующее упражнение:

\begin{Exercise}[counter=SecExercise]
    \noindent
    Пусть имеется система вида
    \[
        \begin{cases}
            x_{k_1} \oplus x_{l_1} = 1 \\
            \ldots \\
            x_{x_n} \oplus x_{l_n} = 1 \\
        \end{cases}
    \]
    Как можно проверить, имеет ли она решение?
\end{Exercise}

\begin{Answer}
    \noindent
    Построим граф, вершинами в котором будут $ x_i $.
    Проведём в графе рёбра между всеми парами вершин, которые фигурируют в системе уравнений из условия.
    Тогда нетрудно проверить, что система имеет решение тогда и только тогда, когда полученный граф двураскрашиваем:
    достаточно интерпретировать цвета как значения $ x_i $.
\end{Answer}

Это довольно игрушечная задача, но, оказывается, аналогичные \defemph{сводимости} можно построить и для исследования решений более сложных уравнений в алгебре логики.
Таким образом, умение эффективно раскрашивать граф в три или более цвета, или хотя бы определять, можно ли это сделать, позволяет эффективно решать многие задачи алгебры логики
(на самом деле, даже задачи математической логики вообще).



\subsection{Эйлеровы маршруты}
\label{subsec:graphs:Euler_walks}

Путь в графе~--- это довольно узкое понятие.
Путь, например, не может иметь самопересечений, что сильно ограничивает область использования данного термина.
В этой связи вводится следующее определение:

\begin{definition}
    \defemph{Маршрутом} длины $ n \geqslant 0 $ в графе $ G $ называется последовательность вершин $ v_0, v_1, \ldots, v_n $ такая,
    что $ \forall i \in \{ 0, \ldots, n-1 \} \;\, \left( \{v_i, v_{i+1}\} \in E(G) \right) $.
    Число $ n $ называется \defemph{длиной маршрута}.
    \textit{Отметим отдельно, что одна вершина тоже является маршрутом длины $ 0 $.}
    \\[0.25\baselineskip]
    Вершины $ v_0 $ и $ v_n $ называются \defemph{концами} маршрута;
    говорится, что маршрут \defemph{соединяет} $ v_0 $ и $ v_n $.
    В случае $ v_0 = v_n $ маршрут является \defemph{замкнутым}.
    Говорят, что ребро $ \{x, y\} \in E(G) $ \defemph{лежит} на маршруте, если $ \exists i: \{x, y\} = \{v_i, v_{i+1}\} $.
\end{definition}

\begin{Exercise}[counter=SecExercise, label={exercise:graphs:long_walk}]
    \noindent
    Про граф известно, что в нём $ 1000 $ вершин и $ 2022 $ ребра.
    Верно ли, что в таком графе обязательно есть маршрут длины $ 3000 $.
\end{Exercise}

\begin{Answer}
    \noindent
    В любом графе, имеющим хотя бы одно ребро, найдётся маршрут произвольной длины.
    Пусть $ \{u, v\} \in E(G) $.
    Тогда $ u, v, u, \ldots, u \; (\textnormal{или $ v $}) $~--- это маршрут.
    Его длина может быть выбрана произвольно.
    Таким образом, да, в графе из условия есть маршрут длины $ 3000 $.
\end{Answer}

\begin{Exercise}[counter=SecExercise, label={exercise:graphs:walks_and_coloring}]
    \noindent
    Докажите или опровергните следующие утверждения:
    \begin{enumerate}[label=\textbf{\alph*)}]
        \item
            если в графе есть замкнутый маршрут чётной длины,
            то в графе есть цикл чётной длины.
        \item
            если в графе есть замкнутый маршрут нечётной длины,
            то в графе есть цикл нечётной длины.
    \end{enumerate}
\end{Exercise}

\begin{Answer}
    \noindent
    \begin{enumerate}[label=\textbf{\alph*)}]
        \item
            Утверждение неверное: рассмотрим граф-путь $ P_2 $.
            В нём нет циклов вообще, но следующая последовательность вершин является замкнутым путём чётной длины: $ v_0, v_1, v_0 $.
        \item
            Предположим противное: в графе есть маршрут нечётной длины, но циклов нечётной длины нет.
            Согласно теореме \ref{theorem:graphs:2_coloring_criterion}, данный граф 2-раскрашиваем.
            В таком случае при прохождении маршрута каждое пройденное ребро будет менять цвет текущей вершины на противоположный.
            Если пройти маршрут нечётной длины от начала и до конца, то цвет изменится нечётное число раз.
            Таки образом, цвет конца не совпадает с цветом начала.
            Но по условию нам дан замкнутый маршрут нечётной длины,
            что означает равенство начала и конца.
            Таким образом, имеем противоречие.
            Значит, утверждение верно.
    \end{enumerate}
\end{Answer}


\begin{statement}
    \label{statement:graphs:path_walk_equivalency}
    В графе есть путь между вершинами $ v $ и $ v' $ тогда и только тогда, когда между ними есть и маршрут.
\end{statement}

\begin{definition}
    Маршрут является \defemph{эйлеровым} $ \defarr $ каждое ребро графа лежит на маршруте, причём вхождение единственно.
\end{definition}

\begin{theorem}
    Связный граф $ G $ содержит замкнутый эйлеров маршрут тогда итолько тогда, когда степень каждой вершины чётна.
\end{theorem}



\subsection{Многодольные графы и паросочетания}
\label{subsec:graphs:multipartite_matching}

На $ k $-раскрашиваемые графы иногда бывает полезно посмотреть с других позиций:
раз ни у какого ребра концы не покрашены в один цвет, все вершины графа можно разбить на \defemph{доли} согласно их цвету,
причём рёбра в графе будут только между вершинами из разных долей.
Это разбиение может являться отражением более сложной природы моделируемых объектов
(например, если граф отображает связи между работниками и их задачами).

\begin{definition}
    Будем называть \defemph{$ k $-дольным графом} такой граф $ G $, для которого $ \exists H_1, \ldots, H_k $:
    $ V(G) = H_1 \sqcup \ldots \sqcup H_k $ и $ \forall e \in E(G) \; \forall i \in \{1, \ldots, k\} \;\, | e \cap H_i | \leqslant 1 $.
    То есть, рёбра проведены только между различными \defemph{долями} $ H_i $.
\end{definition}

\begin{remark}
    Граф $ k $-дольный тогда и только тогда, когда он $ k $-раскрашиваемый.
\end{remark}

\begin{definition}
    \defemph{Полным} $ k $-дольным графом называется граф вида
    \[
        K_{|H_1|, \ldots, |H_k|} = \left( H_1 \sqcup \ldots \sqcup H_k, \bigcup_{\substack{i,j = 1 \\ i \neq j}}^k \big\{ \{u, v\} \mid u \in H_i, v \in H_j \big\} \right)
    \]
    То есть это $ k $-дольный граф, в котором проведены все возможные рёбра между его долями.
\end{definition}


Также может возникнуть задача построения разбиения другого вида: разбиения вершин на непересекающиеся пары, или \defemph{паросочетания}.
%Пусть теперь нас интересует не разнесение вершин исходного графа по разным долям, а выбор непересекающихся пар вершин графа.

\begin{definition}
    \defemph{Паросочетание (на графе $ G $)}~--- это множество рёбер $ M \subseteq E(G) $, в котором ни одна пара (рёбер) не имеет общего конца.
\end{definition}

\begin{definition}
    Вершинами графа $ G $, \defemph{покрытыми} паросочетанием $ M $, назовём множество $ V_M = \{ v \in V(G) \mid \exists e \in M: v \in e \} $
    (множество вершин, смежных с рёбрами из $ M $).
    Паросочетание $ M $ назовём \defemph{совершенным} в случае $ V_M = V(G) $ (все вершины покрыты).
\end{definition}


\begin{figure}[ht!]
    \center
    \begin{tikzpicture}
        \node[circle,fill,inner sep=1.5pt,label=left:$v_1$] (v1)  at (0,0) {};
        \node[circle,fill,inner sep=1.5pt,label=left:$v_2$] (v2)  at (0,-1) {};
        \node[circle,fill,inner sep=1.5pt,label=left:$v_3$] (v3)  at (0,-2) {};
        \node[circle,fill,inner sep=1.5pt,label=left:$v_4$] (v4)  at (0,-3) {};

        \node[circle,fill,inner sep=1.5pt,label=right:$u_1$] (u1)  at (2,0) {};
        \node[circle,fill,inner sep=1.5pt,label=right:$u_2$] (u2)  at (2,-1) {};
        \node[circle,fill,inner sep=1.5pt,label=right:$u_3$] (u3)  at (2,-2) {};
        \node[circle,fill,inner sep=1.5pt,label=right:$u_4$] (u4)  at (2,-3) {};

        \path [-] (v1) edge node {}  (u2);
        \path [-] (v2) edge node {}  (u2);
        \path [-] (v3) edge node {}  (u3);
        \path [-] (v4) edge node {}  (u4);

        \path [-, line width=2] (v1) edge node {}  (u1);
        \path [-, line width=2] (v2) edge node {}  (u4);
        \path [-, line width=2] (v3) edge node {}  (u2);
        \path [-, line width=2] (v4) edge node {}  (u3);
    \end{tikzpicture}
    \caption{пример совершенного паросочетания на двудольном графе}
    \label{fig:graphs:perfect_match}
\end{figure}


В случае двудольных графов задача построения паросочетаний и, в частности,
совершенных паросочетаний может быть мотивирована желанием построить взаимнооднозначное соответствие для как можно большего числа вершин из разных долей
(возвращаясь к примеру с работниками и задачами, построение совершенного паросочетания означает наиболее оптимальное распределение задач по работникам).

В этой связи формулируется теорема, гарантирующая существование совершенного паросочетания в двудольном графе при выполнении определённого условия.
Для строгой формулировки этого условия придётся ввести некоторые вспомогательные понятия.

\begin{definition}
    \defemph{Множеством соседей (окрестностью) вершины} $ v $ графа $ G $ будем называть множество $ N(v) = \{ u \mid \{u, v\} \in E(G) \} $.
    \defemph{Множеством соседей подмножества вершин} $ U \subseteq V(G) $ графа $ G $ будем называть множество $ N(U) = \left( \bigcup_{u \in U} N(u) \right) \setminus U $.
\end{definition}

\begin{theorem}[Холла о свадьбах]
    В двудольном графе с долями $ L $ и $ R $ существует совершенное паросочетание тогда и только тогда,
    когда $ |L| = |R| $ и для любого подмножества $ S \subseteq L $ справедливо $ |N(S)| \geqslant |S| $.
\end{theorem}

\begin{Exercise}[counter=SecExercise, label={exercise:almost_BFS}]
    \noindent
    Пусть $ G $ связный граф, $ v_0 \in V(G) $.
    Пусть $ S_0 = \{ v_0 \} $, $ S_{k+1} = N(S_k) \cup S_k $.
    Задайте множества $ S_k $ явно в терминах расстояний в графе.
    %Найдите минимальное $ k_0 $, при котором $ S_{k_0+1} = S_{k_0} $.
\end{Exercise}

\begin{Answer}
    \noindent
    %Покажем, что $\displaystyle k_0 = \max_{u \in V(G)} \rho(v_0, u) $.

    %Действительно, для этого
    По индукции докажем, что $ S_k = \{ u \in V(G) \mid \rho(v_0, u) \leqslant k \} $.
    \newline
    \textbf{База:}
    $ S_0 = \{ v_0 \} = \{ u \mid \rho(v_0, u) \leqslant 0 \} $ по определению.
    \newline
    \textbf{Шаг:}
    Пусть для $ k $ утверждение истинно.
    Рассмотрим $ k + 1 $.
    Пусть $ \rho(v_0, u) \leqslant k $.
    Тогда $ u \in S_k \subseteq S_{k+1} $.
    %\newline
    Пусть теперь $ \rho(v_0, u) = k + 1 $.
    Кратчайший путь из $ v_0 $ в $ u $ содержит вершину $ u' $, смежную с $ u $ и такую, что $ \rho(v_0, u') = k $.
    Тогда $ u' \in S_k $, из чего следует, что $ u \in N(S_k) \subseteq S_{k+1} $.
    %\newline
    Наконец, пусть $ \rho(v_0, u) > k + 1 $.
    По предположению индукции $ u \notin S_k $.
    Также $ u \notin N(S_k) $, иначе был бы путь из $ v_0 $ в $ u $ длины $ k + 1 $.
    Значит, $ u \notin S_{k+1} $.
    \newline
    По индукции доказано.

    %Из утверждения выше имеем, что $ S_{k+1} \setminus S_{k} = \{ u \mid \rho(v_0, u) = k+1 \} $.
    %Отсюда, если $ S_{k_0+1} = S_{k_0} $, то $ \forall u \in V(G) \;\, \rho(v_0, u) \neq k_0+1 $.
    %Но тогда не имеем путей из $ v_0 $ длины больше $ k_0 $.
    %Это возможно тогда и только тогда, когда $ \displaystyle k_0 \geqslant \max_{u \in V(G)} \rho(v_0, u) $.
    %Отсюда получаем ответ.
\end{Answer}

\begin{Exercise}[counter=SecExercise]
    \noindent
    Используя задачу \ref{exercise:almost_BFS}, постройте алгоритм поиска кратчайших путей из заданной вершины во все остальные в простом неориентированном графе.
\end{Exercise}



\newpage



\section{Ориентированные графы}
\label{sec:oriented_graphs}

\defemph{Ориентированные графы}~--- естественное обобщение неориентированных.
Они получаеются простой заменой неупорядоченной пары на упорядоченную в определении ребра:
\begin{definition}
    \defemph{Ориентированный граф (возможно, с петлями)}~--- это упорядоченная пара $ (V, E) $ множества \defemph{вершин} $ V $
    и \defemph{рёбер} $ E \subseteq V^2 $.
    В дальнейшем будет подразумеваться, что в ориентированном графе петель нет, то есть $ E \cap \{(v, v) \mid v \in V \} = \varnothing $.
\end{definition}

Определения, введённые нами для неориентированных графов, с поправками переносятся на ориентированные.

\begin{definition}
    \defemph{Исходящей степенью} $ d_+(v) $ вершины $ v $ называется число рёбер, началом которых является $ v $,
    то есть $ d_+(v) = |\{ (v, u) \mid u \in V \}| $.
    Симметрично вводится понятие \defemph{входящей степени} $ d_-(v) $ вершины $ v $: $ d_-(v) = |\{ (u, v) \mid u \in V \}| $.
\end{definition}

Для ориентированного графа есть утверждение, аналогичное теореме \ref{theorem:graphs:sum_of_degs} о рукопожатиях:

\begin{statement}
    $ \displaystyle \sum_{v \in V} d_+(v) = \sum_{v \in V} d_-(v) = |E| $
\end{statement}

Определим отдельно несколько частных случаев простого неориентированного графа.
\begin{definition}
    \begin{enumerate}[label=\arabic*)]
        \item[]
        \item
            \defemph{Ориентированный граф-путь} $ P_n $, $ n \geqslant 0 $~--- граф вида
            \[
                V(P_n) = \{ v_1, \ldots, v_n \}, \qquad
                E(P_n) = \left\{ (v_0, v_1), (v_1, v_2), \ldots, (v_{n-1}, v_n) \right\}
            \]
            Вершины $ v_1 $ и $ v_n $ называются \defemph{концами пути}, а $ n = |E| $~--- \defemph{длиной}.
            \textit{Ещё раз акцентируем внимание на том, что $ n \geqslant 0 $, а вершины нумеруются с нуля.}
        \item
            \defemph{Ориентированный граф-цикл} $ C_n $, $ n \geqslant 2 $~--- граф вида
            \[
                V(G) = \{ v_1, \ldots, v_n \}, \qquad
                E(G) = \left\{ (v_1, v_2), (v_2, v_3), \ldots, (v_{n-1}, v_n), (v_n, v_1) \right\}
            \]
            \textit{Ещё раз акцентируем внимание на том, что \uline{в отличие от неориентированного графа-цикла}, $ n \geqslant 2 $.}
    \end{enumerate}
\end{definition}

Без изменений вводится понятие подграфа ориентированного графа, а также индуцированного графа.

\begin{definition}
    Ориентированный граф, в котором нет подграфов-циклов, является \defemph{ациклическим}.
\end{definition}

\begin{definition}
    Вершина $ u $ в ориентированном графе $ G $ является \defemph{достижимой} из вершины $ v $ $ \defarr $ существует подграф-путь графа $ G $,
    концами которого являются вершины $ u $ и $ v $.
    Это обозначается как $ u \leadsto v $.
\end{definition}

Из свойств, описанных в замечании \ref{remark:graphs:connectivity_relation},
для ориентированного сохраняются только \emph{рефлексивность} и \emph{транзитивность};
симметричности в общем случае нет.
Однако понятие достижимости для ориентированного графа можно симметризовать:

\begin{definition}
    Вершины $ u $ и $ v $ являются \defemph{двусторонне достижимыми} $ \defarr $ $ (u \leadsto v) \wedge (v \leadsto u) $.
    Это обозначается как $ u \connected v $.
\end{definition}

Для отношения двусторонней достижимости можно ввести понятие \defemph{компонент сильной связности}, аналогичное \ref{definition:graphs:connectivity_component}:

\begin{definition}
    \label{definition:oriented_graphs:strong_connectivity_component}
    \defemph{Компонентой сильной связности} ориентированного графа $ G(V, E) $ будем называть подграф $ G $,
    индуцированный на некотором \underline{непустом} множестве $ U \subseteq V $,
    удовлетворяющем свойству $ \forall u, v \in U \; (u \connected v) $ и являющемся максимальным относительно него.
\end{definition}

\begin{definition}
    \defemph{Маршрутом} длины $ n \geqslant 0 $ в ориентированном графе $ G $ называется последовательность вершин $ v_0, v_1, \ldots, v_n $ такая,
    что $ \forall i \in \{ 0, \ldots, n-1 \} \;\, \left( (v_i, v_{i+1}) \in E(G) \right) $.
    \newline
    Число $ n $ называется \defemph{длиной маршрута}.
    \newline
    \textit{Отметим отдельно, что одна вершина тоже является маршрутом длины $ 0 $.}
    \\[0.25\baselineskip]
    Вершины $ v_0 $ и $ v_n $ называются \defemph{концами} маршрута;
    говорится, что маршрут \defemph{соединяет} $ v_0 $ и $ v_n $.
    В случае $ v_0 = v_n $ маршрут является \defemph{замкнутым}.
    \newline
    Говорят, что ребро $ (x, y) \in E(G) $ \defemph{лежит} на маршруте, если $ \exists i: \; (x, y) = (v_i, v_{i+1}) $.
\end{definition}

С учётом введённых определений утверждение \ref{statement:graphs:path_walk_equivalency} справедливо и для ориентированного графа.
Также можно сформулировать замечание, аналогичное \ref{remark:graphs:cc_partition}.

Введём теперь новое определение, которое можно обобщить и на случай неориентированного графа:
\begin{definition}
    Два графа $ G(V, E) $ и $ G'(V', E') $ называются \defemph{изоморфными} $ \defarr $ существует биекция $ f: V \to V' $ такая,
    что $ \forall (u, v) \in E \; \left[ (f(u), f(v)) \in E' \right] $.
\end{definition}
         % Графы.
\section{Функции}
\label{sec:functions}

Понятие функции уже встречалось нам ранее.
Например, оно фигурировало при определении \textit{булевой функции}, а также \textit{раскраски графа}.
Неформально, \defemph{функция}~--- это некоторое правило, в одностороннем порядке сопостовляющая каждому объекту из одного множества некоторый объект из другого множества.
Сопоставляемый объект не обязан быть уникальным, однако не может быть ситуации, когда одному элементу сопоставляются два и более.

Например, если некоторому действительному числу мы сопоставляем его квадрат, то это правило сопоставления~--- функция.
Если же мы попытаемся каждому положительному действительному числу $ y $ сопоставить решение уравнения $ x^2 = y $, то натолкнёмся на проблему неоднозначности:
функции не получается.

\subsection{Формальное определение}
\label{subsec:functions:definition}

На лекции вам давалось определение функции через понятие ориентированного двудольного графа.
В рамках семинара мы дадим другое, эквивалентное, но более часто используемое определение.
Для этого нам потребуется ввести некоторые вспомогательные обозначения.

\begin{definition}
    \defemph{Декартовым произведением} множеств $ A $ и $ B $ называется множество всех \defemph{упорядоченных пар}, где первый элемент взят из $ A $, а второй~--- из $ B $:
    \[
        A \times B = \{ (a, b) \mid a \in A, \; b \in B \}
    \]
    \textit{Напомним, что один из вариантов строгого определения упорядоченной пары давался в разделе \ref{subsec:sets:theory}.}
\end{definition}

Определение легко обобщается и на случай множественного произведения: вместо пар будут использоваться \defemph{кортежи}.
Также заметим, что $ A \times A $ обозначают как $ A^2 $.

\begin{definition}
    \label{definition:functions:function}
    \defemph{Функцией (частичной функцией)}, принимающей аргументы из множества $ X $ и значения во множестве $ Y $,
    называется подмножество $ f \subseteq X \times Y $ такое, что ни у каких двух пар из $ f $ первый элемент не совпадает.
    \\[0.25\baselineskip]
    Если $ x \in X $ и $ \exists y \in Y: (x, y) \in f $, то говорят, что функция $ f $ \defemph{определена} в точке $ x $, и $ f(x) = y $.
\end{definition}

\begin{example}
    Рассмотрим пару примеров функций и не функций.
    \begin{enumerate}
        \item
            $ X = \{ 0, 1 \} $, $ Y = \{ x, y \} $.
            Тогда $ X \times Y = \left\{ (0, x), (0, y), (1, x), (1, y) \right\} $.
            \newline
            $ f = \{ (0, y), (1, y) \} $ является функцией, в то время как
            $ g = \{ (0, x), (0, y), (1, x) \} $~--- нет, хотя и $ g \subseteq X \times Y $.
        \item
            $ X = \R_{+} $, $ Y = \R $.
            \newline
            $ f = \{ (x, x^2) \mid x \in X \} $ является функцией, в то время как
            $ g = \{ (x, y) \mid x \in X, \; y \in B, \; y^2 = x \} $~--- нет, хотя и $ g \subseteq X \times Y $.
    \end{enumerate}
\end{example}

Понятно, что каждый раз излишне формально определять функцию как множество пар не стоит.
Достаточно записать непосредственно правило, по которому одному элементу ставится в соответствие другой:
$ f: x \mapsto f(x) $.
Например, $ \exp: x \mapsto e^x $, или просто $ \exp(x) = e^x $.

Заметим, что функция не должна быть определена в каждой точке множества аргументов.
Например, функцию $ f(x) = 1 / |x| $ можно рассматривать в контексте $ X = Y = \R $, хотя она и не определена в точке $ x = 0 $.
Эта же самая функция также принимает не все возможные значения из $ Y $.
В этой связи полезно ввести следующие определения:

\begin{definition}
    \defemph{Областью определения} функции $ f \subseteq X \times Y $ называется множество
    \[
        \dom(f) = \{ x \in X \mid \exists y \in Y: f(x) = y \}
    \]
    \defemph{Областью значений} функции $ f \subseteq X \times Y $ называется множество
    \[
        \range(f) = \{ y \in Y \mid \exists x \in X: f(x) = y \}
    \]
\end{definition}

\begin{definition}
    Если $ f(x) = y $, то $ y $ называется \defemph{образом} элемента $ x $, а $ x $~--- \defemph{прообразом} элемента $ y $
\end{definition}

Также нас будет интересовать то, как функция отображает целое множество, а не только один элемент.
Для этого вводятся следующие определения:

\begin{definition}
    \defemph{Образом} некоторого подмножества $ A \subseteq X $ называется множество
    \[
        f(A) = \{ y \in Y \mid \exists x \in A: f(x) = y \}
    \]
    \defemph{Полным прообразом} некоторого подмножества $ B \subseteq Y $ называется множество
    \[
        f^{-1}(B) = %\{ x \in X \mid \exists y \in B: f(x) = y \} =
        \{ x \in X \mid f(x) \in B \}
    \]
    \defemph{Полным прообразом} некоторого элемента $ y \in Y $ называется полный прообраз множества $ \{ y \} $.
\end{definition}

\begin{remark}
    $ f(X) = f(\dom(f)) = \range(f) $, $ f^{-1}(Y) = f^{-1}(\range(f)) = \dom(f) $.
\end{remark}


\begin{Exercise}[counter=SecExercise, label={exercise:functions:sets_examples}]
    \noindent
    Частичная функция $ f $ из множества $ \{1, 2, \ldots, 8 \} $ в множество $ \{a, b, \ldots, e \} $ определена следующим образом:
    \[
        f: 1 \mapsto a, \quad 2 \mapsto a, \quad 3 \mapsto c, \quad 4 \mapsto d, \quad 5 \mapsto c, \quad 7 \mapsto d.
    \]
    Найдите
    \begin{enumerate}[label=\textbf{\alph*)}]
        \item $ \dom(f) $;
        \inlineitem $ \range(f) $;
        \inlineitem $ f(\{1, 2, 3\}) $;
        \inlineitem $ f^{-1}(c) $;
        \inlineitem $ f(\{1, 2, 3, 5, 6\}) $;
        \item $ f^{-1}(\{a, b, c\}) $.
    \end{enumerate}
\end{Exercise}

\begin{Answer}
    \noindent
    \begin{enumerate}[label=\textbf{\alph*)}]
        \item $ \dom(f) = \{1, 2, 3, 4, 5, 7\} $;
        \inlineitem $ \range(f) = \{a, c, d\} $;
        \inlineitem $ f(\{1, 2, 3\}) = \{a, c\} $;
        \item $ f^{-1}(c) = \{3, 5\} $;
        \inlineitem $ f(\{1, 2, 3, 5, 6\}) = \{a, c\} $;
        \inlineitem $ f^{-1}(\{a, b, c\}) = \{1, 2, 3, 5\} $.
    \end{enumerate}
\end{Answer}


\begin{Exercise}[counter=SecExercise, label={exercise:functions:max_prime_devisor}]
    \noindent
    Частичная функция $ g $ из множества положительных целых чисел в множество положительных целых чисел
    сопоставляет числу $ x $ наибольший простой делитель $ x $.
    \begin{enumerate}[label=\textbf{\alph*)}]
        \item Какова область определения $ g $?
        \item Верно ли, что если $ X $~--- конечное, то и $ g^{-1}(X) $ конечное?
        \item Найдите $ g^{-1}(3) $.
    \end{enumerate}
\end{Exercise}

\begin{Answer}
    \noindent
    \begin{enumerate}[label=\textbf{\alph*)}]
        \item
            Единица~--- единственное положительное число, не имеющее простых делителей.
            Поэтому $ \dom(f) = \N $.
        \item
            Нет, неверно.
            Пример~--- $ f^{-1}(\{ 2 \}) = \{ 2^n \mid n \in \N \} $.
            Действительно, у всех положительных целых степеней двойки, причём только у них, наибольший простой делитель~--- $ 2 $.
        \item
            Аналогично предыдущему пункту замечаем, что $ f^{-1}(\{ 3 \}) = \{ 2^n 3^m \mid n \in \N_0, \; m \in \N \} $.
            Действительно, раз наибольший простой делитель~--- $ 3 $,
            никаких простых делителей, кроме $ 2 $ и $ 3 $ быть не может,
            причём тройка обязана присутствовать в качестве делителя.
    \end{enumerate}
\end{Answer}


\begin{Exercise}[counter=SecExercise, label={exercise:functions:set_operations}]
    \noindent
    Пусть $ f $~--- частичная функция из множества $ A $ в множество $ B $,
    $ X, Y \subseteq A $, $ U, V \subseteq B $.
    Верны ли для любых множеств $ f, A, B, X, Y, U, V $ следующие утверждения:
    \begin{enumerate}[label=\textbf{\alph*)}]
        \item $ f (X \cup Y) = f (X) \cup f (Y) $;
        \item из равенства $ f(X) = f(Y) $ следует $ X \cap Y \neq \varnothing $;
        \item $ f^{-1}(U \cap V) = f^{-1}(U) \cap f^{-1}(V) $;
        \item из равенства $ f^{-1}(U) = f^{-1}(V) $ следует $ U = V $.
    \end{enumerate}
\end{Exercise}

\begin{Answer}
    \noindent
    \begin{enumerate}[label=\textbf{\alph*)}]
        \item
            Верно:
            \begin{multline*}
                f(X \cup Y) = \{ z \mid \exists x \in X \cup Y: f(x) = z \} = \\
                = \{z \mid \left(\exists x \in X: f(x) = z \right) \vee \left( \exists y \in Y: f(y) = z \right) \} = f(X) \cup f(Y)
            \end{multline*}
        \item
            \label{item:functions:constant_mapping}
            Неверно: рассмотрим $ f: \R \ni x \mapsto 0 $ и $ X = Y = \R $.
            Тогда $ f(X) = f(Y) = \{ 0 \} $, но $ X \cap Y \neq \varnothing $.
        \item
            Верно:
            \begin{multline*}
                f^{-1}(U \cap V) = \{ x \mid f(x) \in U \cap V \} = \\
                = \{x \mid (f(x) \in U) \wedge (f(x) \in V) \} = f^{-1}(U) \cap f^{-1}(V)
            \end{multline*}
        \item
            Неверно: рассмотрим тот же пример функции, что и в пункте \ref{item:functions:constant_mapping}.
            В качестве $ U $ и $ V $ возьмём $ \{0\} $ и $ \{0, 1\} $ соответственно.
            Тогда $ f^{-1}(U) = f^{-1}(V) = \R $, но $ U \neq V $.
    \end{enumerate}
\end{Answer}


\begin{Exercise}[counter=SecExercise, label={exercise:functions:map_unmap}]
    \noindent
    Частичная функция $ f $ определена на множестве $ X $ и принимает
    значения в множестве $ Y $, при этом $ B \subseteq Y $.
    Какой знак сравнения можно поставить вместо <<?>>, чтобы утверждение <<$ f (f^{-1}(B)) \; ? \; B $>> стало верным?
\end{Exercise}

\begin{Answer}
    \noindent
    Покажем, что можно поставить знак <<$ = $>>
    (и, как следствие, <<$ \subset $>> и <<$ \supset $>>).
    \[
        y \in f(f^{-1}(B)) \quad \Longleftrightarrow \quad \exists x \in f^{-1}(B): f(x) = y
    \]
    Но $ x \in f^{-1}(B) \; \Longleftrightarrow \; f(x) \in B $.
    Тогда
    \[
        \exists x \in f^{-1}(B): f(x) = y \quad \Longleftrightarrow \quad y \in B
    \]
    Таким образом, $ y \in B \; \Longleftrightarrow \; y \in f(f^{-1}(B)) $,
    что и означает $ B = f(f^{-1}(B)) $.
\end{Answer}

\subsection{Отображения}
\label{subsec:functions:mappings}

Отдельно рассматривается случай, когда функция определена в любой точке из множества аргументов.
В этом случае говорится, что функция является \defemph{отображением}, или \defemph{всюду определённой функцией}.

\begin{definition}
    Функция $ f \subseteq X \times Y $ называется \defemph{отображением} в случае $ \dom(f) = X $.
    При этом пишут $ f: X \to Y $.
\end{definition}

\begin{remark}
    Любая функция становится отображением при сужении множества аргументов до области определения: $ f: \dom(f) \to Y $.
\end{remark}

\begin{definition}
    В случае $ f: X \to X $ отображение $ f $ называют \defemph{преобразованием}
\end{definition}

Среди отображений выделяют следующие три важных вида:

\begin{definition}
    Отображение $ f: X \to Y $ называется \defemph{инъекцией} $ \defarr $ $ \forall x_1, x_2 \in X \;\, (x_1 \neq x_2) \rightarrow (f(x_1) \neq f(x_2)) $.
    То есть, из неравенства аргументов следует неравенство значений отображения.
\end{definition}

\begin{definition}
    Отображение $ f: X \to Y $ называется \defemph{сюръекцией} $ \defarr $ $ \range(f) = Y $.
    То есть, у любого элемента из $ Y $ существует прообраз.
\end{definition}

\begin{definition}
    Отображение $ f: X \to Y $ называется \defemph{биекцией} в случае, когда оно и инъекция, и сюръекция.
\end{definition}

\begin{remark}
    Биекция является правилом, взаимнооднозначно сопостовляющим каждому элементу из $ X $ некоторый элемент из $ Y $ и наоборот.
\end{remark}

\begin{corollary}
    Каждое биективное отображение \defemph{обратимо}, то есть если $ f: X \to Y $~--- биекция, то
    \[
        g = f^{-1} \defeq \{ (y, x) \in Y \times X \mid (x, y) \in f \}
    \]
    является отображением, причём биекцией.
\end{corollary}

\begin{example}
    \label{example:functions:mappings}
    \begin{enumerate}
        \item[]
        \item
            Пусть $ f: \R^2 \to \R^3 $~--- отображение, ставящее в соответствие паре чисел $ (x_1, x_2) $ коэффициенты $ (a, b, c) $
            квадратного уравнения, корнями которого являются $ (x_1, x_2) $, причём $ a = 1 $.

            Это \textbf{инъекция}, так как $ (a, b, c) = (1, \, - x_1 - x_2, \, x_1 x_2) $, и равенство всех значений невозможно при $ (x_1', x_2') \neq (x_1'', x_2'') $.
            С другой стороны, это \textbf{не сюръекция}, так как у троек с $ a \neq 1 $ нет прообразов.
        \item
            Пусть $ X $~--- множество многочленов степени не выше $ m+1 $, а $ Y $~--- многочленов степени не выше $ m $.

            Рассмотрим операцию взятия производной $ \frac{d}{dx}: X \to Y $, то есть $ \frac{d}{dx}: p(x) \mapsto p'(x) $.

            Это \textbf{сюръекция}, но \textbf{не инъекция}: для каждого многочлена из $ Y $ есть прообраз в $ X $~--- его интеграл, но при этом любой константный многочлен отображается в ноль.
        \item
            Пусть $ f: [0;1] \to [0;1] $ непрерывна и монотонно возрастает.
            Тогда по теореме об обратной функции это \textbf{биекция}.
    \end{enumerate}
\end{example}

\begin{Exercise}[counter=SecExercise]
    \noindent
    Пусть $ f: X \to X $~--- сюръективное преобразование.
    Верно ли, что $ f $ инъективно?
\end{Exercise}

\begin{Answer}
    \noindent
    Нет.
    Приведём контрпример: пусть $ X $~--- множество всех многочленов, а $ f = \frac{d}{dx} $.
    Аналогично рассуждениям в примере \ref{example:functions:mappings} получаем, что $ f $~--- сюръекция, но не инъекция.

    Заметим однако, что если $ X $ конечно, то утверждение в условии верно.
    Предлагаю вам самим это проверить.
\end{Answer}

\begin{Exercise}[counter=SecExercise, label={exercise:functions:infinite_surjection}]
    \noindent
    Приведите пример сюръективного преобразования $ f: \N \to \N $ такого, что полный прообраз каждого элемента $ \N $ бесконечен.
\end{Exercise}

\begin{Answer}
    \noindent
    Выпишем подряд все элементы $ \N $.
    Вычеркнем все числа, стоящие на нечётных позициях.
    Для оставшихся чисел повторим операцию, и так далее.

    Пусть теперь $ f $ сопоставляет числу номер шага, на котором его вычеркнули.
    Это действительно отображение, так как после каждого шага минимальное число среди невычеркнутых растёт,
    из чего следует, что любое число будет вычеркнуто на каком-то шаге.
    Это действительно сюръекция, так как на любом шаге вычёркивается хотя бы одно число.
    И, наконец, полным прообразом каждого элемента $ \N $ будет бесконечное множество,
    так как на каждом шаге вычёркивается бесконечное множество чисел.
\end{Answer}


\subsection{Функции и мощность множества}
\label{subsec:functions:cardinality}

Изученные нами понятия также играют важную роль и в теории множеств.
Помимо состава множеств и взаимоотношений между ними нас часто будет интересовать то, насколько некоторое множество <<велико>>.
Легко определить <<размер>> множества в случае, когда оно конечно: это просто число элементов.
Но что делать, если множество содержит бесконечно много элементов?
Хочется сказать, что если два множества бесконечны, то они <<равновелики>>.
Однако это противоречит интуитивным представлениям о том, что, например, $ 2^A $ содержит элементов больше, чем $ A $.

Оказывается, эти интуитивные представления можно формализовать, если по-другому взглянуть на размер конечных множеств.
Если множества $ A $ и $ B $ конечны, то можно сказать, что они равновелики, если в них одинаковое число элементов.
По сути, это эквивалентно тому, что можно задать взаимнооднозначное правило соответствия~--- \textit{биекцию}~--- между каждым элементом $ A $ и $ B $.

\begin{statement}
    \label{statement:functions:finite_bijection}
    Если $ A $ и $ B $~--- конечные множества, то они содержат одинаковое число элементов тогда и только тогда,
    когда существует биекция из одного множества в другое.
\end{statement}

Формальное доказательство утверждения становится очевидным, если любым способом пронумеровать элементы множеств.

Данное утверждение позволяет по-иному формально определить размер, или \defemph{мощность} множества, и обобщить это определение на все множества вообще.
\begin{definition}
    Множества $ A $ и $ B $ называются \defemph{равномощными} в том и только том случае,
    если существует биекция между элементами множеств.
\end{definition}
Стоит обратить внимание, что требуемая биекция не обязана быть единственной.

\begin{definition}
    Множество $ A $ называется \defemph{счётным} $ \defarr $ $ A $ равномощно $ \N_0 $.
\end{definition}

\begin{example}
    \begin{enumerate}
        \item[]
        \item
            Множества $ \{1, 2\} $ и $ \{a, x\} $ равномощны, причём можно построить две биекции между ними:
            \[
                1 \sim a, \; 2 \sim x \qquad \text{или} \qquad 1 \sim x, \; 2 \sim a
            \]
        \item
            Множества $ \N_0 $ и $ E = \{ x \in \N_0 \mid \exists k \, (x = 2k) \} $ равномощны, биекция задаётся, например, правилом $ E \ni x = 2 \cdot k $, где $ k $~--- любой элемент $ \N_0 $.
        \item
            Множества $ \Q $ и $ \N_0 $ равномощны.
            Идея доказательства: $ \Q $ можно задать бесконечной таблицей, номер строки и столбца в которой~--- числитель и знаменатель.
            А все ячейки таблицы можно пронумеровать, идя <<змейкой>> (при этом сократимые дроби не нумеруются).
    \end{enumerate}
\end{example}

Может создасться впечатление, что все бесконечные множества счётны.
Однако это неверно.

\begin{theorem}[Кантора]
    Для любого $ A $ множества $ A $ и $ 2^A $ неравномощны.
\end{theorem}

\begin{statement}
    Множество $ \R $ несчётно.
\end{statement}

Доказательство данного утверждения обычно приводят в курсе математического анализа.

\begin{Exercise}[counter=SecExercise]
    \noindent
    Счётно ли множество всех корректных программ, написанных на языке C++?
\end{Exercise}

\begin{Answer}
    \noindent
    Да, оно счётно.
    Для доказательства этого заметим, что можно построить следующую таблицу:
    номер строки равен длине программы в символах, а номер столбца~--- лексикографическому порядковому номеру программы среди всех программ заданной длины.
    Обходя таблицу <<змейкой>>, получаем взаимнооднозначную нумерацию всех программ.
\end{Answer}

На текущий момент мы формально определили лишь случай равенства мощностей двух бесконечных множеств.
Можно пойти дальше и определить оставшиеся операции сравнения.
Нетрудно проверить, что утверждение \ref{statement:functions:finite_bijection} можно обобщить в виде следующей леммы:

\begin{lemma}
    \label{lemma:functions:finite_cardinality_compare}
    Пусть $ A $ и $ B $~--- конечные множества.
    Тогда
    \begin{enumerate}
        \item $ |A| = |B| $ $ \Longleftrightarrow $ существует биекция между $ A $ и $ B $.
        \item $ |A| \leqslant |B| $ $ \Longleftrightarrow $ существует инъекция из $ A $ в $ B $.
        \item $ |A| \geqslant |B| $ $ \Longleftrightarrow $ существует сюръекция из $ A $ в $ B $.
    \end{enumerate}
\end{lemma}

Обобщим эту лемму на случай произвольных множеств, определив соответствующим образом операции сравнения.

\begin{definition}
    \label{definition:functions:cardinaluty_compare}
    В случае существования инъекции $ f: A \to B $ говорят, что \defemph{$ B $ не менее мощно, чем $ A $}.
    Это обозначается как $ |A| \leqslant |B| $.
    При этом по определению полагают $ (|A| < |B|) \defarr (|A| \leqslant |B|) \wedge (|A| \neq |B|) $.
\end{definition}

%\begin{statement}
%    Данное определение операции сравнения мощностей эквивалентно сравнению числа элементов в случае конечных множеств.
%\end{statement}

\begin{statement}
    \label{statement:functions:cardinality_compare_equivalence}
    Множество $ B $ не менее мощно, чем множество $ A $, тогда и только тогда, когда существует сюръекция $ g: B \to A $.
\end{statement}

\begin{proof}
    \begin{enumerate}
        \item[]
        \item[$\Rightarrow$]
            По определению, существует инъекция $ f: A \to B $.
            Заметим тогда, что $ g' = \{ (y, x) \in B \times A \mid (x, y) \in f \} $ является частичной функцией.
            Действительно, раз $ f $~--- инъекция, ни у каких двух (разных) пар из $ f $ не совпадают вторые элементы.
            Значит, ни у каких двух (разных) пар из $ g' $ не совпадают первые элементы.

            Заметим также, что $ g': \dom(g') \to A $~--- сюръекция.
            Действительно, так как $ f $~--- отображение, любой элемент $ A $ является первым элементом хотя бы какой-то пары из $ f $.
            Но тогда он же является и вторым элементом некоторой пары из $ g' $.

            Тогда построим $ g $ как произвольное доопределение $ g' $ на $ B $.
            Таким образом, получена сюръекция $ g: B \to A $.
        \item[$\Leftarrow$]
            Аналогично.
    \end{enumerate}
\end{proof}

Из данного утверждения следует, что лемма \ref{lemma:functions:finite_cardinality_compare}
обобщается и на случай бесконечных множеств при использовании определения \ref{definition:functions:cardinaluty_compare}.
      % Функции.
\section{Комбинаторика}
\label{sec:combinatorics}

В предыдущем разделе мы подробно изучили инструментарий для сравнения мощностей множеств.
Кажется, что полученные результаты полезны, скорее, при работе с бесконечными множествами,
так как в ином случае достаточно сравнивать мощности как обычные числа.
Однако это неверно: изученные методы
%сравнения мощностей посредством построения определённых отображений
оказываются хорошим подспорьем и в задачах,
например, определения точного числа элементов в некотором множестве, или хотя бы в деле построения верхних и нижних оценок на это число.
Данными задачами занимается \defemph{комбинаторика}.

\begin{example}
    Чего больше: разбиений числа $ n $ на $ k $ слагаемых, или разбиений $ N $ на слагаемые, не превосходящие $ k $?

    %Легче всего её задать следующим правилом:
    %в некотором разбиении числа $ N $ на $ k $ слагаемых будем вычитать из каждого слагаемого по единице до тех пор, пока не появится первое нулевое слагаемое;
    %уберём его и продолжим процедуру, и так до полного опустошения суммы.

    %Тогда соответствующее разбиение $ N $ на слагаемые, не превосходящие $ k $, получается путём сложения чисел $ k $ столько раз, сколько было сделано шагов до появления первого нуля,
    %чисел $ k-1 $ столько раз, сколько было сделано последующих шагов до второго нуля, и так далее.
    %Постройте обратное отображение и проверьте, что, перед нами действительно биекция.

    Оказывается, в обоих случаях разбиений одно и то же число, ведь между данными множествами можно построить биекцию.
    Легче всего построить биекцию путём рассмотрения \textit{диаграмм Юнга} для разбиений.
    Пример таких диаграмм для двух разбиений числа $ N = 11 $: $ N = 5 + 3 + 2 + 2 $ и $ N = 4 + 4 + 2 + 1 + 1 $.
    \begin{center}
        \raisebox{-0.5\height}{
            \ydiagram{5,3,2,2}
        }
        \hspace{2\baselineskip}
        \raisebox{-0.5\height}{
            \ydiagram{4,4,2,1,1}
        }
    \end{center}
    Видно, что одно разбиение получается из другого транспонированием диаграммы.
    Также видно, что в первом случае имеется разбиение на $ k $ слагаемых, а во втором~--- на слагаемые, не превосходящие $ k $.
    Детали биекции предлагаю вам додумать самостоятельно.
\end{example}

Таким образом, получаем \emph{важный факт}: если требуется определить число элементов в некотором множестве,
можно попробовать сначала доказать, что элементов в нём столько же, сколько и в некотором другом множестве (возможно, с более понятным составом),
а потом уже пересчитать элементы второго множества.
Данное правило очевидным образом обобщается и на случаи, когда требуется оценить мощность множества сверху или снизу.


\subsection{Базовые комбинаторные задачи}
\label{subsec:combinatorics:basics}

Составим джентльменский набор базовых задач комбинаторики, к которым впоследствии можно будет сводить другие задачи посредством построения биекции.


\subsubsection{Правило суммы}

Начнём с задачи подсчёта числа элементов в множестве вида $ A = A_1 \cup A_2 \cup \ldots \cup A_n $.
Понятно, что если $ \forall i \neq j \;\, A_i \cap A_j = \varnothing $, то $ |A| = |A_1| + |A_1| + \ldots + |A_n| $.
Для строгого доказательства этого факта нам потребуется следующее утверждение:

\begin{statement}
    \label{statement:combinatorics:sum_indicators}
    Если $ A $ конечно, то $ \displaystyle |A| = \sum_{x \in U} \I_A(x) $.
\end{statement}

\begin{proof}
    $ |A| $ равно числу элементов, которые лежат в $ A $.
    Но заметим, что каждый такой элемент добавляет единицу в сумму $ \displaystyle \sum_{x \in U} \I_A(x) $,
    причём никакие другие элементы на сумму не влияют, так как $ \I_A $ принимает на них значение $ 0 $.
\end{proof}

\begin{corollary}
    Если $ A = A_1 \sqcup A_2 \sqcup \ldots \sqcup A_n $, то $ |A| = |A_1| + |A_2| + \ldots + |A_n| $.
\end{corollary}

\begin{proof}
    Раз $ \forall i \neq j \;\, A_i \cap A_j = \varnothing $, то $ \I_A(x) = \I_{A_1}(x) + \I_{A_2}(x) + \ldots + \I_{A_n}(x) $.
    Отсюда по утверждению \ref{statement:combinatorics:sum_indicators} получаем требуемое.
\end{proof}

Но что делать, если множества пересекаются?
Для случая двух множеств можно заметить, что $ \I_{A \cup B}(x) = \I_A(x) + \I_B(x) - \I_{A \cap B}(x) $.
Но тогда по утверждению \ref{statement:combinatorics:sum_indicators} имеем $ |A \cup B| = |A| + |B| - |A \cap B| $.
Данная формула называется \defemph{формулой включений-исключений}.
Она обобщается и на случай с $ n $ множествами.

\begin{lemma}
    Если $ \displaystyle A = \bigcup_{i=1}^n A_i $, то
    \[
        |A| = \sum_{k=1}^n (-1)^{k+1} \left( \sum_{S \in \mathcal{C}_k} \left| \bigcap_{A \in S} A \right| \right), \qquad
        \mathcal{C}_k = \begin{pmatrix} \{ A_1, A_2, \ldots, A_n \} \\ k \end{pmatrix}
    \]
\end{lemma}

\begin{proof}
    Можно доказать как по индукции, так и просто аккуратно записав $ \I_A(x) $ через обычные математические операции
    и воспользовавшись утверждением \ref{statement:combinatorics:sum_indicators}.
\end{proof}

\begin{Exercise}[counter=SecExercise]
    \noindent
    В группе $ 40 $ туристов.
    Из них $ 20 $ человек говорят по-английски, $ 15 $~--- по-французски, $ 11 $~--- по-испански.
    Английский и французский знают семь человек, английский и испанский~--- пятеро, французский и испанский~--- трое.
    Два туриста говорят на всех трёх языках.
    Сколько человек группы не знают ни одного из этих языков?
\end{Exercise}

\begin{Answer}
    \noindent
    По формуле включений-исключений имеем
    \[
        N = 40 - (20 + 15 + 11 - 7 - 5 - 3 + 2) = 40 - 33 = 7
    \]
\end{Answer}

\subsubsection{Правило произведения}

Рассмотрим задачу подсчёта числа возможных путей из вершины так называемого \defemph{дерева последовательного выбора} в любой его лист.

\begin{definition}
    \defemph{Деревом последовательного выбора} называется дерево, у которого можно выделить вершину (\defemph{корень}) так,
    чтобы все остальные вершины, расположенные на одном и том же расстоянии от выделенной, имели одинаковую степень.
\end{definition}

Пример дерева последовательного выбора можно видеть на рис. \ref{fig:combinatorics:decision_tree}.
На каждой вершине отмечена её степень за вычетом родительского ребра.
Слева от каждого уровня дерева выписано число путей с началом из корня и с концом на данном уровне.

\begin{statement}
    Расстояние от корня дерева последовательного выбора до любого из листьев одинаково.
\end{statement}

\begin{figure}[ht!]
    \center
    \begin{tikzpicture}[
        parent anchor=south,child anchor=north,
        level/.style={sibling distance=40mm/#1},
        >=latex,
        font=\sffamily,
        edge from parent/.style={draw, thick},
        no edge from this parent/.style={
            every child/.append style={
            edge from parent/.style={draw=none}}},
        level 3/.style={yshift=5cm},
        level 4/.style={level distance=5mm}
    ]
        \node (z){$ k_1 $}
        child
        {
            node (a) {$ k_2 $}
            child
            {
                node  (b) {$ k_3 $}
                child
                {
                    node (b1) {$\vdots$}[no edge from this parent]
                    child
                    {
                        node (b11) {$ 0 $}
                    }
                }
                child
                {
                    node (b2) {$\vdots$}[no edge from this parent]
                    child
                    {
                        node (b12) {$ 0 $}
                    }
                }
            }
            child
            {
                node (g) {$ k_3 $}
                child
                {
                    node (g1) {$\vdots$}[no edge from this parent]
                    child
                    {
                        node (g11) {$ 0 $}
                    }
                }
                child
                {
                    node (g2) {$\vdots$}[no edge from this parent]
                    child
                    {
                        node (g12) {$ 0 $}
                    }
                }
            }
        }
        child
        {
            node (d) {$ k_2 $}
            child
            {
                node  (e) {$ k_3 $}
                child
                {
                    node (e1) {$\vdots$}[no edge from this parent]
                    child
                    {
                        node (e11) {$ 0 $}
                    }
                }
                child
                {
                    node (e2) {$\vdots$}[no edge from this parent]
                    child
                    {
                        node (e12) {$ 0 $}
                    }
                }
            }
            child
            {
                node (f) {$ k_3 $}
                child
                {
                    node (f1) {$\vdots$}[no edge from this parent]
                    child
                    {
                        node (f11) {$ 0 $}
                    }
                }
                child
                {
                    node (f2) {$\vdots$}[no edge from this parent]
                    child
                    {
                        node (f12) {$ 0 $}
                    }
                }
            }
        };

        \node[left=6 of z]  (ln1) {$ 1 $}[no edge from this parent]
        child
        {
            node (ln2) {$ k_1 $}[no edge from this parent]
            child
            {
                node (ln3) {$ k_1 \cdot k_2 $}[no edge from this parent]
                child
                {
                    node (ln4) {}[no edge from this parent]
                    child
                    {
                        node (ln5) {$ k_1 \cdot k_2 \cdot \ldots \cdot k_h $}
                    }
                }
            }
        };

        %\path (b12.north east) -- (g11.north west) node [midway] {$\cdots$};
        %\path (e12.north east) -- (f11.north west) node [midway] {$\cdots$};

        \coordinate (cd1) at ($(f12)+(1,0)$);
        \coordinate (nb1) at ($(g12)!.5!(e11)$);

        \draw[thick,<->,]
            (cd1) -- (cd1|-z.east) node [near start, fill=white] {$ h $};

        \draw[dashed,thick,->]
            ($(z.west)+(-1em,0)$) -- (ln1);
        \draw[dashed,thick,->]
            ($(a.west)+(-1em,0)$) -- (ln2.east);
        \draw[dashed,thick,->]
            ($(b.west)+(-1em,0)$) -- (ln3);
        \draw[dashed,thick,->]
            ($(b11.west)+(-1em,0)$) -- (ln5);

        \draw[thick,decorate,decoration={brace,amplitude=10pt,mirror},->,-{latex[flex=1pt]}] (b11.south west) -- (f12.south east);
    \end{tikzpicture}

    \label{fig:combinatorics:decision_tree}
    \caption{дерево последовательного выбора (схематично)}
\end{figure}

\begin{definition}
    Расстояние от корня дерева последовательного выбора до любого из листьев называется \defemph{высотой дерева} (или \defemph{числом выборов}) и обозначается $ h $.
    \newline
    Степень каждой вершины (на расстоянии $ m - 1 $ от корня) за вычетом родительского ребра называется \defemph{мощностью выбора (на шаге $ m $)} и обозначается $ k_m $.
    \newline
    Путь из корня в лист называется \defemph{решающим путём}.
\end{definition}

\begin{statement}
    Пусть дерево последовательного выбора характеризуется высотой $ h $ и мощностями выборов $ k_1, k_2, \ldots, k_h $.
    Тогда число путей из корня в любой лист равно $ k_1 \cdot k_2 \cdot \ldots \cdot k_h $.
\end{statement}

\begin{proof}
    По индукции, проведите сами.
\end{proof}

\begin{Exercise}[counter=SecExercise, label={exercise:combinatorics:students}]
    \noindent
    На одном этаже семёрки живёт $ 100 $ человек.
    Среди них требуется выбрать двух ответственных за южную и северную кухни, одного ответственного за умывальники и санузел, а также его заместителя.
    Сколькими способами это можно сделать?
\end{Exercise}

\begin{Answer}
    \noindent
    Можно построить биекцию из множества способов выбрать ответственных в множество путей от корня к листьям в дереве последовательного выбора высоты $ h = 4 $
    и с мощностями выборов $ k_1 = 100 $, $ k_2 = 100 - 1 $, $ k_3 = 100 - 2 $ и $ k_4 = 100 - 3 $.
    Действительно, движение от корня к листьям пусть будет соответствовать последовательному выбору ответственных.
    Тогда переход по первому ребру соответствует выбору ответственного за южную кухную из $ 100 $ студентов, переход по второму~---
    выбору ответственного за северную кухню из оставшихся $ 99 $ студентов и так далее.
    Тогда всего способов~--- $ 100 \cdot 99 \cdot 98 \cdot 97 $.
\end{Answer}

Мы рассмотрели очень частный случай, когда мощность очередного выбора на единицу меньше мощности предыдущего.
Это не всегда так, и ниже будет рассмотрены две задачи другого типа.
Однако и такой специальный случай встречается настолько часто, что для обозначения соответствующего ответа ввели специальное число:
\[
    A_n^k = n \cdot (n-1) \cdot \ldots \cdot (n - k) = \frac{n!}{(n-k)!}
\]
Это число называется \defemph{числом расстановок}.
Связь названия и класса задач довольно очевидна: действительно, в задаче \ref{exercise:combinatorics:students} мы <<расставили>> $ n = 100 $ студентов по $ k = 4 $ должностям.

\begin{Exercise}[counter=SecExercise]
    \noindent
    Сколькими способами можно выбрать два числа разной чётности из множеств $ \{1, \ldots, 4\} $ и $ \{ 11, \ldots, 16 \} $?
\end{Exercise}

\begin{Answer}
    \noindent
    На первом шаге можно четырьмя способами выбрать число из первого диапазона,
    на втором шаге мы будем выбирать из $ 6 / 2 = 3 $ элементов (так как чётность фиксирована).
    В итоге имеем $ 4 \cdot 3 = 12 $ способов.
\end{Answer}

\begin{Exercise}[counter=SecExercise]
    \noindent
    Сколькими способами можно выбрать два числа из диапазона $ \{ 1, \ldots, 9 \} $, дающие разный остаток при делении на три?
\end{Exercise}

По аналогии с прошлой задачей в голову сразу приходит ответ $ 9 \cdot (9 \cdot 2 / 3) = 54 $.
Однако если честно пересчитать все варианты, получится число в два раза меньшее.
В чём же проблема?

Дело в том, что в предыдущей задаче на каждом шаге числа выбирались из разных множеств,
что позволяло однозначно сопоставить каждому решающему пути число из первого множества и число из второго.
В случае текущей задачи уже двум решающим путям будет соответствовать одна и та же пара чисел, просто выбранная в разном порядке (например, $ \{1, 3\} $ и $ \{3, 1\} $).
Понятно, что учёт возможной перемены местами выбранных чисел как раз и уменьшает ответ в два раза, но как это отражается в построении биекции?

Если оставаться в рамках модели деревьев последовательного выбора, то данная проблема обычно решается введением дополнительных ограничений,
позволяющих зафиксировать порядок получения результатов выбора.
Например, в случае нашей задачи можно потребовать, чтобы второй выбор совершался не среди двух оставшихся классов,
а только среди того класса, что соответствует следующему остатку по модулю три.
То есть, например, если мы выбрали число с остатком $ 0 $, то второе число обязано иметь остаток $ 1 $,
если выбрали число с остатком $ 1 $, то второе~--- с остатком $ 2 $ и так далее.
То, что построена биекция между множеством из задачи и решающими путями в дереве с мощностями выборов $ 9 $ и $ 3 $, проверьте сами.

\subsubsection{Подсчёт подмножеств}

Из возникшей проблемы понятно, что одним правилом произведения сыт не будешь.
Далеко не всегда ясно, какое ограничение надо ввести, чтобы получить биекцию.
Пойдём дальше и рассмотрим другую базовую задачу, к которой уже будет легко свести проблемное упражнение из предыдущего пункта.

Пусть $ A $~--- конечное множество мощности $ |A| = n $.
Тогда чему равно
\[
    \left| \binom{A}{k} \right|,
\]
где $ k \in \{1, \ldots, n\} $?
Ответ уже давался в замечании \ref{remark:graphs:subsets_cardinality}, настало время его строго обосновать.

\begin{proof}[Доказательство замечания \ref{remark:graphs:subsets_cardinality}]
    Задача подсчёта числа расстановок $ n $ объектов по $ k $ позициям уже решена:
    их $ A_n^k $.
    Осталось понять, чем это отличается от числа подмножеств мощности $ k $.

    Заметим, что каждой последовательности элементов $ A $ длины $ k $ соответствует подмножество $ A $ мощности $ k $.
    Но каждому подмножеству $ A $ мощности $ k $ соответствует $ A_k^k = k! $ последовательностей элементов этого множества.
    Но тогда имеем, что мощность множества подмножеств $ A $ мощности $ k $ равна
    \begin{equation}
        \label{eq:combinatorics:binomial}
        \frac{A_n^k}{k!} = \frac{n!}{k!(n-k)!} \defeq \binom{n}{k} \defeq C_n^k
    \end{equation}
    Полученное число называется \defemph{числом сочетаний}.
\end{proof}

\vspace{\baselineskip}

\begin{Answer}
    \noindent
    Выберем два разных класса из трёх классов чисел по остатку по модулю три.
    Из каждого класса затем можно тремя способами выбрать по экземпляру.
    В итоге имеем $ C_3^2 \cdot 3 \cdot 3 = 27 $ вариантов.
\end{Answer}



\subsection{Комбинированные задачи}
\label{subsec:combinatorics:combined}

Разберём некоторые другие примеры, которые разбиваются на несколько базовых комбинаторных задач.

\begin{Exercise}[counter=SecExercise]
    \noindent
    Сколькими способами можно выбрать два подмножества $ A $ и $ B $ множества $ \{1, \ldots, 10 \} $ так, чтобы $ |A| = 2 $, $ |B| = 5 $ и $ A \subseteq B $?
\end{Exercise}

\begin{Answer}
    \noindent
    Выберем $ C_{10}^2 $ способами множество $ A $.
    Далее выберем $ C_8^3 $ способами множество $ B \setminus A $.
    В итоге, $ N = C_{10}^2 \cdot C_8^3 $.
\end{Answer}

\begin{Exercise}[counter=SecExercise]
    \noindent
    Сколькими способами можно разбить $ \{ 1, \ldots, 10 \} $ на два непустых подмножества, а затем упорядочить элементы в одном из блоков любым образом?
\end{Exercise}

\begin{Answer}
    \noindent
    Если размер любого блока фиксирован и равен $ k $, то число способов~--- $ N_k = C_{10}^k \cdot (k! + (10 - k)!) = A_{10}^k + A_{10}^{10-k} $.
    Осталось просуммировать по всем возможным значениям размера меньшего блока: $ k \in \{1, \ldots, 5 \} $.
    \[
        N = \sum_{k=1}^5 N_k = \sum_{k=1}^5 A_{10}^k + A_{10}^{10-k}
    \]
\end{Answer}

\begin{Exercise}[counter=SecExercise]
    \noindent
    В русском алфавите $ 33 $ буквы, $ 10 $ из них~--- гласные.
    Сколько всего можно составить слов длины $ 10 $, в которых есть три различные гласные, а согласные идут в строго возрастающем алфавитном порядке?
\end{Exercise}

\begin{Answer}
    \noindent
    Для начала $ C_{10}^3 $ способами выберем три различные гласные для нашего слова.
    Далее $ C_{23}^7 $ способами выберем согласные
    (так как в слове они должны идти в строго возрастающем алфавитном порядке, они все должны быть различные).
    Выбранные буквы расставим $ 10! $ способами.
    Но порядок согласных фиксирован, а потому на каждое подходящее слово приходится еще $ 7! - 1 $ неподходящих.
    Тогда ответ~---
    \[
        N = \frac{C_{10}^3 \cdot C_{23}^7 \cdot 10!}{7!} = C_{10}^3 \cdot C_{23}^7 \cdot A_{10}^3
    \]
\end{Answer}

\begin{Exercise}[counter=SecExercise]
    \noindent
    В группе студентов есть один, который знает С++, Java, Python, Haskell.
    Каждые три из этих языков знают два студента.
    Каждые два~--- $ 6 $ студентов.
    Каждый из этих языков знают по $ 15 $ студентов.
    Каково наименьшее количество студентов в такой группе?
\end{Exercise}

\begin{Answer}
    \noindent
    Наименьшее число достигается тогда и только тогда, когда нет студентов, не знающих ни один из языков.
    Тогда по формуле включений-исключений имеем
    \[
        N = C_4^1 \cdot 15 - C_4^2 \cdot 6 + C_4^3 \cdot 2 - C_4^4 \cdot 1 = 4 \cdot 15 - 6 \cdot 6 + 4 \cdot 2 - 1 \cdot 1 = 31
    \]
\end{Answer}



\subsection{Биномиальные коэффициенты}
\label{subsec:combinatorics:binomial}

Число сочетаний также называется \defemph{биномиальным коэффициентом}.
Это вызвано его появлением в следующей задаче:

\begin{Exercise}[counter=SecExercise]
    \noindent
    Найдите коэффициент при $ a^k b^{n-k} $ после раскрытия скобок в выражении $ (a + b)^n $. % (\defemph{бином Ньютона}).
\end{Exercise}

\begin{Answer}
    \noindent
    Перепишем выражение в виде длинного произведения:
    \[
        (a + b)^n = \underbrace{(a + b) \cdot (a + b) \cdot \ldots \cdot (a + b)}_{n \; \textnormal{раз}}
    \]
    Любое слагаемое вида $ a^k b^{n - k} $ после раскрытия получается только при выборе из некоторых $ k $ скобок числа $ a $,
    а из оставшихся $ n - k $ скобок~--- числа $ b $.
    То есть таких слагаемых будет ровно столько, сколькими способами можно выбрать из $ n $ указанных скобок некоторые $ k $,
    то есть $ C_n^k $.
    Таким образом,
\end{Answer}

\begin{statement}
    \label{statement:combinatorics:Newton_binom}
    \[
        (a + b)^n = \sum_{k = 0}^n C_n^k \cdot a^k b^{n-k} \qquad \text{\defemph{(Бином Ньютона)}}
    \]
\end{statement}

Из связи числа сочетаний с биномом Ньютона очевидным образом следует, что $ C_n^k = C_n^{n-k} $,
хотя это было понятно и из формулы \eqref{eq:combinatorics:binomial}.

\begin{remark}
    \label{remark:combinatorics:sum_binomial}
    $ \displaystyle \sum_{k=0}^n C_n^k = 2^n $; \>
    $ \displaystyle \sum_{k=0}^n (-1)^k C_n^k = 0 $
\end{remark}

\begin{proof}
    $ \displaystyle 2^n = (1 + 1)^n = \sum_{k=0}^n C_n^k \cdot 1^k 1^{n-k} $; \>
    $ \displaystyle 0   = (-1 + 1)^n = \sum_{k=0}^n C_n^k \cdot (-1)^k 1^{n-k} $.
\end{proof}

Есть и много других подобных замечанию \ref{remark:combinatorics:sum_binomial} фактов касательно суммы биномиальных коэффициентов.
Разберём задачу на эту тему:

\begin{Exercise}[counter=SecExercise]
    \noindent
    Докажите справедливость формул (желательно найти комбинаторное доказательство):
    \begin{enumerate}[leftmargin=*]
        %\item $ \displaystyle \sum_{j=0}^k C_r^j C_s^{k-j} = C_{r+s}^k $;
        %\inlineitem $ \displaystyle \sum_{j=0}^n C_j^k = C_{n+1}^{k+1} $;
        %\inlineitem $ \displaystyle \sum_{j=0}^k C_{n+j}^j = C_{n+k+1}^k $;
        \item $ \displaystyle \sum_{j=0}^k \binom{r}{j} \binom{s}{k-j} = \binom{r + s}{k} $;
        \inlineitem $ \displaystyle \sum_{j=0}^n \binom{j}{k} = \binom{n+1}{k+1} $;
        \inlineitem $ \displaystyle \sum_{j=0}^k \binom{n + j}{j} = \binom{n + k + 1}{k} $;
    \end{enumerate}
\end{Exercise}

\begin{Answer}
    \noindent
    Под комбинаторным доказательством понимается доказательство, использующее построение биекции
    между некоторыми двумя множествами, мощность первого из которых равна левой части, а второго~--- правой.
    Такие доказательства обычно красивее и понятнее доказательств сугубо подсчётных
    (например, использующих формулу \eqref{eq:combinatorics:binomial}).
    \begin{enumerate}
        \item
            $ \displaystyle \sum_{j=0}^k \binom{r}{j} \binom{s}{k-j} = \binom{r + s}{k} $.

            В правой части записано число способов выбрать из $ (r + s) $-элементного множества подмножество размера $ k $.
            Заметим, что и в левой части записано то же число.

            Действительно, разобьём условно исходное множество на два подмножества размера $ r $ и $ s $.
            Пусть после выбора подмножества размера $ k $ в первом подмножестве оказалось $ j $ элементов (во втором тогда $ k - j $).
            Всего имеем $ C_r^j \cdot C_s^{k-j} $ способов получить подмножество, удовлетворяющее указанному свойству.
            Просуммировав по всем возможным $ j $ (от $ 0 $ до $ k $), покроем все возможные исходы, причём без повторений.
            Что и требовалось доказать.
        \item
            $ \displaystyle \sum_{j=0}^n \binom{j}{k} = \binom{n+1}{k+1} $.

            В правой части записано число способов выбрать из $ (n+1) $-элементного множества подмножество размера $ k - 1 $.
            Заметим, что и в левой части записано то же число.

            Действительно, упорядочим произвольным образом исходное множество.
            Пусть после выбора подмножества размера $ k + 1 $ оказалось, что наибольший из индексов его элементов равен $ j + 1 $.
            Всего имеем $ C_k^j $ способов получить подмножество, удовлетворяющее указанному свойству:
            оставшиеся $ k $ элементов выбираются среди первых $ j $ исходного множества.
            Просуммировав по всем возможным $ j $ (от $ 0 $ до $ n $), покроем все возможные исходы, причём без повторений.
            Что и требовалось доказать.

        \item
            $ \displaystyle \sum_{j=0}^k \binom{n + j}{j} = \binom{n + k + 1}{k} $.

            Задача похожа на предыдущую.
            Воспользовавшись симметричностью биномиальных коэффициентов, получаем
            \[
                \sum_{j=0}^k \binom{n + j}{n} = \binom{n + k + 1}{n + 1}
            \]
            В правой части записано число способов выбрать из $ (n+k+1) $-элементного множества подмножество размера $ n + 1 $.
            Заметим, что и в левой части записано то же число.

            Действительно, аналогично предыдущей задаче, упорядочим произвольным образом исходное множество.
            Пусть после выбора подмножества размера $ n + 1 $ оказалось, что наибольший из индексов его элементов равен $ n + j + 1 $.
            Всего имеем $ C_{n+j}^j $ способов получить подмножество, удовлетворяющее указанному свойству.
            Просуммировав по всем возможным $ j $ (от $ 0 $ до $ k $), покроем все возможные исходы, причём без повторений.
            Что и требовалось доказать.
    \end{enumerate}
\end{Answer}

\begin{Exercise}[counter=SecExercise]
    \noindent
    Сколькими способами среди $ n $ солдат можно выбрать командира и набрать ему в подчинение отряд произвольного размера?
\end{Exercise}

\begin{Answer}
    \noindent
    С одной стороны, можно $ n $ способами выбрать командира и каждого оставшегося солдата либо взять в отряд, либо нет.
    По правилу произведения имеем следующее число вариантов: $ n \cdot 2^{n-1} $.
    С другой стороны, можно для всех возможных $ k $ сначала $ C_n^k $ способами выбрать отряд размера $ k $,
    а затем в нём $ k $ способами выбрать командира.

    В итоге имеем два тождественно равных ответа:
    \[
        n \cdot 2^{n-1} = \sum_{k=1}^n k \cdot C_n^k
    \]
\end{Answer}

У полученного в предыдущей задаче тождества есть еще одно красивое доказательство,
которое мы получим ближе к концу курса.
А пока докажем полезное рекуррентное соотношение на биномиальные коэффициенты,
которое полезно при построении \emph{треугольником Паскаля}.

\begin{statement}
    $ C_n^k = C_{n-1}^k + C_{n-1}^{k-1} $.
\end{statement}

\begin{proof}
    Рассмотрим задачу выбора подмножества мощности $ k $ из множества мощности $ n $.
    Зафиксируем в исходном множестве некоторый элемент.
    Тогда при выборе $ k $-элементного подмножества мы можем либо включить данный элемент, либо не включить.
    В первом случае имеем $ C_{n-1}^{k-1} $ вариантов выбора, а во втором~--- $ C_{n-1}^k $.
\end{proof}

Рассмотрим еще одну классическую задачу, в которой возникают биномиальные коэффициенты.

\begin{Exercise}[counter=SecExercise]
    \noindent
    Найдите число решений уравнения $ x_1 + x_2 + \ldots + x_k = n $ в неотрицательных целых числах.
\end{Exercise}

\begin{Answer}
    \noindent
    Решим задачу \emph{методом точек и перегородок}.
    Заметим, что число решений равно числу способов разделить $ n $ неразличимых точек $ (k - 1) $-ой неразличимой перегородкой.
    Действительно, будем интерпретировать число точек в каждой секции как значение соответствующей переменной $ x_i $.
    Таким образом, имеем биекцию.

    Число способов так разделить $ n $ точек можно найти следующим образом:
    <<свалим>> в общую кучу точки и перегородки, перемешаем их $ (n + k - 1)! $ способами,
    а затем разделим на $ n! $ и $ (k-1)! $, учтя тем самым неразличимость точек и перегородок между собой.
    Итоговый ответ:
    \[
        N_{\textnormal{решений}} = \binom{n + k - 1}{k - 1} \qquad \textnormal{\defemph{(формула Муавра)}}
    \]
\end{Answer}


\subsection{Мультиномиальные коэффициенты}
\label{subsec:combinatorics:multinomial}

Утверждение \ref{statement:combinatorics:Newton_binom} сформулировано только для случая возведения в $ n $-ую степень суммы \emph{двух} слагаемых.
Получим общую формулу:

\begin{statement}
    \label{statement:combinatorics:multinom}
    \[
        (x_1 + x_2 + \ldots + x_m)^n = \sum_{k_1 + k_2 + \ldots + k_m = n} \binom{n}{k_1, \; k_2, \; \ldots, k_m} x_1^{k_1} x_2^{k_2} \ldots x_m^{k_m},
    \]
    где
    \[
        \binom{n}{k_1, \; k_2, \; \ldots, k_m} = \binom{n}{k_1} \cdot \binom{n - k_1}{k_2} \cdot \ldots \cdot \binom{n - k_1 - \ldots - k_m}{k_m} =
        \frac{n!}{k_1! k_2! \ldots k_m!}
    \]
    --- \defemph{мультиномиальный коэффициент}.
\end{statement}

\begin{proof}
    Аналогично доказательству утверждения \ref{statement:combinatorics:Newton_binom}:
    для получения монома вида $ x_1^{k_1} x_2^{k_2} \ldots x_m^{k_m} $ при раскрытии скобок
    мы $ C_n^{k_1} $ способами выбираем скобки, из которых берём $ x_1 $, $ C_{n-k_1}^{k_2} $ способами~--- скобки, из которых берём $ x_2 $, и так далее.
\end{proof}

\begin{Exercise}[counter=SecExercise]
    \noindent
    Сколько различных слов (не обязательно осмысленных) можно получить, переставляя буквы в словах
    \begin{enumerate}[label=\textbf{\alph*)}]
        \item <<КОМПЬЮТЕР>>;
        \inlineitem <<ЛИНИЯ>>;
        \inlineitem <<ПАРАБОЛА>>;
        \item <<ОБОРОНОСПОСОБНОСТЬ>>?
    \end{enumerate}
\end{Exercise}

\begin{Answer}
    \noindent
    \begin{enumerate}[label=\textbf{\alph*)}]
        \item
            Всё девять букв различны, поэтому достаточно переставить их произвольным образом: $ N = 9! $.
        \item
            Среди пяти букв повторяется только <<И>>, причём два раза.
            Сначала $ 5! $ способами расставим буквы из предположения, что они все различимы,
            а затем разделим на $ 2! $, учтя тем самым, что две буквы неразличимы, а потому их перестановка ничего не изменит: $ N = 5! / 2! $.
        \item
            Среди восьми букв повторяется только <<А>>, причём три раза.
            Аналогично, $ N = 8! / 3! $.
        \item
            Среди восемнадцати букв <<О>> повторяется семь раз, буква <<С>>~--- три, <<Б>> и <<Н>>~---  два, остальные буквы встречаются по одному разу.
            Тогда $ \displaystyle N = \frac{18!}{7! 3! 2! 2!} $.
    \end{enumerate}
\end{Answer}

\begin{remark}
    Пусть имеется $ m $ букв в количествах $ k_1 $, $ k_2 $, \ldots, $ k_m $ соответственно ($ k_1 + k_2 + \ldots + k_m = n $).
    Тогда число различных (не обязательно осмысленных) слов, которые можно из данных букв составить~--- $ \binom{n}{k_1, \; k_2, \; \ldots, \; k_m} $.
\end{remark}
  % Комбинаторика.
\section{Бинарные отношения}
\label{sec:binary_relations}

В данном разделе мы поговорим о некоторого рода обобщении понятия <<функция>>~--- о \defemph{бинарном отношении}.
Напомним, что согласно определению \ref{definition:functions:function} функция $ f $~---
это подмножество некоторого декартового произведения $ X \times Y $,
обладающее свойством \defemph{функциональности}:
\[
    \left[ (x, y) \in f \wedge (x, y') \in f \right] \rightarrow (y = y')
\]
Иначе говоря, любому элементу из $ X $ ставится в соответствее не более одного элемента из $ Y $.
Однако это довольно сильное ограничение.
Например, функциями невозможно описать отношения между некоторыми студентами и их увлечениями:
любой студент вполне может иметь несколько интересных ему занятий,
ровно как и некоторым занятием могут увлекаться несколько человек.

В этой связи в математике отдельно рассматриваются и подмножества $ X \times Y $,
не обязательно обладающие свойством функциональности~--- \defemph{бинарные отношения}.

\begin{definition}
    \label{definition:binary_relations:binary_relation}
    \defemph{Бинарным отношением} между двумя множествами $ A $ и $ B $ называется любое множество $ R \subseteq A \times B $.
    Если $ A = B $, то $ R \subseteq A^2 $ и говорят, что бинарное отношение задано на множестве $ A $.
    Принадлежность $ (a, b) \in R $ кратко записывают как $ a R b $.
\end{definition}

\begin{remark}
    Заметим, что определение \ref{definition:binary_relations:binary_relation} обобщается и на случай произвольного (пусть $ n $) числа множителей в декартовом произведении.
    В таком случае говорят об \defemph{отношении арности $ n $}.
    В нашем курсе мы подробно изучать их не будем.
\end{remark}

Помимо свойства функциональности бинарное отношение может обладать целым набором других интересных свойств.
Перечислим наиболее важные из них:
\begin{definition}
    \label{definition:binary_relations:types_AB}
    Пусть $ R \subseteq A \times B $.
    Отношение $ R $ называется
    \begin{enumerate}
        \item \defemph{функциональным} в случае $ \forall a \in A \;\; \forall b, b' \in B \; \left[ a R b \wedge a R b' \right] \rightarrow (b = b') $.
        \item \defemph{(левым) тотальным} в случае $ \forall a \in A \;\; \exists b \in B: \; a R b $.
        \item \defemph{инъективным} в случае $ \forall a, a' \in A \;\; \forall b \in B \; \left[ a R b \wedge a' R b \right] \rightarrow (a = a') $.
        \item \defemph{сюръективным} в случае $ \forall b \in B \;\; \exists a \in A: \; a R b $.
    \end{enumerate}
\end{definition}
Используя \ref{definition:binary_relations:types_AB}, легко дать определение, например, инъекции,
как функционального тотального инъективного бинарного отношения.

Еще большим числом особых свойств могут обладать бинарные отношения,
заданные на одном множестве.
\begin{definition}
    \label{definition:binary_relations:types_AA}
    Пусть $ R \subseteq A^2 $.
    Отношение $ R $ называется
    \begin{enumerate}
        \item \defemph{рефлексивным} в случае $ \forall a \in A \;\; a R a $.
        \item \defemph{антирефлексивным} в случае $ \forall a \in A \;\; \neg (a R a) $.
        %\item \defemph{корефлексивным} в случае $ \forall a, b \in A \;\; a R b \rightarrow (a = b) $.
        \item \defemph{симметричным} в случае $ \forall a, b \in A \;\; \left( a R b \rightarrow b R a \right) $.
        \item \defemph{антисимметричным} в случае $ \forall a, b \in A \;\; \left[ a R b \wedge b R a \rightarrow (a = b) \right] $.
        \item \defemph{асимметричным} в случае $ \forall a, b \in A \;\; \left[ a R b \rightarrow \neg (b R a) \right] $.
        \item \defemph{транзитивным} в случае $ \forall a, b, c \in A \;\; \left[ a R b \wedge b R c \rightarrow a R c \right] $.
    \end{enumerate}
\end{definition}

\begin{remark}
    Любое рефлексивное бинарное отношение по определению является тотальным и сюръективным.
\end{remark}

Множество уже известных вам математических объектов и операций по сути являются бинарными отношениями.
Например, операции сравнения: $ \{ (a, b) \mid a \, (*) \, b \} \subseteq \R^2 $, где вместо $ (*) $ можно подставить $ <, \leqslant, =, \neq, \geqslant, > $.
Когда рассматривают операции сравнения в терминах бинарных отношений, их обычно заключают в круглые скобки.
Например, $ (<) = \{ (a, b) \mid a < b \} \subseteq \R^2 $.

\begin{Exercise}[counter=SecExercise]
    \noindent
    Какие из указанных в \ref{definition:binary_relations:types_AB} и \ref{definition:binary_relations:types_AA}
    свойств выполнены для следющих бинарных отношений: $ (<), (\leqslant), (=), (\neq), (\geqslant), (>) $?
    Считайте, что $ \R $~--- множество, на котором заданы отношения.
\end{Exercise}

\begin{Answer}
    \noindent
    Нетрудно заметить, что все отношения тотальные и сюръективные.
    Также все отношения, кроме $ (\neq) $, транзитивные.
    Отношение $ (=) $ при этом также функциональное, инъективное, рефлексивное и симметричное.
    Отношение $ (\neq) $ антирефлексивное и симметричное.
    Отношения $ (\leqslant) $ и $ (\geqslant) $ рефлексивные и антисимметричные.
    Наконец, отношения $ (<) $ и $ (>) $ антирефлексивные и асимметричные.
\end{Answer}

Так как бинарные отношения являются множествами, к ним применимы теоретикомножественные операции:
отношения можно объединять, пересекать, вычитать, брать дополнение к ним и так далее.
Однако этим операции над бинарными отношениями не ограничиваются.

\begin{definition}
    \defemph{Обратным отношением} к бинарному отношению $ R \subseteq A \times B $ называется отношение
    $ R^{-1} = \{(b, a) \mid a R b \} \subseteq B \times A $.
\end{definition}

\begin{Exercise}[counter=SecExercise, label={exercise:relations:is_function}]
    \noindent
    Нарисуйте двудольный граф, соответсвующий бинарному отношению
    $ R \subseteq \{a, b, c, d, e\} \times \{1, 2, 3, 4, 5, 6\} $:
    \[
        R = \{ (a, 1), (a, 2), (b, 4), (c, 3), (d, 5) \}
    \]
    \begin{enumerate}[label=\textbf{\alph*)}]
        \item Является ли $ R $ функцией?
        \inlineitem Является ли $ R^{-1} $ функцией?
    \end{enumerate}
\end{Exercise}

\begin{Answer}
    \noindent
    Граф приведён на рисунке \ref{fig:relations:from_is_function}.
    \begin{enumerate}[label=\textbf{\alph*)}]
        \item
            Нет, не является, так как $ (a, 1) \in R $ и $ (a, 2) \in R $.
        \item
            Да, является, так как $ \forall b \in \{1, \ldots, 6 \} $
            множество $ \{ a \mid b R^{-1} a \} $ содержит не более одного элемента.
    \end{enumerate}
\end{Answer}

\begin{figure}[ht!]
    \center
    \begin{tikzpicture}
        \node[circle,fill,inner sep=1.5pt,label=left:$ a $] (a)  at (0,-0.5) {};
        \node[circle,fill,inner sep=1.5pt,label=left:$ b $] (b)  at (0,-1.5) {};
        \node[circle,fill,inner sep=1.5pt,label=left:$ c $] (c)  at (0,-2.5) {};
        \node[circle,fill,inner sep=1.5pt,label=left:$ d $] (d)  at (0,-3.5) {};
        \node[circle,fill,inner sep=1.5pt,label=left:$ e $] (e)  at (0,-4.5) {};

        \node[circle,fill,inner sep=1.5pt,label=right:$ 1 $] (u1)  at (2,0) {};
        \node[circle,fill,inner sep=1.5pt,label=right:$ 2 $] (u2)  at (2,-1) {};
        \node[circle,fill,inner sep=1.5pt,label=right:$ 3 $] (u3)  at (2,-2) {};
        \node[circle,fill,inner sep=1.5pt,label=right:$ 4 $] (u4)  at (2,-3) {};
        \node[circle,fill,inner sep=1.5pt,label=right:$ 5 $] (u5)  at (2,-4) {};
        \node[circle,fill,inner sep=1.5pt,label=right:$ 6 $] (u6)  at (2,-5) {};

        \path [->] (a) edge node {}  (u1);
        \path [->] (a) edge node {}  (u2);
        \path [->] (b) edge node {}  (u4);
        \path [->] (c) edge node {}  (u3);
        \path [->] (d) edge node {}  (u5);
    \end{tikzpicture}
    \caption{граф отношения $ R $ из задачи \ref{exercise:relations:is_function}}
    \label{fig:relations:from_is_function}
\end{figure}


\begin{definition}
    \defemph{Композицией} двух бинарных отношений $ R \subseteq A \times B $ и $ Q \subseteq B \times C $
    называется бинарное отношение $ (Q \circ P) = \{ (a, c) \mid \exists b \in B: \; a R b \wedge b Q c \} $.
    %\newline
    \\[0.25\baselineskip]
    Акцентируем особое внимание на порядок записи отношений в композиции.
    Он повторяет порядок записи функций в композиции.
\end{definition}

\begin{Exercise}[counter=SecExercise]
    \noindent
    Найдите результат операций над отношениями, определенными на множестве действительных чисел.
    \begin{enumerate}[label=\textbf{\alph*)}]
        \item $ (>)^c $;
        \inlineitem $ (>)^{-1} $;
        \inlineitem $ (\geqslant) \symdiff (\leqslant) $;
        \inlineitem $ (>) \cap (<) $;
        \inlineitem $ (=) \circ (>) $;
        \inlineitem $ (<) \circ (<) $;
        \item $ (<) \circ (>) $.
    \end{enumerate}
\end{Exercise}

\begin{Answer}
    \noindent
    \begin{enumerate}[label=\textbf{\alph*)}]
        \item $ (\leqslant) $;
        \inlineitem $ (<) $;
        \inlineitem $ (\neq) $;
        \inlineitem $ \varnothing $;
        \inlineitem $ (>) $;
        \inlineitem $ (<) $;
        \inlineitem $ \R^2 $;
    \end{enumerate}
\end{Answer}


\subsection{Отношения эквивалентности}
\label{subsec:binary_relations:equivalency}

Отношения подобия треугольников или параллельности прямых также являются бинарными отношениями.
Можно заметить, что оба этих отношения (если считать прямую параллельной самой себе)
вместе с отношением $ (=) $ и многими другими являются рефлексивными, симметричными и транзитивными.
Такие отношения выделяют в отдельную группу.

\begin{definition}
    Рефлексивное, симметричное и транзитивное бинарное отношение называют \defemph{отношением эквивалентности}.
\end{definition}

\begin{theorem}
    \label{theorem:binary_relations:eq_classes_partition}
    Пусть на конечном множестве $ A $ задано отношение эквивалентности $ R $.
    Тогда $ A = B_1 \sqcup B_2 \sqcup \ldots \sqcup B_k $,
    причём $ \forall i \; B_i \neq \varnothing $ и $ \forall a \in B_i \; \forall b \in B_j \; \left[ a R b \leftrightarrow (i = j) \right] $.
    То есть, два элемента образуют пару, лежащую в отношении, тогда и только тогда, когда они взяты из одного блока разбиения.
\end{theorem}

\begin{remark}
    Теорема \ref{theorem:binary_relations:eq_classes_partition} справедлива и для бесконечных $ A $.
    Тогда число блоков не обязательно конечно или даже счётно.
\end{remark}

\begin{definition}
    Блоки $ B_i $ из \ref{theorem:binary_relations:eq_classes_partition} называются \defemph{классами эквивалентности},
    а про $ A $ говорят, что оно разбито на классы эквивалентности.
\end{definition}

Из теоремы \ref{theorem:binary_relations:eq_classes_partition} следует,
что для любого элемента $ A $ однозначно определён класс эквивалентности, в котором данный элемент лежит.
В связи с этим получаем эквивалентное определение класса эквивалентности:

\begin{definition}
    Пусть $ R \subseteq A^2 $~--- отношение эквивалентности, $ a \in A $.
    Тогда \defemph{классом эквивалентности} отношения $ R $, построенным по представителю $ a $ называется множество
    $ [a]_R = \{ b \in A \mid a R b \} $.
\end{definition}

\begin{corollary}
    Любые два класса эквивалентности либо совпадают, либо не пересекаются:
    $ \forall a, b \in A \;\; ([a]_R = [b]_R) \oplus ([a]_R \cap [b]_R = \varnothing) $.
\end{corollary}

\begin{Exercise}[counter=SecExercise]
    \noindent
    Рассмотрим множество $ S $ фундаментальных последовательностей, состоящих из рациональных чисел.
    Введём следующее бинарное отношение $ (\sim) \subseteq S^2 $:
    \[
        \{x_i\}_{i = 1}^{\infty} \sim \{y_j\}_{j = 1}^{\infty}
        \quad \Longleftrightarrow \quad
        \forall \varepsilon > 0 \;\; \exists N \in \N: \forall n,m > N \;\; |x_n - y_m| < \varepsilon
    \]
    Является ли $ (\sim) $ отношением эквивалентности?
\end{Exercise}

\begin{Answer}
    \noindent
    Симметричность очевидным образом следует из симметричности формулы в условии.
    Рефлексивность эквивалентна фундаментальности любой последовательности из $ S $ (совпадение с определением буквальное),
    что дано по условию.
    Осталось проверить транзитивность.
    Для этого достаточно взять $ \varepsilon' = \varepsilon / 2 $ и воспользоваться неравенством треугольника:
    \begin{multline*}
        \left .
        \begin{aligned}
            \forall n,k > N_1(\varepsilon') \;\; |x_n - y_k| < \varepsilon' \\
            \forall k,m > N_2(\varepsilon') \;\; |y_k - z_m| < \varepsilon'
        \end{aligned}
        \right \}
        \quad \Longrightarrow \quad
        \forall n, m, k > \max\{N_1, N_2\} \;\; |x_n - z_m| = \\
        = |x_n - y_k + y_k - z_m| < |x_n - y_k| + |y_k - z_m| < 2 \varepsilon' = \varepsilon
    \end{multline*}
    Отсюда имеем транзитивность.

    Интересно, что для классов эквивалентности рассматриваемого отношения можно ввести стандартные математические операции
    (например, $ [\{x_i\}_{i=1}^{\infty}]_{(\sim)} + [\{y_i\}_{i=1}^{\infty}]_{(\sim)} = [\{x_i + y_i\}_{i=1}^{\infty}]_{(\sim)} $;
    проверьте непротиворечивость такого определения).
    Более того, после этого классы приобретаеют свойства \emph{действительных чисел}.
    Можно в некотором смысле саказать, что это и есть все действительные числа (см. \emph{изоморфизм}).
\end{Answer}


\subsection{Отношения частичного порядка}
\label{subsec:binary_relations:order}

В прошлом разделе мы упомянули один из важнейших классов бинарных отношений, представители которого по свойствам близки к отношению равенства.
Логично предположить, что и у других изученных бинарных отношениях на числах есть аналогичные <<братья>>.
Действительно, можно, например, заметить, что у отношений $ (\leqslant) $ и $ (\subseteq) $\footnote{не будем пока формально определять, на чём они заданы}
много общих свойств: оба отношения \emph{рефлексивные}, \emph{антисимметричные} и \emph{транзитивные}.
Также, например, $ (<) $ и $ (\subsetneq) $ \emph{антирефлексивные}, \emph{антисимметричные} и \emph{транзитивные}.
Такие отношения также выделяют в отдельную группу.

\begin{definition}
    Транзитивное, антисимметричное и \underline{либо} рефлексивное, \underline{либо} антирефлексивное бинарное отношение называют \defemph{отношением (частичного) порядка}.
    %\newline
    В случае рефлексивности говорят, что отношение порядка \defemph{нестрогое}, иначе~--- \defemph{строгое}.
\end{definition}

\begin{remark}
    \label{remark:binary_relations:order_bijection}
    Из любого отношения нестрогого порядка можно получить строгое, вычтя из него отношение $ (=) $ как множество.
    Аналогично можно совершить и обратное преобразование.
    Таким образом, между множеством строгих и нестрогих порядков определена биекция.
\end{remark}

\begin{definition}
    Пусть $ R $~--- отношение частичного порядка.
    Будем обозначать $ (<_R) $ и $ (\leqslant_R) $ строгую и нестрогую версию $ R $ соответственно,
    полученные согласно биекции из \ref{remark:binary_relations:order_bijection}.
    \textit{Во избежание путанницы заметим, что хотя бы с одной из этих версий $ R $ совпадает по определению.}
\end{definition}

\begin{definition}
    Отношения порядка, в которых любые два элемента сравнимы (то есть $ \forall a \forall b \in A \; \left[a R b \vee b R a \right]$),
    называются \defemph{линейными}.
    Заметим, что линейными могут быть только отношения нестрогого порядка.
\end{definition}

В некоторых задачах исследование отношения порядка может быть неудобным из-за большого числа <<неинфомативных>> пар.
Например, отношение $ (<) $ на $ \Z $ однозначно задаётся парами вида $ (n, n+1) $ и знанием о его транзитивности и линейности;
необязательно рассматривать все пары, для того чтобы полностью восстановить всё отношение.
В связи с этим вводится следующее понятие:

\begin{definition}
    Пусть $ R \subseteq A^2 $~--- отношение частичного порядка.
    Отношением \defemph{непосредственного следования}, построенным по $ R $, является бинарное отношение
    \[
        (\prec_R) = \left\{ (x, y) \mid (x <_R y) \wedge \neg \left( \exists z \in A: (x <_R z) \wedge (z <_R y) \right) \right\}
    \]
\end{definition}

Название отношения полностью соответствует его определению.
Рассмотрим несколько примеров:

\begin{example}
    \begin{enumerate}
        \item[]
        \item
            Рассмотрим $ (<) \subseteq \Z^2 $.
            Тогда $ \prec_{(<)} = \left\{ (n, n+1) \mid n \in \Z \right\} $.
        \item
            Рассмотрим $ (<) \subseteq \R^2 $.
            Из плотности действительных чисел самих в себе следует, что $ \prec_{(<)} = \varnothing $.
            Это действительно согласуется с интуицией: ни для какого действительного числа нельзя назвать другое,
            следующее непосредственно за ним.
        \item
            Рассмотрим $ (\subsetneq) \in \mathcal{S}^2 $, где $ \mathcal{S} $~--- некоторое семейство множеств.
            Тогда
            \[
                \prec_{(\subsetneq)} = \left\{ (x, y) \mid (x \subseteq y) \wedge (|y \setminus x| = 1) \right\}
            \]
            Видно, что полученное отношение не обязательно функционально или инъективно:
            у множества может быть несколько непосредственно следующих после него множеств;
            точно так же и наоборот.
    \end{enumerate}
\end{example}

\begin{definition}
    Пусть $ R \subseteq A^2 $~--- отношение частичного порядка.
    Ориентированный граф $ G(A, \prec_R) $ называется \defemph{диаграммой Хассе} отношения $ R $.
\end{definition}

\begin{definition}
    Диаграмма Хассе, построенная для частичного порядка $ (\subseteq) \subseteq 2^A \times 2^A $, где $ |A| = n < \infty $,
    называется \defemph{ориентированным булевым кубом}.
    Его неориентированная версия называется просто \defemph{булевым кубом} и обозначается $ B_n $.
    %\newline
    %\textit{Отметим, что это эквивалентное определение, а каноническое будет дано ниже.}
\end{definition}

\begin{remark}
    Вершины булева куба обычно обозначают двоичными словами длины $ n $,
    являющимися векторами значений индикаторных функций соответствующих подмножеств
    ($ i $-ый бит отвечает за наличие $ i $-ого элемента $ A $ в соответствующем подмножестве).
\end{remark}

\begin{figure}[ht!]
    \center
    \begin{tikzpicture}[
    back line/.style={densely dotted},
    cross line/.style={preaction={draw=white, -,
    line width=6pt}}]
        \matrix (m) [matrix of math nodes,
        row sep=3em, column sep=3em,
        text height=1.5ex,
        text depth=0.25ex]{
        & 110 & & 111 \\
        100 & & 101 \\
        & 010 & & 011 \\
        000 & & 001 \\
        };
        \path[->] (m-4-1) edge (m-2-1);
        \path[->] (m-2-1) edge (m-1-2);
        \path[->] (m-1-2) edge (m-1-4);

        \path[->] (m-2-1) edge (m-2-3);
        \path[->] (m-2-3) edge (m-1-4);

        \path[->] (m-4-1) edge (m-4-3);
        \path[->] (m-4-3) edge (m-3-4);
        \path[->] (m-3-4) edge (m-1-4);

        \path[->] (m-4-1) edge [back line] (m-3-2);
        \path[->] (m-3-2) edge [back line] (m-3-4);

        \path[->] (m-3-2) edge [back line] (m-1-2);
        \path[->] (m-4-3) edge (m-2-3);
        %(m-2-3) edge [cross line] (m-4-3);
    \end{tikzpicture}
    \label{fig:binary_relations:boolean_cube}
    \caption{пример ориентированного булева куба для $ n = 3 $.}
\end{figure}

На практике часто встречаются задачи, в которых надо ввести частичный порядок на некотором декартовом произведении.
При этом на множителях обычно некоторый порядок уже введён.
В этих случаях удобна следующая конструкция:

\begin{definition}
    Пусть $ P \subseteq A^2 $, $ Q \subseteq B^2 $~--- отношения порядка.
    Их \defemph{произведением} называется бинарное отношение $ P \subseteq (A \times B)^2 $:
    \[
        (x, y) \, P \, (x', y') \; \Longleftrightarrow \;
        \begin{cases}
            x R x' \\
            y Q y'
        \end{cases}
    \]
    Вводится следующее обозначение: $ P = A \times B $, хотя, формально, с точки зрения множеств оно означает другое.
\end{definition}

Заметим, что произведение отношения порядков не всегда является отношеним порядка.
Для этого требуется, чтобы множители были либо одновременно нестрогими порядками, либо строгими.

\begin{remark}
    \label{remark:binary_relations:boolean_cube_isomorphism}
    Ориентированный булев куб также является диаграммой Хассе для отношения порядка
    $ (\leqslant \times \leqslant \times \ldots \times \leqslant) \subseteq \{ 0, 1 \}^n \times \{ 0, 1 \}^n $~---
    \defemph{отношения покоординатного порядка}, введённого на двоичных словах длины $ n $.
    Обратим внимание, что \textbf{\uline{это каноническое определение булева куба в нашем курсе}}.
\end{remark}

Данное замечание хорошо иллюстрируется следующей задачей:
\begin{Exercise}[counter=SecExercise]
    \noindent
    Граф $ G_n = (V, E) $ имеет множество вершин $ V = 2^{\{1, 2, \ldots, n\}} $
    (вершина $ v \in V $~--- подмножество множества $ \{1, 2, 3, \ldots, n \} $);
    вершины $ v $ и $ u $ соединены ребром тогда и только тогда, когда $ | u \symdiff v | = 1 $.
    \begin{enumerate}[label=\alph*)]
        \item
            Докажите, что граф $ G_n $ изоморфен булеву кубу $ B_n $.
        \item
            Сколько существует различных наборов (попарно различных) подмножеств $ A_1, A_2, A_3 \subseteq \{1, 2, \ldots, n\} $,
            для которых выполняется условие $ |A_1 \symdiff A_2| = |A_2 \symdiff A_3| = 1 $?
    \end{enumerate}
\end{Exercise}

\begin{Answer}
    \noindent
    \begin{enumerate}[label=\alph*)]
        \item
            Изоморфизм, по сути, уже построен в замечании \ref{remark:binary_relations:boolean_cube_isomorphism}:
            действиетльно, любому подмножеству $ A \subseteq V $ взаимнооднозначно сопоставляется двоичное слово длины $ |V| $ (вектор значений индикаторной функции),
            причём $ A_1 \subseteq A_2 $ тогда и только тогда, когда вектор, соответствующий $ A_1 $,
            сравним по отношению покоординатного порядка с вектором, соответвтующим $ A_2 $,
            и покоординатно не больше его.
            Убирая ориентацию рёбер, мы не нарушаем изоморфизм.
            Что и требовалось доказать.
        \item
            В силу предыдущего пункта задача эквивалентна поиску путей длины два в булевом кубе.
            Так как всего вершин $ 2^n $, а степень каждой вершины в $ B_n $ равна $ n $, таких путей~--- $ \frac{1}{2} \cdot 2^n \cdot A_n^2 $
            (делим на два так как посчитали каждый путь дважды).
            То есть, ответ~--- $ N = n (n-1) 2^{n-1} $.
    \end{enumerate}
\end{Answer}

Введём, наконец, некоторый дополнительный глоссарий, связанный с отношениями порядка.

\begin{definition}
    \label{definition:binary_relations:max}
    Рассмотрим отношение частичного порядка $ R \subseteq A^2 $.
    Элемент $ x \in A $ является
    \begin{enumerate}
        \item
            \defemph{максимальным} в случае $ \neg \left[ \exists a \in A: (x <_R a) \right] $.
        \item
            \defemph{наибольшим} в случае $ \forall a \in A \; (a \leqslant_R x) $.
    \end{enumerate}
    Аналогично определяются \defemph{минимальный} и \defemph{наименьший} элементы.
\end{definition}

\begin{remark}
    \label{remark:binary_relations:minimal}
    Наибольший/наименьший элемент обязательно является максимальным/минимальным.
\end{remark}

\begin{proof}
    Действительно, предположим противное: есть наибольший, но не максимальный по отношению порядка $ R \subseteq A^2 $ элемент $ x $.
    Тогда, отрицая максимальность, имеем $ \exists a \in A: (x <_R a) $.
    Но в силу антисимметричности $ R $ тогда не может быть, чтобы $ a \leqslant_R x $.
    Противоречие с максимальностью.
    Аналогично для наименьшего элемента.
\end{proof}

\begin{Exercise}[counter=SecExercise]
    \noindent
    Постройте отношение частичного порядка, в котором деревья (и только они) на некотором наборе вершин $ V $ будут минимальными элементами.
    Существуют ли для этого отношения наименьшие элементы?
\end{Exercise}

\begin{Answer}
    \noindent
    Рассмотрим $ A = \left\{ G(V, E) \mid \left( E \subseteq \binom{V}{2} \right) \wedge (G \; \textnormal{связный} ) \right\} $.
    Отношение введём следующим образом: $ R = \left\{ (G, G') \mid E(G) \subseteq E(G') \right\} $.
    Нетрудно заметить, что это отношение нестрогого частичного порядка, так как таковым является $ (\subseteq) $.

    В терминах введённого отношения определение минимального элемента эквивалентно определению дерева:
    связный граф, удаление любого ребра из которого нарушает связность, что эквивалентно связному графу,
    для которого не существует связного подграфа на тех же вершинах.

    Наименьшего элемента в общем случае нет: если $ |V| > 2 $, то может быть построено несколько деревьев.
    Так как это все минимальные элементы, среди них согласно \ref{remark:binary_relations:minimal} и надо искать наименьший элемент.
    Но понятно, что любые два разных дерева не сравнимы.
    Значит, наименьшего элемента нет.
\end{Answer}

В предыдущей задаче мы ввели отношение порядка, которое, вообще говоря, может быть задано на всех графах с вершинами $ V $, а не только на связных.
Здесь логично ввести следующее определение:

\begin{definition}
    Пусть бинарное отношение $ R $ задано на множестве $ A $.
    Пусть $ B \subseteq A $.
    Тогда $ R \cap B^2 $ является бинарным отношением, заданным на множестве $ B $,
    и называется \defemph{сужением} $ R $ на $ B $.
\end{definition}

В связи с этим определением аналогично \ref{definition:binary_relations:max} вводятся понятия максимального,
минимального, наибольшего и наименьшего элемента в подмножестве.
      % Отношения.

\silentsection{Задачи из канонического задания} \label{sec:exercises}

В этом разделе приведён список указателей на задачи из канонического задания.
Не все решения добавлены в данный конспект.

\subsection*{Алгебра логики: введение}

\subsection*{Множества и логика}

\subsection*{Математические определения, утверждения и доказательства}

\subsection*{Графы I: неориентированные графы}

\begin{enumerate}[label=\textbf{№\arabic*}:]
    \item Определение \ref{def:graphs:named_graphs} и утверждение \ref{remark:graphs:subsets_cardinality}.
    \item Задача \ref{exercise:graphs:inconsistent_degrees}.
    \item Задача \ref{exercise:graphs:sum_of_degs_less}.
    \item Задача \ref{exercise:graphs:clique_independent_set}.
    \item Задача \ref{exercise:graphs:connected_graphs_union}.
    \item Задача \ref{exercise:graphs:graph_or_complement_connected}.
    \item Задача \ref{exercise:graphs:max_edges_in_disconnected_graph}.
    \item Задача \ref{exercise:graphs:union_of_cycles}.
    \item Задача \ref{exercise:graphs:BFS_sort}.
    \item Задача \ref{exercise:graphs:disjoined}.
\end{enumerate}

\subsection*{Графы II: деревья и раскраски}

\begin{enumerate}[label=\textbf{№\arabic*}:]
    \item Задача \ref{exercise:graphs:star_tree}.
    \item Задача \ref{exercise:graphs:small_tree_big_deg}.
    \item Задача \ref{exercise:graphs:more_then_half_leaves}.
    \item Утверждение \ref{statement:graphs:connected_has_spanning_tree}.
    \item Задача \ref{exercise:graphs:long_walk} и \ref{exercise:graphs:disjoined}.
    \item Задача \ref{exercise:graphs:walks_and_coloring}.
    \item Следствие \ref{corollary:graphs:tree_2_coloring} и задача \ref{exercise:graphs:tree_2_coloring}.
    \item Задача \ref{exercise:graphs:clique_max_colors}.
    \item Задача \ref{exercise:graphs:can_delete_one_vertex}.
    \item Задача \ref{exercise:graphs:rad_and_diam}.
\end{enumerate}

\subsection*{Двудольные графы, паросочетания и функции}

\begin{enumerate}[label=\textbf{№\arabic*}:]
    \item Задача \ref{exercise:functions:sets_examples}.
    \item Задача \ref{exercise:functions:max_prime_devisor}.
    \item Задача \ref{exercise:functions:set_operations}.
    \item Задача \ref{exercise:functions:map_unmap}.
    \item Определение \ref{definition:functions:cardinaluty_compare} и утверждение \ref{statement:functions:cardinality_compare_equivalence}.
    \item Задача \ref{exercise:functions:infinite_surjection}.
\end{enumerate}

\subsection*{Комбинаторика I. Правила суммы и произведения}

\begin{enumerate}[label=\textbf{№\arabic*}:]
    \item Задача \ref{exercise:combinatorics:flowers}.
    \item Задача \ref{exercise:combinatorics:triangles}.
    \item Задача \ref{exercise:combinatorics:monotonic_digits}.
    \item Задача \ref{exercise:combinatorics:at_least_one_vowel}.
    \item Задача \ref{exercise:combinatorics:subsets_and_binary_words}.
    \item Задача \ref{exercise:combinatorics:stairs}.
    \item Задача \ref{exercise:combinatorics:subsets_decomposition}.
    \item Задача \ref{exercise:combinatorics:queue}.
    \item Задача \ref{exercise:combinatorics:num_of_pairs}.
    \item Задача \ref{exercise:combinatorics:checkers}.
    \item Задача \ref{exercise:combinatorics:odd_even_monotonic}.
\end{enumerate}

\subsection*{Комбинаторика II. Биномиальные коэффициенты}

\subsection*{Комбинаторика III. Формула включений-исключений}

\subsection*{Бинарные отношения и их графы. Отношения эквивалентности}

\subsection*{Ориентированные графы и отношения порядка}

\subsection*{Булевы функции}

\begin{enumerate}[label=\textbf{№\arabic*}:]
    \item Задача \ref{exercise:boolean:DNF_example}.
    \item Задача \ref{exercise:boolean:preserves_0_1}.
    \item Задача \ref{exercise:boolean:graph_to_function}.
    \item Задача \ref{exercise:boolean:f_existance}.
    \item Задача \ref{exercise:boolean:Post_example}.
    \item Задача \ref{exercise:boolean:linear_closure}.
    \item Задача \ref{exercise:boolean:monotonous_and_MAJ}.
    \item Задача \ref{exercise:boolean:self_dual}.
    \item Задача \ref{exercise:boolean:not_self_dual}.
    \item Задача \ref{exercise:boolean:move_negation_to_arguments}.
\end{enumerate}


\subsection*{Производящие функции-1}

\subsection*{Производящие функции-2}
      % Указатель для задач.

\end{document}
