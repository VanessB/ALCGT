\section{Комбинаторика}
\label{sec:combinatorics}

В предыдущем разделе мы подробно изучили инструментарий для сравнения мощностей множеств.
Кажется, что полученные результаты полезны, скорее, при работе с бесконечными множествами,
так как в ином случае достаточно сравнивать мощности как обычные числа.
Однако это неверно: изученные методы
%сравнения мощностей посредством построения определённых отображений
оказываются хорошим подспорьем и в задачах,
например, определения точного числа элементов в некотором множестве, или хотя бы в деле построения верхних и нижних оценок на это число.
Данными задачами занимается \defemph{комбинаторика}.

\begin{example}
    Чего больше: разбиений числа $ n $ на $ k $ слагаемых, или разбиений $ N $ на слагаемые, не превосходящие $ k $?

    %Легче всего её задать следующим правилом:
    %в некотором разбиении числа $ N $ на $ k $ слагаемых будем вычитать из каждого слагаемого по единице до тех пор, пока не появится первое нулевое слагаемое;
    %уберём его и продолжим процедуру, и так до полного опустошения суммы.

    %Тогда соответствующее разбиение $ N $ на слагаемые, не превосходящие $ k $, получается путём сложения чисел $ k $ столько раз, сколько было сделано шагов до появления первого нуля,
    %чисел $ k-1 $ столько раз, сколько было сделано последующих шагов до второго нуля, и так далее.
    %Постройте обратное отображение и проверьте, что, перед нами действительно биекция.

    Оказывается, в обоих случаях разбиений одно и то же число, ведь между данными множествами можно построить биекцию.
    Легче всего построить биекцию путём рассмотрения \textit{диаграмм Юнга} для разбиений.
    Пример таких диаграмм для двух разбиений числа $ N = 11 $: $ N = 5 + 3 + 2 + 2 $ и $ N = 4 + 4 + 2 + 1 + 1 $.
    \begin{center}
        \raisebox{-0.5\height}{
            \ydiagram{5,3,2,2}
        }
        \hspace{2\baselineskip}
        \raisebox{-0.5\height}{
            \ydiagram{4,4,2,1,1}
        }
    \end{center}
    Видно, что одно разбиение получается из другого транспонированием диаграммы.
    Также видно, что в первом случае имеется разбиение на $ k $ слагаемых, а во втором~--- на слагаемые, не превосходящие $ k $.
    Детали биекции предлагаю вам додумать самостоятельно.
\end{example}

Таким образом, получаем \emph{важный факт}: если требуется определить число элементов в некотором множестве,
можно попробовать сначала доказать, что элементов в нём столько же, сколько и в некотором другом множестве (возможно, с более понятным составом),
а потом уже пересчитать элементы второго множества.
Данное правило очевидным образом обобщается и на случаи, когда требуется оценить мощность множества сверху или снизу.

\begin{Exercise}[counter=SecExercise, label={exercise:combinatorics:subsets_and_binary_words}]
    \noindent
    Докажите, что двоичных последовательностей (слов) длины $ n $,
    в которых ровно $ k $ единиц, столько же,
    сколько и подмножеств размера $ k $ множества $ \{1, 2, \ldots, n\} $.
\end{Exercise}

\begin{Answer}
    \noindent
    Построим биекцию: каждому подмножеству сопоставим следующее слово длины $ n $:
    на $ k $-ой позиции стоит единица в том и только в том случае, если $ k $ входит в выбранное подмножество.
    Обратное отображение: по любому бинарному слову длины $ n $ согласно алгоритму выше однозначно восстанавливается подмножество.
    Равномощность доказана.
\end{Answer}



\subsection{Базовые комбинаторные задачи}
\label{subsec:combinatorics:basics}

Составим джентльменский набор базовых задач комбинаторики, к которым впоследствии можно будет сводить другие задачи посредством построения биекции.


\subsubsection{Правило суммы}

Начнём с задачи подсчёта числа элементов в множестве вида $ A = A_1 \cup A_2 \cup \ldots \cup A_n $.
Понятно, что если $ \forall i \neq j \;\, A_i \cap A_j = \varnothing $, то $ |A| = |A_1| + |A_1| + \ldots + |A_n| $.
Для строгого доказательства этого факта нам потребуется следующее утверждение:

\begin{statement}
    \label{statement:combinatorics:sum_indicators}
    Если $ A $ конечно, то $ \displaystyle |A| = \sum_{x \in U} \I_A(x) $.
\end{statement}

\begin{proof}
    $ |A| $ равно числу элементов, которые лежат в $ A $.
    Но заметим, что каждый такой элемент добавляет единицу в сумму $ \displaystyle \sum_{x \in U} \I_A(x) $,
    причём никакие другие элементы на сумму не влияют, так как $ \I_A $ принимает на них значение $ 0 $.
\end{proof}

\begin{corollary}
    Если $ A = A_1 \sqcup A_2 \sqcup \ldots \sqcup A_n $, то $ |A| = |A_1| + |A_2| + \ldots + |A_n| $.
\end{corollary}

\begin{proof}
    Раз $ \forall i \neq j \;\, A_i \cap A_j = \varnothing $, то $ \I_A(x) = \I_{A_1}(x) + \I_{A_2}(x) + \ldots + \I_{A_n}(x) $.
    Отсюда по утверждению \ref{statement:combinatorics:sum_indicators} получаем требуемое.
\end{proof}

Но что делать, если множества пересекаются?
Для случая двух множеств можно заметить, что $ \I_{A \cup B}(x) = \I_A(x) + \I_B(x) - \I_{A \cap B}(x) $.
Но тогда по утверждению \ref{statement:combinatorics:sum_indicators} имеем $ |A \cup B| = |A| + |B| - |A \cap B| $.
Данная формула называется \defemph{формулой включений-исключений}.
Она обобщается и на случай с $ n $ множествами.

\begin{lemma}
    Если $ \displaystyle A = \bigcup_{i=1}^n A_i $, то
    \[
        |A| = \sum_{k=1}^n (-1)^{k+1} \left( \sum_{S \in \mathcal{C}_k} \left| \bigcap_{A \in S} A \right| \right), \qquad
        \mathcal{C}_k = \begin{pmatrix} \{ A_1, A_2, \ldots, A_n \} \\ k \end{pmatrix}
    \]
\end{lemma}

\begin{proof}
    Можно доказать как по индукции, так и просто аккуратно записав $ \I_A(x) $ через обычные математические операции
    и воспользовавшись утверждением \ref{statement:combinatorics:sum_indicators}.
\end{proof}

\begin{Exercise}[counter=SecExercise]
    \noindent
    В группе $ 40 $ туристов.
    Из них $ 20 $ человек говорят по-английски, $ 15 $~--- по-французски, $ 11 $~--- по-испански.
    Английский и французский знают семь человек, английский и испанский~--- пятеро, французский и испанский~--- трое.
    Два туриста говорят на всех трёх языках.
    Сколько человек группы не знают ни одного из этих языков?
\end{Exercise}

\begin{Answer}
    \noindent
    По формуле включений-исключений имеем
    \[
        N = 40 - (20 + 15 + 11 - 7 - 5 - 3 + 2) = 40 - 33 = 7
    \]
\end{Answer}

\subsubsection{Правило произведения}

Рассмотрим задачу подсчёта числа возможных путей из вершины так называемого \defemph{дерева последовательного выбора} в любой его лист.

\begin{definition}
    \defemph{Деревом последовательного выбора} называется дерево, у которого можно выделить вершину (\defemph{корень}) так,
    чтобы все остальные вершины, расположенные на одном и том же расстоянии от выделенной, имели одинаковую степень.
\end{definition}

Пример дерева последовательного выбора можно видеть на рис. \ref{fig:combinatorics:decision_tree}.
На каждой вершине отмечена её степень за вычетом родительского ребра.
Слева от каждого уровня дерева выписано число путей с началом из корня и с концом на данном уровне.

\begin{statement}
    Расстояние от корня дерева последовательного выбора до любого из листьев одинаково.
\end{statement}

\begin{figure}[ht!]
    \center
    \begin{tikzpicture}[
        parent anchor=south,child anchor=north,
        level/.style={sibling distance=40mm/#1},
        >=latex,
        font=\sffamily,
        edge from parent/.style={draw, thick},
        no edge from this parent/.style={
            every child/.append style={
            edge from parent/.style={draw=none}}},
        level 3/.style={yshift=5cm},
        level 4/.style={level distance=5mm}
    ]
        \node (z){$ k_1 $}
        child
        {
            node (a) {$ k_2 $}
            child
            {
                node  (b) {$ k_3 $}
                child
                {
                    node (b1) {$\vdots$}[no edge from this parent]
                    child
                    {
                        node (b11) {$ 0 $}
                    }
                }
                child
                {
                    node (b2) {$\vdots$}[no edge from this parent]
                    child
                    {
                        node (b12) {$ 0 $}
                    }
                }
            }
            child
            {
                node (g) {$ k_3 $}
                child
                {
                    node (g1) {$\vdots$}[no edge from this parent]
                    child
                    {
                        node (g11) {$ 0 $}
                    }
                }
                child
                {
                    node (g2) {$\vdots$}[no edge from this parent]
                    child
                    {
                        node (g12) {$ 0 $}
                    }
                }
            }
        }
        child
        {
            node (d) {$ k_2 $}
            child
            {
                node  (e) {$ k_3 $}
                child
                {
                    node (e1) {$\vdots$}[no edge from this parent]
                    child
                    {
                        node (e11) {$ 0 $}
                    }
                }
                child
                {
                    node (e2) {$\vdots$}[no edge from this parent]
                    child
                    {
                        node (e12) {$ 0 $}
                    }
                }
            }
            child
            {
                node (f) {$ k_3 $}
                child
                {
                    node (f1) {$\vdots$}[no edge from this parent]
                    child
                    {
                        node (f11) {$ 0 $}
                    }
                }
                child
                {
                    node (f2) {$\vdots$}[no edge from this parent]
                    child
                    {
                        node (f12) {$ 0 $}
                    }
                }
            }
        };

        \node[left=6 of z]  (ln1) {$ 1 $}[no edge from this parent]
        child
        {
            node (ln2) {$ k_1 $}[no edge from this parent]
            child
            {
                node (ln3) {$ k_1 \cdot k_2 $}[no edge from this parent]
                child
                {
                    node (ln4) {}[no edge from this parent]
                    child
                    {
                        node (ln5) {$ k_1 \cdot k_2 \cdot \ldots \cdot k_h $}
                    }
                }
            }
        };

        %\path (b12.north east) -- (g11.north west) node [midway] {$\cdots$};
        %\path (e12.north east) -- (f11.north west) node [midway] {$\cdots$};

        \coordinate (cd1) at ($(f12)+(1,0)$);
        \coordinate (nb1) at ($(g12)!.5!(e11)$);

        \draw[thick,<->,]
            (cd1) -- (cd1|-z.east) node [near start, fill=white] {$ h $};

        \draw[dashed,thick,->]
            ($(z.west)+(-1em,0)$) -- (ln1);
        \draw[dashed,thick,->]
            ($(a.west)+(-1em,0)$) -- (ln2.east);
        \draw[dashed,thick,->]
            ($(b.west)+(-1em,0)$) -- (ln3);
        \draw[dashed,thick,->]
            ($(b11.west)+(-1em,0)$) -- (ln5);

        \draw[thick,decorate,decoration={brace,amplitude=10pt,mirror},->,-{latex[flex=1pt]}] (b11.south west) -- (f12.south east);
    \end{tikzpicture}

    \label{fig:combinatorics:decision_tree}
    \caption{дерево последовательного выбора (схематично)}
\end{figure}

\begin{definition}
    Расстояние от корня дерева последовательного выбора до любого из листьев называется \defemph{высотой дерева} (или \defemph{числом выборов}) и обозначается $ h $.
    Степень каждой вершины (на расстоянии $ m - 1 $ от корня) за вычетом родительского ребра называется \defemph{мощностью выбора (на шаге $ m $)} и обозначается $ k_m $.
    Путь из корня в лист называется \defemph{решающим путём}.
\end{definition}

\begin{statement}
    Пусть дерево последовательного выбора характеризуется высотой $ h $ и мощностями выборов $ k_1, k_2, \ldots, k_h $.
    Тогда число путей из корня в любой лист равно $ k_1 \cdot k_2 \cdot \ldots \cdot k_h $.
\end{statement}

\begin{proof}
    По индукции, проведите сами.
\end{proof}


\begin{Exercise}[counter=SecExercise, label={exercise:combinatorics:stairs}]
    \noindent
    Лестница состоит из 13 ступенек, не считая верхней и нижней площадок.
    Спускаясь, можно перепрыгивать через некоторые ступеньки (можно даже через все).
    Сколькими способами можно спуститься по этой лестнице?
\end{Exercise}

\begin{Answer}
    \noindent
    На каждую ступеньку можно либо наступить, либо не наступить во время спуска.
    Тогда по правилу произведения искомое число способов~--- $ 2^{13} = 8192 $.
\end{Answer}


\begin{Exercise}[counter=SecExercise, label={exercise:combinatorics:students}]
    \noindent
    На одном этаже семёрки живёт $ 100 $ человек.
    Среди них требуется выбрать двух ответственных за южную и северную кухни, одного ответственного за умывальники и санузел, а также его заместителя.
    Сколькими способами это можно сделать?
\end{Exercise}

\begin{Answer}
    \noindent
    Можно построить биекцию из множества способов выбрать ответственных в множество путей от корня к листьям в дереве последовательного выбора высоты $ h = 4 $
    и с мощностями выборов $ k_1 = 100 $, $ k_2 = 100 - 1 $, $ k_3 = 100 - 2 $ и $ k_4 = 100 - 3 $.
    Действительно, движение от корня к листьям пусть будет соответствовать последовательному выбору ответственных.
    Тогда переход по первому ребру соответствует выбору ответственного за южную кухную из $ 100 $ студентов, переход по второму~---
    выбору ответственного за северную кухню из оставшихся $ 99 $ студентов и так далее.
    Тогда всего способов~--- $ 100 \cdot 99 \cdot 98 \cdot 97 $.
\end{Answer}

Мы рассмотрели очень частный случай, когда мощность очередного выбора на единицу меньше мощности предыдущего.
Это не всегда так, и ниже будет рассмотрены две задачи другого типа.
Однако и такой специальный случай встречается настолько часто, что для обозначения соответствующего ответа ввели специальное число:
\[
    A_n^k = n \cdot (n-1) \cdot \ldots \cdot (n - k) = \frac{n!}{(n-k)!}
\]
Это число называется \defemph{числом расстановок}.
Связь названия и класса задач довольно очевидна:
действительно, в задаче \ref{exercise:combinatorics:students} мы <<расставили>> $ k = 4 $ из $ n = 100 $ студентов по $ k $ должностям.

\begin{Exercise}[counter=SecExercise]
    \noindent
    Сколькими способами можно выбрать два числа разной чётности из множеств $ \{1, \ldots, 4\} $ и $ \{ 11, \ldots, 16 \} $?
\end{Exercise}

\begin{Answer}
    \noindent
    На первом шаге можно четырьмя способами выбрать число из первого диапазона,
    на втором шаге мы будем выбирать из $ 6 / 2 = 3 $ элементов (так как чётность фиксирована).
    В итоге имеем $ 4 \cdot 3 = 12 $ способов.
\end{Answer}

\begin{Exercise}[counter=SecExercise]
    \noindent
    Сколькими способами можно выбрать два числа из диапазона $ \{ 1, \ldots, 9 \} $, дающие разный остаток при делении на три?
\end{Exercise}

По аналогии с прошлой задачей в голову сразу приходит ответ $ 9 \cdot (9 \cdot 2 / 3) = 54 $.
Однако если честно пересчитать все варианты, получится число в два раза меньшее.
В чём же проблема?

Дело в том, что в предыдущей задаче на каждом шаге числа выбирались из разных множеств,
что позволяло однозначно сопоставить каждому решающему пути число из первого множества и число из второго.
В случае текущей задачи уже двум решающим путям будет соответствовать одна и та же пара чисел, просто выбранная в разном порядке (например, $ (1, 3) $ и $ (3, 1) $).
Понятно, что учёт возможной перемены местами выбранных чисел как раз и уменьшает ответ в два раза, но как это отражается в построении биекции?

Если оставаться в рамках модели деревьев последовательного выбора, то данная проблема обычно решается введением дополнительных ограничений,
позволяющих зафиксировать порядок получения результатов выбора.
Например, в случае нашей задачи можно потребовать, чтобы второй выбор совершался не среди двух оставшихся классов,
а только среди того класса, что соответствует следующему остатку по модулю три.
То есть, например, если мы выбрали число с остатком $ 0 $, то второе число обязано иметь остаток $ 1 $,
если выбрали число с остатком $ 1 $, то второе~--- с остатком $ 2 $ и так далее.
То, что построена биекция между множеством из задачи и решающими путями в дереве с мощностями выборов $ 9 $ и $ 3 $, проверьте сами.

\subsubsection{Подсчёт подмножеств}

Из указанной выше проблемы понятно, что одним правилом произведения сыт не будешь.
Далеко не всегда ясно, какое ограничение надо ввести, чтобы получить биекцию.
Пойдём дальше и рассмотрим другую базовую задачу, к которой уже будет легко свести проблемное упражнение из предыдущего пункта.

Пусть $ A $~--- конечное множество мощности $ |A| = n $.
Тогда чему равно
\[
    \left| \binom{A}{k} \right|,
\]
где $ k \in \{1, \ldots, n\} $?
Ответ уже давался в замечании \ref{remark:graphs:subsets_cardinality}, настало время его строго обосновать.

\begin{proof}[Доказательство замечания \ref{remark:graphs:subsets_cardinality}]
    Задача подсчёта числа расстановок $ n $ объектов по $ k $ позициям уже решена:
    их $ A_n^k $.
    Осталось понять, чем это отличается от числа подмножеств мощности $ k $.

    Заметим, что каждой последовательности элементов $ A $ длины $ k $ соответствует подмножество $ A $ мощности $ k $.
    Но каждому подмножеству $ A $ мощности $ k $ соответствует $ A_k^k = k! $ последовательностей элементов этого множества.
    Но тогда имеем, что мощность множества подмножеств $ A $ мощности $ k $ равна
    \begin{equation}
        \label{eq:combinatorics:binomial}
        \frac{A_n^k}{k!} = \frac{n!}{k!(n-k)!} \defeq \binom{n}{k} \defeq C_n^k
    \end{equation}
    Полученное число называется \defemph{числом сочетаний}.
\end{proof}

\vspace{\baselineskip}

\begin{Answer}
    \noindent
    Выберем два разных класса из трёх классов чисел по остатку по модулю три.
    Из каждого класса затем можно тремя способами выбрать по экземпляру.
    В итоге имеем $ C_3^2 \cdot 3 \cdot 3 = 27 $ вариантов.
\end{Answer}


\begin{Exercise}[counter=SecExercise, label={exercise:combinatorics:triangles}]
    \noindent
    На плоскости отмечено $ 10 $ точек так,
    что никакие три из них не лежат на одной прямой.
    Сколько существует треугольников с вершинами в этих точках?
\end{Exercise}

\begin{Answer}
    \noindent
    Так как никакие три точки не лежат на одной прямой,
    любая тройка точек образует треугольник.
    Число способов ыбрать три точки (без учёта порядка на них)~--- $ C_{10}^3 = \frac{10!}{7! 3!} = 120 $.
    Это и будет искомым числом треугольников.
\end{Answer}


\begin{Exercise}[counter=SecExercise, label={exercise:combinatorics:odd_even_monotonic}]
    \noindent
    Сколькими способами можно выписать в ряд цифры от $ 0 $ до $ 9 $ так,
    чтобы четные цифры шли в порядке возрастания,
    а нечетные~--- в порядке убывания?
\end{Exercise}

\begin{Answer}
    \noindent
    Всё число однозначно определяется положением чётных цифр.
    Таким образом, искомое число способов~--- $ C_{10}^5 $.
\end{Answer}




\subsection{Комбинированные задачи}
\label{subsec:combinatorics:combined}

Разберём некоторые другие примеры, которые разбиваются на несколько базовых комбинаторных задач.

\begin{Exercise}[counter=SecExercise, label={exercise:combinatorics:flowers}]
    \noindent
    Есть $ 3 $ гвоздики, $ 4 $ розы и $ 5 $ тюльпанов.
    \begin{enumerate}[label=\textbf{\alph*)}]
        \item Сколькими способами можно составить букет из цветов одного вида?
        \item Сколькими способами из них можно составить букет, в котором нечётное количество цветов каждого вида?
        \item Сколькими способами можно составить букет, используя любые из имеющихся цветов?
    \end{enumerate}
    (Цветы одного сорта считаем одинаковыми,
     количество цветов в букете не ограничено, но не равно $ 0 $.)
\end{Exercise}

\begin{Answer}
    \noindent
    \begin{enumerate}[label=\textbf{\alph*)}]
        \item
            Число способов составить букет из гвоздик~--- $ 3 $ ($ 1 $, $ 2 $ или $ 3 $ гвоздики).
            Из роз~--- $ 4 $.
            Из тюльпанов~--- $ 5 $.
            По правилу суммы, ответ~--- $ 3 + 4 + 5 = 12 $.
        \item
            Можно взять $ 1 $ или $ 3 $ гвоздики, $ 1 $ или $ 3 $ розы, $ 1 $, $ 3 $ или $ 5 $ тюльпанов.
            По правилу произведения, ответ~--- $ 2 \cdot 2 \cdot 3 = 12 $.
        \item
            Можно выбрать от $ 0 $ до $ 3 $ гвоздик, от $ 0 $ до $ 4 $ роз, от $ 0 $ до $ 5 $ тюльпанов.
            Но нельзя не брать ничего, поэтому букет <<из нуля цветов>> отбросим.
            Ответ~--- $ 4 \cdot 5 \cdot 6 - 1 = 120 - 1 = 119 $.
    \end{enumerate}
\end{Answer}


\begin{Exercise}[counter=SecExercise]
    \noindent
    В группе студентов есть один, который знает С++, Java, Python, Haskell.
    Каждые три из этих языков знают два студента.
    Каждые два~--- $ 6 $ студентов.
    Каждый из этих языков знают по $ 15 $ студентов.
    Каково наименьшее количество студентов в такой группе?
\end{Exercise}

\begin{Answer}
    \noindent
    Наименьшее число достигается тогда и только тогда, когда нет студентов, не знающих ни один из языков.
    Тогда по формуле включений-исключений имеем
    \[
        N = C_4^1 \cdot 15 - C_4^2 \cdot 6 + C_4^3 \cdot 2 - C_4^4 \cdot 1 = 4 \cdot 15 - 6 \cdot 6 + 4 \cdot 2 - 1 \cdot 1 = 31
    \]
\end{Answer}


\begin{Exercise}[counter=SecExercise, label={exercise:combinatorics:monotonic_digits}]
    \noindent
    Сколько существует $ 9 $-значных чисел,
    цифры которых расположены в порядке убывания (то есть каждая следующая меньше предыдущей)?
\end{Exercise}

\begin{Answer}
    \noindent
    Заметим, что последняя цифра в таком числе обязательно равна $ 0 $ или $ 1 $
    (иначе числа цифр десятичной системы счисления недостаточно для обеспечения строгого убывания).
    Если последнияя цифра~--- $ 1 $,
    то число определеяется однозначно~--- $ 987654321 $.
    Если последняя цифра~--- $ 0 $,
    то имеем число $ 876543210 $,
    а также $ 8 $ чисел, в которых есть пара соседних цифр, отличающихся друг от друга на $ 2 $
    (имеется $ 9 - 1 = 8 $ способов выбрать, где будет стоять данная пара).
    В итоге имеем $ 1 + 1 + 8 = 10 $ чисел.
\end{Answer}


\begin{Exercise}[counter=SecExercise, label={exercise:combinatorics:subsets_decomposition}]
    \noindent
    Чего больше, разбиений $ 20 $-элементного множества на $ 6 $ (непустых) подмножеств или его подмножеств размера $ 5 $?
\end{Exercise}

\begin{Answer}
    \noindent
    Разбиений $ 20 $-элементного множества на $ 6 $ непустых подмножеств~--- $ 6^{20} / 6! $
    (каждому элементу сопоставляется номер множества, в которое он попадает;
    далее эти номера можно $ 6! $ способами переставить, так как множества неотличимы).
    Число подмножеств размера $ 5 $~--- $ C_{20}^5 $.
    Таким образом, первое число больше:
    \[
        C_{20}^5 = \frac{20!}{5! \cdot 15!} = \frac{20 \cdot 19 \cdot 18 \cdot 17 \cdot 16}{5!} < 6 \cdot \frac{6^2 \cdot 6^2 \cdot 6^2 \cdot 6^2 \cdot 6^2}{6!} = \frac{6^{11}}{6!} < \frac{6^{20}}{6!}
    \]
\end{Answer}


\begin{Exercise}[counter=SecExercise]
    \noindent
    Сколькими способами можно выбрать два подмножества $ A $ и $ B $ множества $ \{1, \ldots, 10 \} $ так, чтобы $ |A| = 2 $, $ |B| = 5 $ и $ A \subseteq B $?
\end{Exercise}

\begin{Answer}
    \noindent
    Выберем $ C_{10}^2 $ способами множество $ A $.
    Далее выберем $ C_8^3 $ способами множество $ B \setminus A $.
    В итоге, $ N = C_{10}^2 \cdot C_8^3 $.
\end{Answer}


\begin{Exercise}[counter=SecExercise]
    \noindent
    Сколькими способами можно разбить $ \{ 1, \ldots, 10 \} $ на два непустых подмножества, а затем упорядочить элементы в одном из блоков любым образом?
\end{Exercise}

\begin{Answer}
    \noindent
    Если размер любого блока фиксирован и равен $ k $, то число способов~--- $ N_k = C_{10}^k \cdot (k! + (10 - k)!) = A_{10}^k + A_{10}^{10-k} $.
    Осталось просуммировать по всем возможным значениям размера меньшего блока: $ k \in \{1, \ldots, 5 \} $.
    \[
        N = \sum_{k=1}^5 N_k = \sum_{k=1}^5 A_{10}^k + A_{10}^{10-k}
    \]
\end{Answer}


\begin{Exercise}[counter=SecExercise, label={exercise:combinatorics:at_least_one_vowel}]
    \noindent
    Найдите среди четырёхбуквенных слов (не обязательно осмысленных), составленных из букв русского алфавита,
    долю слов, имеющих хотя бы одну гласную.
    \newline
    \textit{Всего букв $ 33 $, из них $ 10 $~--- гласные.}
\end{Exercise}

\begin{Answer}
    \noindent
    Число четырёхбуквенных слов, не имеющих ни одной гласной буквы~--- $ (33 - 10)^4 $.
    Всего четырёхбуквенных слов~--- $ 33^4 $.
    Тогда искомая доля:
    \[
        p = 1 - \frac{(33 - 10)^4}{33^4} \approx 0.764
    \]
\end{Answer}


\begin{Exercise}[counter=SecExercise]
    \noindent
    В русском алфавите $ 33 $ буквы, $ 10 $ из них~--- гласные.
    Сколько всего можно составить слов длины $ 10 $, в которых есть три различные гласные, а согласные идут в строго возрастающем алфавитном порядке?
\end{Exercise}

\begin{Answer}
    \noindent
    Для начала $ C_{10}^3 $ способами выберем три различные гласные для нашего слова.
    Далее $ C_{23}^7 $ способами выберем не гласные буквы
    (так как в слове они должны идти в строго возрастающем алфавитном порядке, они все должны быть различные).
    Выбранные буквы расставим $ 10! $ способами.
    Но порядок согласных фиксирован, а потому на каждое подходящее слово приходится еще $ 7! - 1 $ неподходящих.
    Тогда ответ~---
    \[
        N = \frac{C_{10}^3 \cdot C_{23}^7 \cdot 10!}{7!} = C_{10}^3 \cdot C_{23}^7 \cdot A_{10}^3
    \]
\end{Answer}


\begin{Exercise}[counter=SecExercise, label={exercise:combinatorics:queue}]
    \noindent
    Десять человек случайно выстроились в очередь.
    Найдите долю случаев, когда
    \begin{enumerate}[label=\textbf{\alph*)}]
        \item Иванов, Петров и Сидоров стоят подряд (в произвольном порядке);
        \item Иванов стоит раньше Петрова;
        \item Иванов и Петров не стоят друг за другом?
    \end{enumerate}
\end{Exercise}

\begin{Answer}
    \noindent
    Всего возможных вариантов построить очередь из десяти человек~--- $ 10! $.
    \begin{enumerate}[label=\textbf{\alph*)}]
        \item
            Мысленно объединим Иванова, Петрова и Сидорова в группу.
            Число способов расставить их внутри своей группы~--- $ 3! $.
            Число способов расставить всех остальных студентов и данную группу~--- $ 8! $.
            Значит, искомая доля:
            \[
                p = \frac{8! 3!}{10!} = \frac{2 \cdot 3}{9 \cdot 10} = \frac{1}{15}
            \]
        \item
            Из любой очереди, в которой Иванов стоит раньше Петрова,
            путём перестановки местами Петрова и Иванова
            можно получить очередь, в которой Петров стоит раньше Иванова.
            Имеем биекцию между указанными двумя множествами очередей.
            Так как эти множества не пересекаются и при объединении дают множество всех возможных очередей,
            имеем искомую долю~--- $ p = 1/2 $.
        \item
            Перестановок, где Иванов и Петров стоят друг за другом~--- $ 9! \cdot 2! $
            (см. первый пункт задачи).
            Искомая доля тогда~---
            \[
                p = 1 - \frac{9! 2!}{10!} = \frac{4}{5}
            \]
    \end{enumerate}
\end{Answer}


\begin{Exercise}[counter=SecExercise, label={exercise:combinatorics:num_of_pairs}]
    \noindent
    Сколькими способами можно образовать 6 пар из 12 человек?
\end{Exercise}

\begin{Answer}
    \noindent
    Пусть пары пронумерованы (отличимы друг от друга).
    Число способов получить первую пару~--- $ C_{12}^2 $.
    Вторую~--- $ C_{10}^2 $; и так далее.
    По правилу произведения получаем $ C_{12}^2 \cdot C_{10}^2 \cdot C_8^2 \cdot C_6^2 \cdot C_4^2 \cdot C_2^2 $.
    Теперь уберём нумерацию пар.
    Это уменьшит число вариантов в $ 6! $ раз.
    В итоге имеем ответ
    \[
        N = \frac{C_{12}^2 \cdot C_{10}^2 \cdot C_8^2 \cdot C_6^2 \cdot C_4^2 \cdot C_2^2}{6!} = \frac{12!}{6! \cdot (2!)^6} = 10395
    \]

    Есть и альтернативный способ решения: $ 12! $ способами расставить студентов,
    фиксированную растсановку разбить на шесть идущих подряд пар,
    убрать порядок внутри каждой пары (делить на $ (2!)^6 $) и на самих парах (делить на $ 6! $).
\end{Answer}


\begin{Exercise}[counter=SecExercise, label={exercise:combinatorics:checkers}]
    \noindent
    Сколькими способами можно расставить $ 12 $ белых и $ 12 $ черных шашек на черных полях шахматной доски?
\end{Exercise}

\begin{Answer}
    \noindent
    Число способов расставить белые шашки~--- $ C_{8 \cdot 8 / 2}^{12} = C_{32}^{12} $.
    Для каждой такой расстановки число способов расставить чёрные шашки~--- $ C_{32 - 12}^{12} = C_{20}^{12} $.
    По правилу произведения имеем ответ:
    \[
        N = C_{32}^{12} \cdot C_{20}^{12} = \frac{32!}{(12!)^2 8!}
    \]
\end{Answer}



\subsection{Биномиальные коэффициенты}
\label{subsec:combinatorics:binomial}

Число сочетаний также называется \defemph{биномиальным коэффициентом}.
Это вызвано его появлением в следующей задаче:

\begin{Exercise}[counter=SecExercise]
    \noindent
    Найдите коэффициент при $ a^k b^{n-k} $ после раскрытия скобок в выражении $ (a + b)^n $. % (\defemph{бином Ньютона}).
\end{Exercise}

\begin{Answer}
    \noindent
    Перепишем выражение в виде длинного произведения:
    \[
        (a + b)^n = \underbrace{(a + b) \cdot (a + b) \cdot \ldots \cdot (a + b)}_{n \; \textnormal{раз}}
    \]
    Любое слагаемое вида $ a^k b^{n - k} $ после раскрытия получается только при выборе из некоторых $ k $ скобок числа $ a $,
    а из оставшихся $ n - k $ скобок~--- числа $ b $.
    То есть таких слагаемых будет ровно столько, сколькими способами можно выбрать из $ n $ указанных скобок некоторые $ k $,
    то есть $ C_n^k $.
    Таким образом,
\end{Answer}

\begin{statement}
    \label{statement:combinatorics:Newton_binom}
    \[
        (a + b)^n = \sum_{k = 0}^n C_n^k \cdot a^k b^{n-k} \qquad \text{\defemph{(Бином Ньютона)}}
    \]
\end{statement}

Из связи числа сочетаний с биномом Ньютона очевидным образом следует, что $ C_n^k = C_n^{n-k} $,
хотя это было понятно и из формулы \eqref{eq:combinatorics:binomial}.

\begin{remark}
    \label{remark:combinatorics:sum_binomial}
    $ \displaystyle \sum_{k=0}^n C_n^k = 2^n $; \>
    $ \displaystyle \sum_{k=0}^n (-1)^k C_n^k = 0 $
\end{remark}

\begin{proof}
    $ \displaystyle 2^n = (1 + 1)^n = \sum_{k=0}^n C_n^k \cdot 1^k 1^{n-k} $; \>
    $ \displaystyle 0   = (-1 + 1)^n = \sum_{k=0}^n C_n^k \cdot (-1)^k 1^{n-k} $.
\end{proof}

Есть и много других подобных замечанию \ref{remark:combinatorics:sum_binomial} фактов касательно суммы биномиальных коэффициентов.
Разберём задачу на эту тему:

\begin{Exercise}[counter=SecExercise]
    \noindent
    Докажите справедливость формул (желательно найти комбинаторное доказательство):
    \begin{enumerate}[leftmargin=*]
        %\item $ \displaystyle \sum_{j=0}^k C_r^j C_s^{k-j} = C_{r+s}^k $;
        %\inlineitem $ \displaystyle \sum_{j=0}^n C_j^k = C_{n+1}^{k+1} $;
        %\inlineitem $ \displaystyle \sum_{j=0}^k C_{n+j}^j = C_{n+k+1}^k $;
        \item $ \displaystyle \sum_{j=0}^k \binom{r}{j} \binom{s}{k-j} = \binom{r + s}{k} $;
        \inlineitem $ \displaystyle \sum_{j=0}^n \binom{j}{k} = \binom{n+1}{k+1} $;
        \inlineitem $ \displaystyle \sum_{j=0}^k \binom{n + j}{j} = \binom{n + k + 1}{k} $;
    \end{enumerate}
\end{Exercise}

\begin{Answer}
    \noindent
    Под комбинаторным доказательством понимается доказательство, использующее построение биекции
    между некоторыми двумя множествами, мощность первого из которых равна левой части, а второго~--- правой.
    Такие доказательства обычно красивее и понятнее доказательств сугубо подсчётных
    (например, использующих формулу \eqref{eq:combinatorics:binomial}).
    \begin{enumerate}
        \item
            $ \displaystyle \sum_{j=0}^k \binom{r}{j} \binom{s}{k-j} = \binom{r + s}{k} $.

            В правой части записано число способов выбрать из $ (r + s) $-элементного множества подмножество размера $ k $.
            Заметим, что и в левой части записано то же число.

            Действительно, разобьём условно исходное множество на два подмножества размера $ r $ и $ s $.
            Пусть после выбора подмножества размера $ k $ в первом подмножестве оказалось $ j $ элементов (во втором тогда $ k - j $).
            Всего имеем $ C_r^j \cdot C_s^{k-j} $ способов получить подмножество, удовлетворяющее указанному свойству.
            Просуммировав по всем возможным $ j $ (от $ 0 $ до $ k $), покроем все возможные исходы, причём без повторений.
            Что и требовалось доказать.
        \item
            $ \displaystyle \sum_{j=0}^n \binom{j}{k} = \binom{n+1}{k+1} $.

            В правой части записано число способов выбрать из $ (n+1) $-элементного множества подмножество размера $ k - 1 $.
            Заметим, что и в левой части записано то же число.

            Действительно, упорядочим произвольным образом исходное множество.
            Пусть после выбора подмножества размера $ k + 1 $ оказалось, что наибольший из индексов его элементов равен $ j + 1 $.
            Всего имеем $ C_k^j $ способов получить подмножество, удовлетворяющее указанному свойству:
            оставшиеся $ k $ элементов выбираются среди первых $ j $ исходного множества.
            Просуммировав по всем возможным $ j $ (от $ 0 $ до $ n $), покроем все возможные исходы, причём без повторений.
            Что и требовалось доказать.

        \item
            $ \displaystyle \sum_{j=0}^k \binom{n + j}{j} = \binom{n + k + 1}{k} $.

            Задача похожа на предыдущую.
            Воспользовавшись симметричностью биномиальных коэффициентов, получаем
            \[
                \sum_{j=0}^k \binom{n + j}{n} = \binom{n + k + 1}{n + 1}
            \]
            В правой части записано число способов выбрать из $ (n+k+1) $-элементного множества подмножество размера $ n + 1 $.
            Заметим, что и в левой части записано то же число.

            Действительно, аналогично предыдущей задаче, упорядочим произвольным образом исходное множество.
            Пусть после выбора подмножества размера $ n + 1 $ оказалось, что наибольший из индексов его элементов равен $ n + j + 1 $.
            Всего имеем $ C_{n+j}^j $ способов получить подмножество, удовлетворяющее указанному свойству.
            Просуммировав по всем возможным $ j $ (от $ 0 $ до $ k $), покроем все возможные исходы, причём без повторений.
            Что и требовалось доказать.
    \end{enumerate}
\end{Answer}

\begin{Exercise}[counter=SecExercise]
    \noindent
    Сколькими способами среди $ n $ солдат можно выбрать командира и набрать ему в подчинение отряд произвольного размера?
\end{Exercise}

\begin{Answer}
    \noindent
    С одной стороны, можно $ n $ способами выбрать командира и каждого оставшегося солдата либо взять в отряд, либо нет.
    По правилу произведения имеем следующее число вариантов: $ n \cdot 2^{n-1} $.
    С другой стороны, можно для всех возможных $ k $ сначала $ C_n^k $ способами выбрать отряд размера $ k $,
    а затем в нём $ k $ способами выбрать командира.

    В итоге имеем два тождественно равных ответа:
    \[
        n \cdot 2^{n-1} = \sum_{k=1}^n k \cdot C_n^k
    \]
\end{Answer}

У полученного в предыдущей задаче тождества есть еще одно красивое доказательство,
которое мы получим ближе к концу курса.
А пока докажем полезное рекуррентное соотношение на биномиальные коэффициенты,
которое полезно при построении \emph{треугольником Паскаля}.

\begin{statement}
    $ C_n^k = C_{n-1}^k + C_{n-1}^{k-1} $.
\end{statement}

\begin{proof}
    Рассмотрим задачу выбора подмножества мощности $ k $ из множества мощности $ n $.
    Зафиксируем в исходном множестве некоторый элемент.
    Тогда при выборе $ k $-элементного подмножества мы можем либо включить данный элемент, либо не включить.
    В первом случае имеем $ C_{n-1}^{k-1} $ вариантов выбора, а во втором~--- $ C_{n-1}^k $.
\end{proof}

Рассмотрим еще одну классическую задачу, в которой возникают биномиальные коэффициенты.

\begin{Exercise}[counter=SecExercise]
    \noindent
    Найдите число решений уравнения $ x_1 + x_2 + \ldots + x_k = n $ в неотрицательных целых числах.
\end{Exercise}

\begin{Answer}
    \noindent
    Решим задачу \emph{методом точек и перегородок}.
    Заметим, что число решений равно числу способов разделить $ n $ неразличимых точек $ (k - 1) $-ой неразличимой перегородкой.
    Действительно, будем интерпретировать число точек в каждой секции как значение соответствующей переменной $ x_i $.
    Таким образом, имеем биекцию.

    Число способов так разделить $ n $ точек можно найти следующим образом:
    <<свалим>> в общую кучу точки и перегородки, перемешаем их $ (n + k - 1)! $ способами,
    а затем разделим на $ n! $ и $ (k-1)! $, учтя тем самым неразличимость точек и перегородок между собой.
    Итоговый ответ:
    \[
        N_{\textnormal{решений}} = \binom{n + k - 1}{k - 1} \qquad \textnormal{\defemph{(формула Муавра)}}
    \]
\end{Answer}


\subsection{Мультиномиальные коэффициенты}
\label{subsec:combinatorics:multinomial}

Утверждение \ref{statement:combinatorics:Newton_binom} сформулировано только для случая возведения в $ n $-ую степень суммы \emph{двух} слагаемых.
Получим общую формулу:

\begin{statement}
    \label{statement:combinatorics:multinom}
    \[
        (x_1 + x_2 + \ldots + x_m)^n = \sum_{k_1 + k_2 + \ldots + k_m = n} \binom{n}{k_1, \; k_2, \; \ldots, k_m} x_1^{k_1} x_2^{k_2} \ldots x_m^{k_m},
    \]
    где
    \[
        \binom{n}{k_1, \; k_2, \; \ldots, k_m} = \binom{n}{k_1} \cdot \binom{n - k_1}{k_2} \cdot \ldots \cdot \binom{n - k_1 - \ldots - k_m}{k_m} =
        \frac{n!}{k_1! k_2! \ldots k_m!}
    \]
    --- \defemph{мультиномиальный коэффициент}.
\end{statement}

\begin{proof}
    Аналогично доказательству утверждения \ref{statement:combinatorics:Newton_binom}:
    для получения монома вида $ x_1^{k_1} x_2^{k_2} \ldots x_m^{k_m} $ при раскрытии скобок
    мы $ C_n^{k_1} $ способами выбираем скобки, из которых берём $ x_1 $, $ C_{n-k_1}^{k_2} $ способами~--- скобки, из которых берём $ x_2 $, и так далее.
\end{proof}

\begin{Exercise}[counter=SecExercise]
    \noindent
    Сколько различных слов (не обязательно осмысленных) можно получить, переставляя буквы в словах
    \begin{enumerate}[label=\textbf{\alph*)}]
        \item <<КОМПЬЮТЕР>>;
        \inlineitem <<ЛИНИЯ>>;
        \inlineitem <<ПАРАБОЛА>>;
        \item <<ОБОРОНОСПОСОБНОСТЬ>>?
    \end{enumerate}
\end{Exercise}

\begin{Answer}
    \noindent
    \begin{enumerate}[label=\textbf{\alph*)}]
        \item
            Всё девять букв различны, поэтому достаточно переставить их произвольным образом: $ N = 9! $.
        \item
            Среди пяти букв повторяется только <<И>>, причём два раза.
            Сначала $ 5! $ способами расставим буквы из предположения, что они все различимы,
            а затем разделим на $ 2! $, учтя тем самым, что две буквы неразличимы, а потому их перестановка ничего не изменит: $ N = 5! / 2! $.
        \item
            Среди восьми букв повторяется только <<А>>, причём три раза.
            Аналогично, $ N = 8! / 3! $.
        \item
            Среди восемнадцати букв <<О>> повторяется семь раз, буква <<С>>~--- три, <<Б>> и <<Н>>~---  два, остальные буквы встречаются по одному разу.
            Тогда $ \displaystyle N = \frac{18!}{7! 3! 2! 2!} $.
    \end{enumerate}
\end{Answer}

\begin{remark}
    Пусть имеется $ m $ букв в количествах $ k_1 $, $ k_2 $, \ldots, $ k_m $ соответственно ($ k_1 + k_2 + \ldots + k_m = n $).
    Тогда число различных (не обязательно осмысленных) слов, которые можно из данных букв составить~--- $ \binom{n}{k_1, \; k_2, \; \ldots, \; k_m} $.
\end{remark}
