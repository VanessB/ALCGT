\silentsection{Задачи из канонического задания} \label{sec:exercises}

В этом разделе приведён список указателей на задачи из канонического задания.
Не все решения добавлены в данный конспект.

\subsection*{Алгебра логики: введение}

\subsection*{Множества и логика}

\subsection*{Математические определения, утверждения и доказательства}

\subsection*{Графы I: неориентированные графы}

\begin{enumerate}[label=\textbf{№\arabic*}:]
    \item Определение \ref{def:graphs:named_graphs} и утверждение \ref{remark:graphs:subsets_cardinality}.
    \item Задача \ref{exercise:graphs:inconsistent_degrees}.
    \item Задача \ref{exercise:graphs:sum_of_degs_less}.
    \item Задача \ref{exercise:graphs:clique_independent_set}.
    \item Задача \ref{exercise:graphs:connected_graphs_union}.
    \item Задача \ref{exercise:graphs:graph_or_complement_connected}.
    \item Задача \ref{exercise:graphs:max_edges_in_disconnected_graph}.
    \item Задача \ref{exercise:graphs:union_of_cycles}.
    \item Задача \ref{exercise:graphs:BFS_sort}.
    \item Задача \ref{exercise:graphs:disjoined}.
\end{enumerate}

\subsection*{Графы II: деревья и раскраски}

\begin{enumerate}[label=\textbf{№\arabic*}:]
    \item Задача \ref{exercise:graphs:star_tree}.
    \item Задача \ref{exercise:graphs:small_tree_big_deg}.
    \item Задача \ref{exercise:graphs:more_then_half_leaves}.
    \item Утверждение \ref{statement:graphs:connected_has_spanning_tree}.
    \item Задача \ref{exercise:graphs:long_walk} и \ref{exercise:graphs:disjoined}.
    \item Задача \ref{exercise:graphs:walks_and_coloring}.
    \item Следствие \ref{corollary:graphs:tree_2_coloring} и задача \ref{exercise:graphs:tree_2_coloring}.
    \item Задача \ref{exercise:graphs:clique_max_colors}.
    \item Задача \ref{exercise:graphs:can_delete_one_vertex}.
    \item Задача \ref{exercise:graphs:rad_and_diam}.
\end{enumerate}

\subsection*{Двудольные графы, паросочетания и функции}

\begin{enumerate}[label=\textbf{№\arabic*}:]
    \item Задача \ref{exercise:functions:sets_examples}.
    \item Задача \ref{exercise:functions:max_prime_devisor}.
    \item Задача \ref{exercise:functions:set_operations}.
    \item Задача \ref{exercise:functions:map_unmap}.
    \item Определение \ref{definition:functions:cardinaluty_compare} и утверждение \ref{statement:functions:cardinality_compare_equivalence}.
    \item Задача \ref{exercise:functions:infinite_surjection}.
\end{enumerate}

\subsection*{Комбинаторика I. Правила суммы и произведения}

\begin{enumerate}[label=\textbf{№\arabic*}:]
    \item Задача \ref{exercise:combinatorics:flowers}.
    \item Задача \ref{exercise:combinatorics:triangles}.
    \item Задача \ref{exercise:combinatorics:monotonic_digits}.
    \item Задача \ref{exercise:combinatorics:at_least_one_vowel}.
    \item Задача \ref{exercise:combinatorics:subsets_and_binary_words}.
    \item Задача \ref{exercise:combinatorics:stairs}.
    \item Задача \ref{exercise:combinatorics:subsets_decomposition}.
    \item Задача \ref{exercise:combinatorics:queue}.
    \item Задача \ref{exercise:combinatorics:num_of_pairs}.
    \item Задача \ref{exercise:combinatorics:checkers}.
    \item Задача \ref{exercise:combinatorics:odd_even_monotonic}.
\end{enumerate}

\subsection*{Комбинаторика II. Биномиальные коэффициенты}

\subsection*{Комбинаторика III. Формула включений-исключений}

\subsection*{Бинарные отношения и их графы. Отношения эквивалентности}

\subsection*{Ориентированные графы и отношения порядка}

\subsection*{Булевы функции}

\begin{enumerate}[label=\textbf{№\arabic*}:]
    \item Задача \ref{exercise:boolean:DNF_example}.
    \item Задача \ref{exercise:boolean:preserves_0_1}.
    \item Задача \ref{exercise:boolean:graph_to_function}.
    \item Задача \ref{exercise:boolean:f_existance}.
    \item Задача \ref{exercise:boolean:Post_example}.
    \item Задача \ref{exercise:boolean:linear_closure}.
    \item Задача \ref{exercise:boolean:monotonous_and_MAJ}.
    \item Задача \ref{exercise:boolean:self_dual}.
    \item Задача \ref{exercise:boolean:not_self_dual}.
    \item Задача \ref{exercise:boolean:move_negation_to_arguments}.
\end{enumerate}


\subsection*{Производящие функции-1}

\subsection*{Производящие функции-2}
